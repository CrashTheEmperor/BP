% Soubory musí být v kódování, které je nastaveno v příkazu \usepackage[...]{inputenc}

\documentclass[%        Základní nastavení
%  draft,    				  % Testovací překlad
  12pt,       				% Velikost základního písma je 12 bodů
  a4paper,    				% Formát papíru je A4
  oneside,      			% Jednostranný tisk
    %twoside,      			% Dvoustranný tisk (kapitoly a další důležité části tedy začínají na lichých stranách)
	unicode,						% Záložky a metainformace ve výsledném  PDF budou v kódování unicode
]{report}				    	% Dokument třídy 'zpráva', vhodná pro sazbu závěrečných prací s kapitolami

\usepackage[utf8]		  %	Kódování zdrojových souborů je UTF-8
	{inputenc}					% Balíček pro nastavení kódování zdrojových souborů

\usepackage{sectsty}
	%přetypuje nadpisy všech úrovní na bezpatkové, kromě \chapter, která je přenastavena zvlášť v thesis.sty
	\allsectionsfont{\sffamily}

\usepackage{graphicx} % Balíček 'graphicx' pro vkládání obrázků
											% Nutné pro vložení logotypů školy a fakulty

\usepackage[          % Balíček 'acronym' pro sazby zkratek a symbolů
	nohyperlinks				% Nebudou tvořeny hypertextové odkazy do seznamu zkratek
]{acronym}						
											% Nutné pro použití prostředí 'acronym' balíčku 'thesis'

\usepackage[
	breaklinks=true,		% Hypertextové odkazy mohou obsahovat zalomení řádku
	hypertexnames=false % Názvy hypertext. odkazů budou tvořeny nezávisle na názvech TeXu
]{hyperref}						% Balíček 'hyperref' pro sazbu hypertextových odkazů
											% Nutné pro použití příkazu 'pdfsettings' balíčku 'thesis'

\usepackage{pdfpages} % Balíček umožňující vkládat stránky z PDF souborů
                      % Nutné při vkládání titulních listů a zadání přímo
                      % ve formátu PDF z informačního systému

\usepackage{enumitem} % Balíček pro nastavení mezerování v odrážkách
  \setlist{topsep=0pt,partopsep=0pt,noitemsep} % konkrétní nastavení

\usepackage{cmap} 		% Balíček cmap zajišťuje, že PDF vytvořené `pdflatexem' je
											% plně "prohledávatelné" a "kopírovatelné"

%\usepackage{upgreek}	% Balíček pro sazbu stojatých řeckých písmem
											%% např. stojaté pí: \uppi
											%% např. stojaté mí: \upmu (použitelné třeba v mikrometrech)
											%% pozor, grafická nekompatibilita s fonty typu Computer Modern!
                      
%\usepackage{amsmath} %balíček pro sabu náročnější matematiky                 

\usepackage{dirtree}	% sazba adresářové struktury
                      % vhodné pro prezentaci obsahu elektronické přílohy (např. CD)

\usepackage[formats]{listings}	% Balíček pro sazbu zdrojových textů
\lstset{              % nastavení
%	Definice jazyka použitého ve výpisech
%    language=[LaTeX]{TeX},	% LaTeX
%	language={Matlab},		% Matlab
	language={C},           % jazyk C
    basicstyle=\ttfamily,	% definice základního stylu písma
    tabsize=2,			% definice velikosti tabulátoru
    inputencoding=utf8,         % pro soubory uložené v kódování UTF-8
		columns=fixed,  %fixed nebo flexible,
		fontadjust=true %licovani sloupcu
    extendedchars=true,
    literate=%  definice symbolů s diakritikou
    {á}{{\'a}}1
    {č}{{\v{c}}}1
    {ď}{{\v{d}}}1
    {é}{{\'e}}1
    {ě}{{\v{e}}}1
    {í}{{\'i}}1
    {ň}{{\v{n}}}1
    {ó}{{\'o}}1
    {ř}{{\v{r}}}1
    {š}{{\v{s}}}1
    {ť}{{\v{t}}}1
    {ú}{{\'u}}1
    {ů}{{\r{u}}}1
    {ý}{{\'y}}1
    {ž}{{\v{z}}}1
    {Á}{{\'A}}1
    {Č}{{\v{C}}}1
    {Ď}{{\v{D}}}1
    {É}{{\'E}}1
    {Ě}{{\v{E}}}1
    {Í}{{\'I}}1
    {Ň}{{\v{N}}}1
    {Ó}{{\'O}}1
    {Ř}{{\v{R}}}1
    {Š}{{\v{S}}}1
    {Ť}{{\v{T}}}1
    {Ú}{{\'U}}1
    {Ů}{{\r{U}}}1
    {Ý}{{\'Y}}1
    {Ž}{{\v{Z}}}1
}
% Definice jazyka Structured Text pro listings
\lstdefinelanguage{ST}{
  keywords={FUNCTION_BLOCK, END_FUNCTION_BLOCK, VAR_INPUT, END_VAR, VAR_OUTPUT, 
            VAR_IN_OUT, VAR, VAR_TEMP, IF, THEN, ELSE, ELSIF, END_IF, CASE, OF, 
            END_CASE, FOR, TO, BY, DO, END_FOR, WHILE, END_WHILE, REPEAT, UNTIL, 
            END_REPEAT, RETURN, EXIT, BOOL, INT, DINT, REAL, TIME, TON, TOF, R_TRIG, 
            F_TRIG, RS, SR, TP, R_EDGE, F_EDGE, TRUE, FALSE, AND, OR, NOT,
            XOR, ADD, SUB, MUL, DIV, MOD, EQ, NEQ, LT, LE, GT, GE, <, <=, >, >=,
            PROGRAM, END_PROGRAM, STRING, USINT, UINT},
  sensitive=false,
  comment=[l]{//},
  morecomment=[s]{(*}{*)},
  string=[b]{'}{'},
  morestring=[b]''
}

% Definice jazyka Bash s rozšířenou podporou pro příkazy
\lstdefinelanguage{bash}{
  morekeywords={if, then, else, elif, fi, for, do, done, while, until, case, esac,
            function, select, in, break, continue, return, exit, shift, 
            source, alias, unalias, set, unset, export, local, readonly,
            declare, typeset, let, eval, exec, trap, wait, kill, getopts,
            test, command, builtin, enable, help, type, hash, sudo, su,
            mkdir, rmdir, touch, mv, cp, rm, ls, cd, pwd, echo, cat, less,
            more, head, tail, grep, egrep, fgrep, find, sed, awk, cut, sort, 
            uniq, tr, wc, diff, patch, tar, gzip, gunzip, bzip2, bunzip2, zip,
            unzip, chown, chmod, chgrp, ps, top, htop, kill, pkill, pgrep,
            lsof, netstat, ifconfig, ip, ping, traceroute, ssh, scp, rsync, curl,
            wget, apt, apt-get, dnf, yum, pacman, systemctl, journalctl, crontab,
            printf, pushd, popd},
  sensitive=false,
  morecomment=[l]{\#},
  morestring=[b]"",
  morestring=[b]'',
  morestring=[d]``,
  alsoletter={.},
  alsoother={$},
  otherkeywords={ [, ], \{, \}, |, >, <, >>, >|},
  moredelim=[s][\color{black}]{\$\{}{\}},
  literate=
    {├}{{\textcolor{gray}{├}}}{1}
    {─}{{\textcolor{gray}{─}}}{1}
    {└}{{\textcolor{gray}{└}}}{1}
    {│}{{\textcolor{gray}{│}}}{1}
}

% Definice jazyka YAML s podporou pro různé datové struktury
\lstdefinelanguage{yaml}{
  keywords={true, false, null, yes, no, on, off},
  sensitive=false,
  comment=[l]{\#},
  morecomment=[s]{/*}{*/},
  morestring=[b]{'},
  morestring=[b]{"},
  morestring=[s]{[}{]},
  morestring=[s]{\{}{\}},
  literate=
    {:}{{\textcolor{red}{:}}}{1}
    {-}{{\textcolor{blue}{-}}}{1}
    {>}{{\textcolor{blue}{>}}}{1}
    {|}{{\textcolor{blue}{|}}}{1}
    {\ -\ }{{\textcolor{blue}{\ -\ }}}{3}
    {├}{{\textcolor{gray}{\textbackslash u251C}}}{1}
    {─}{{\textcolor{gray}{\textbackslash u2500}}}{1}
    {└}{{\textcolor{gray}{\textbackslash u2514}}}{1}
    {│}{{\textcolor{gray}{\textbackslash u2502}}}{1}
}

% Definice jazyka JSON s podporou pro různé datové struktury
\lstdefinelanguage{json}{
  keywords={true, false, null},
  sensitive=false,
  comment=[l]{//},
  morecomment=[s]{/*}{*/},
  morestring=[b]{'},
  morestring=[b]{"},
  morestring=[s]{[}{]},
  morestring=[s]{\{}{\}},
  literate=
    {:}{{\textcolor{red}{:}}}{1}
    {,}{{\textcolor{blue}{,}}}{1}
    {>}{{\textcolor{blue}{>}}}{1}
    {|}{{\textcolor{blue}{|}}}{1}
    {\ -\ }{{\textcolor{blue}{\ -\ }}}{3}
}

% Definice jazyka InfluxDB (Flux)
\lstdefinelanguage{flux}{
  keywords={from, range, filter, map, reduce, yield, window, sum, mean, last, group, aggregateWindow, fill, union, join, pivot, experimental, to, by, fn, r, with, and, or, if, then, else, tables, every, createEmpty, unit, tables},
  sensitive=true,
  comment=[l]{//},
  morecomment=[s]{/*}{*/},
  morestring=[b]{'},
  morestring=[b]{"},
  literate=
    {=>}{{{\color{blue}{=>}}}}{2}
    {|}{{{\color{blue}{|}}}}{1}
    {>}{{{\color{blue}{>}}}}{1}
    {<}{{{\color{blue}{<}}}}{1}
    {=}{{{\color{blue}{=}}}}{1}
    {:}{{{\color{red}{:}}}}{1}
    {,}{{{\color{blue}{,}}}}{1}
    {[}{{{\color{blue}[}}}{1}
    {]}{{{\color{blue}]}}}{1}
    {\{}{{{\color{blue}\{}}}{1}
    {\}}{{{\color{blue}\}}}}{1}
}

% Nastavení pro adresářové struktury
\lstset{
  backgroundcolor={\color{white}},
  basicstyle=\ttfamily\small,
  columns=fullflexible,
  frame=single,
  breaklines=true,
  postbreak=\mbox{\textcolor{red}{$\hookrightarrow$}\space},
  showstringspaces=false,
  keepspaces=true,
  upquote=true,
}

%%%%%%%%%%%%%%%%%%%%%%%%%%%%%%%%%%%%%%%%%%%%%%%%%%%%%%%%%%%%%%%%%
%%%%%%      Definice informací o dokumentu             %%%%%%%%%%
%%%%%%%%%%%%%%%%%%%%%%%%%%%%%%%%%%%%%%%%%%%%%%%%%%%%%%%%%%%%%%%%%

\input{nastaveni}  % do tohoto souboru doplňte údaje o sobě, druhu práce, názvu...

%%%%%%%%%%%%%%%%%%%%%%%%%%%%%%%%%%%%%%%%%%%%%%%%%%%%%%%%%%%%%%%%%%%%%%%%

%%%%%%%%%%%%%%%%%%%%%%%%%%%%%%%%%%%%%%%%%%%%%%%%%%%%%%%%%%%%%%%%%%%%%%%%
%%%%%%     Nastavení polí ve Vlastnostech dokumentu PDF      %%%%%%%%%%%
%%%%%%%%%%%%%%%%%%%%%%%%%%%%%%%%%%%%%%%%%%%%%%%%%%%%%%%%%%%%%%%%%%%%%%%%
%% Při načteném balíčku 'hyperref' lze použít příkaz '\pdfsettings':
\pdfsettings
%  Nastavení polí je možné provést také ručně příkazem:
%\hypersetup{
%  pdftitle={Název studentské práce},    	% Pole 'Document Title'
%  pdfauthor={Autor studenstké práce},   	% Pole 'Author'
%  pdfsubject={Typ práce}, 						  	% Pole 'Subject'
%  pdfkeywords={Klíčová slova}           	% Pole 'Keywords'
%}
%%%%%%%%%%%%%%%%%%%%%%%%%%%%%%%%%%%%%%%%%%%%%%%%%%%%%%%%%%%%%%%%%%%%%%%

\pdfmapfile{=vafle.map}

%%%%%%%%%%%%%%%%%%%%%%%%%%%%%%%%%%%%%%%%%%%%%%%%%%%%%%%%%%%%%%%%%%%%%%%
%%%%%%%%%%%       Začátek dokumentu               %%%%%%%%%%%%%%%%%%%%%
%%%%%%%%%%%%%%%%%%%%%%%%%%%%%%%%%%%%%%%%%%%%%%%%%%%%%%%%%%%%%%%%%%%%%%%
\begin{document}
\pagestyle{empty} %vypnutí číslování stránek

%%% Vložení desek -- od září 2021 na žádost fakulty nepoužíváno
%\includepdf[pages=1]%  buďto generovaných informačním systémem
  %{pdf/student-desky}% název souboru nesmí obsahovat mezery!
%%% NEBO vytvoření desek z balíčku
%%\makecover
%%%
%\oddpage % při dvojstranném tisku přidá prázdnou stránku
%% kazdopádně ale:
%\setcounter{page}{1} %resetovaní čítače stránek -- desky do číslování nezahrnujeme

%% Vložení titulního listu
\includepdf[pages=1]%    buďto generovaného informačním systémem
  {pdf/student-titulka}% název souboru nesmí obsahovat mezery!
%% NEBO vytvoření titulní stránky z balíčku
%\maketitle
%%
\oddpage  % při dvojstranném tisku se přidá prázdná stránka
   
%% Vložení zadání
\includepdf[pages=1]%   buďto generovaného informačním systémem
  {pdf/student-zadani}% název souboru nesmí obsahovat mezery!
%% NEBO lze vytvořit prázdný list příkazem ze šablony
%\patternpage{}%
%	{\sffamily\Huge\centering ZDE VLOŽIT LIST ZADÁNÍ}%
%	{\sffamily\centering Z~důvodu správného číslování stránek}
%%
\oddpage% při dvojstranném tisku se přidá prázdná stránka

%% Vysázení stránky s abstraktem
\makeabstract

% Vysázení stránky s rozšířeným abstraktem
% (pokud píšete práci v češtině či slovenštině, vložení rozšířeného abstraktu zrušte;
%  pro semestrální projekt také není potřeba rozšířený abstrakt uvádět)
%\input{text/rozsireny_abstrakt}

%%% Vysázení citace práce
\makecitation

%%% Vysázení prohlášení o samostatnosti
\makedeclaration

%%% Vysázení poděkování
\makeacknowledgement

%%% Vysázení obsahu
\tableofcontents

%%% Vysázení seznamu obrázků
% (vynechejte, pokud máte dva nebo méně obrázků)
\listoffigures

%%% Vysázení seznamu tabulek
% (vynechejte, pokud máte dvě nebo méně tabulek)
\listoftables

%%% Vysázení seznamu výpisů kódu
% (vynechejte, pokud máte dva nebo méně výpisů)
\lstlistoflistings

\cleardoublepage\pagestyle{plain}   % zapnutí číslování stránek

%Pro vkládání kapitol i příloh používejte raději \include než \input
%%% Vložení souboru 'text/uvod.tex' s úvodem
\chapter*{Úvod}
\phantomsection
\addcontentsline{toc}{chapter}{Úvod}

S postupem času se rozšiřují možnosti využití elektrotechniky ve všech technických oblastech. Elektronika je dnes běžnou součástí většiny produktů, což se promítá i do moderních instalací. Oproti minulosti lze nyní ovládat nejen osvětlení, ale také topení, klimatizaci, žaluzie, zabezpečení budov, spotřebu energie a mnoho dalších funkcí. S rostoucí komplexností instalací dříve narůstalo i množství kabeláže, která sloužila převážně ke spínání jednotlivých spotřebičů. V současnosti lze tyto požadavky řešit pomocí napájecích zdrojů řízených sběrnicemi, což umožňuje automatizaci rozsáhlejších objektů, jako jsou hotely, nemocnice, vily nebo i továrny. Důležitým aspektem je také možnost vzdáleného řízení, které uživateli umožňuje ovládat celý systém odkudkoliv. Sběrnicové instalace se tak stávají stále populárnějším řešením, přičemž mezi nejrozšířenější standardy patří KNX, využívaný po celém světě.

Tato bakalářská práce si klade za cíl seznámit čtenáře se sběrnicovým systémem KNX, vybraným serverem pro řízení demonstračního panelu a tvorbou programu pro demonstraci funkcí tohoto systému. Panel bude využíván při různých akcích k prezentaci systému KNX, a to nejen prostřednictvím fyzických ovládacích prvků, ale také pomocí webového rozhraní programovatelného logického automatu nebo aplikace Home Assistant. Zároveň bude sloužit k prezentaci společností, které poskytly zařízení použité v panelu.

Teoretická část práce se zaměřuje na základy sběrnicového systému KNX, jeho historii, možnosti využití a principy fungování sběrnicových přístrojů včetně jejich adresace a komunikace. Dále je popsáno zabezpečení systému a topologie KNX. Následuje kapitola věnovaná použitému PLC, jeho knihovnám, komunikačnímu protokolu MQTT a vizualizačnímu serveru. Závěrem je popsána realizace vizualizace pomocí Raspberry Pi 5 s využitím kontejnerů (Portainer, Home Assistant, InfluxDB, Mosquitto, Grafana).

Praktická část práce se soustředí na tvorbu instalace v softwaru ETS, programování PLC a práci s Raspberry Pi. Nejprve je popsán software ETS a jeho možnosti, následně tvorba instalace od přidání zařízení do projektu, přes nastavení komunikace až po přiřazení skupinových adres. V části věnované PLC jsou popsány základní vytvořené funkční bloky, nastavení komunikace a realizace vizualizace prostřednictvím integrovaného webového serveru. Poslední kapitola se věnuje instalaci a nastavení Raspberry Pi, Dockeru a jednotlivých kontejnerů.

%%% Vložení souboru 'text/cile.tex' s úvodem
\chapter*{Cíle práce}
\phantomsection
\addcontentsline{toc}{chapter}{Cíle práce}

Cílem této bakalářské práce je seznámení s technologií KNX, její nastavení pomocí softwaru ETS, vytvoření vzdáleného ovládání a vizualizace skrze programovatelný automat a rozebrání možností jiného přístupu k ovládání a vizualizaci skrze software s otevřeným zdrojovým kódem. 
% Komentář
\chapter{Sběrnicový systém KNX}
Existuje velké množství sběrnicových systémů, ale asociace KNX s 500 členskými společnostmi a 8000 produkty je v této době největší na trhu. \cite{Asociace KNX}

Pro vstup do asociace~je nutné aby žadatel splňoval požadovanou kvalitu (kompatibilita s ISO 9001~- zavedený systém kontroly kvality v podniku), vzájemná kompatibilita výrobků s ostatními členy, konfigurační kompatibilita (možná konfigurace za použití KNX Engineering Tool Software, zkráceně ETS), zpětná kompatibilita (kompatibilita starých instalací s nynějšími a budoucími instalacemi). \cite{KNX principles}

Výhodou takto velké asociace je již zmíněná vzájemná kompatibilita komponent členských společností, tisíce KNX certifikovaných skupin výrobků (pokrytí jakéhokoliv myslitelného pole aplikací), podpora všech komunikačních médií (kroucený pár TP, powerline PL, radiofrekvenční RF a rozhraní IP/Ethernet/WLAN), použití jednoho softwaru~(ETS) na projektování a programování všech výrobků členských společností. \cite{KNX principles}

KNX je také normalizováno v Evropě, USA, Číně a mezinárodně prostřednictvím norem. Tyto normy zajišťují snadné rozšíření a výměnu instalace za novou a již zmíněnou kompatibilitu mezi společnostmi. \cite{KNX basics}

\section{Historie}
\subsection{EIBA}
Asociace byla založena v Belgii, roku 1990 pod názvem European Installation Bus Association~(EIBA) se záměrem vytvářet instalace schopné komunikace pomocí sběrnic. Jako první komunikační médium byl použitý TP a aby se zajistila kompatibilita mezi produkty se členské společnosti dohodly na používání jednoho systému (standardu). Mezi další důležité milníky patří \cite{KNX history}:
\begin{itemize}
\item 1991 - první školení EIBA
\item 1992 - první certifikované zařízení na trhu
\item 1993 - představení první verze ETS na trhu 
\item 1994 - vznikl prvních školících center
\item 1996 - vznik The Scientific Partnership (spolupráce s výzkumnými institucemi)
\item[] \hspace{0.821cm} - použití PL jako komunikační médium
\end{itemize}

\subsection{KNX}
Roku 1999 se EIBA sloučila se společností Batibus Club International (BCI) a European Home Systems Association (EHSA) a~přijaly název Konnex Association. Sídlem asociace byl ustanoven Brusel. Toto sloučení nemělo vliv na zpětnou kompatibilu a~tudíž jsou všechny nové produkty kompatibilní s produkty nesoucími logo EIB. Důležité milníky pro KNX \cite{KNX  history}:
\begin{itemize}
\item 2001 - vytvoření nového standardu KNX se základem ve standardu EIB
\item 2003 - standard schválen, jako evropská norma  EN 50090
\item 2004 - standard schválen, jako americká norma  ANSI/ASHRAE 135
\item[] \hspace{0.821cm} - přidání přenosového média RF do standardu KNX
\item 2006 - standard schválen, jako světová norma  ISO/IEC 14543-3
\item[] \hspace{0.821cm} - přejmenování asociace na KNX
\item 2007 - standard schválen, jako jedna z čínských norem GB/Z 20965
\item[] \hspace{0.821cm} - KNX IP bylo představeno jako čtvrté přenosové médium
\item 2013 - standard schválen, jako jediná čínská norma GB/T 20965
\end{itemize}

\section{Možnosti použití technologie}
Použití inteligentní instalace umožňuje využití celého objektu s maximálním potenciálem a tím maximálně ulehčit uživateli práci. Níže jsou uvedeny příklady použití instalace KNX \cite{Systemove Argumenty}:
\begin{itemize}
    \item{Centrální ovládání - Možnost ovládat celou instalaci z jednoho zařízení (např. centrální panel, mobil) odkudkoli.}
    \item Realizace centrálních funkcí - Při odchodu z domu zhasnutí světel, spuštění žaluzií, vypnutí zásuvkových obvodů, nebo naopak při vstupu zapnutí topení a osvětlení.
    \item Regulace teplot (topení, chlazení) - Regulace teploty každé místnosti zvlášť. Lze také nastavit při otevření okna vypnutí topení.
    \item Režimy nastavených teplot - Lze nastavit tepelné režimy (Ekonomický, Komfort,...), které by měly budovu chránit před přehřátím, či promrznutím. 
    \item Světelné scény -  Lze nastavit intenzitu osvětlení, která osvětlení budou svítit, případně i barvu, kterou budou zářit.
    \item Rozdělení místností na více obvodů
    \item Použití virtuálních asistentů - Je možno ovládat instalaci hlasovými povely přes virtuální asistenty (Alexa, Google Home,...)
    \item Simulace přítomnosti - Při nepřítomnosti na delší dobu lze nastavit spínání světel, které navodí dojem, že obyvatel neopustil budovu.
    \item Kontrola spotřeby energií - Lze monitorovat spotřebu energií v každém obvodu zvlášť a díky tomu omezit spotřebu, vypnout spotřebič při překročení určité hranice, nebo optimalizovat vlastní zdroje energie (fotovoltaické panely).
\end{itemize}

\section{Sběrnicové instalace}
Sběrnicová instalace je založená na koncepci ICT (Information and Communication Technology). Tři hlavní aspekty této koncepce jsou \cite{KNX principles}:
\begin{itemize}
\item Náhrada klasických spínačů tlačítkovými ovladači schopnými komunikace, nebo připojení klasických spínačů k rozhraním schopných komunikace
\item Připojení rozhraní se schopností komunikace, nebo nepřímého ovládání (spínací přístroje schopné komunikace) ke všem spotřebičům
\item Propojení veškerých přístrojů schopných komunikace kabelem určeným na bezpečné malé napětí
\end{itemize}
\subsection{Sběrnicové přístroje}
Zařízení připojené ke sběrnici se schopností komunikovat s dalšími přístroji se nazývá sběrnicovým přístrojem a je tvořeno těmito částmi (viz. Obr. \ref{fig:Součásti sběrnicového přístroje}) \cite{KNX principles}:

\begin{itemize}
\item Přenosový modul - vytváří rozhraní pro přenos informací
\item Mikrokontroler - komunikace mezi přenosovým modulem a aplikačním modulem
\item Aplikační modul - obvod tvořící přístroj\footnote{Spojení přenosového modulu a Mikrokontroleru tvoří tzv. sběrnicovou spojku (bus coupler unit BCU.}
\end{itemize}
\begin{figure}[!h]
  \begin{center}
    \includegraphics[scale=0.7]{obrazky/sbernice.png}
  \end{center}
  \caption[Součásti sběrnicového přístroje \cite{KNX principles}] {Součásti sběrnicového přístroje \cite{KNX principles}}
  \label{fig:Součásti sběrnicového přístroje}
\end{figure}
Přístroje lze ještě dělit na aktivní a pasivní. Pasivní přístroje nejsou součástí ICT, ale jedná se o podpůrné přístroje určené pro podporu procesu. Ve zkratce to znamená, že nekomunikují s ostatními přístroji. Jedním z příkladů pasivních přístrojů jsou napájecí zdroje.  Příkladem pasivních přístrojů jsou napájecí zdroje (Napájecí zdroje mohou být rozšířené ještě o ICT, ale není to časté).  
Aktivní přístroje lze rozdělit do těchto kategorií \cite{KNX principles}:
\begin{itemize}
\item Rozhraní - propojuje sběrnici a PC
\item Spojky - Optimalizují komunikaci v systému
\item Snímače - Předávají informace sběrnicovému systému
\item Akční členy - propojují klasické spotřebiče se sběrnicovým systémem
\end{itemize}

\subsection{Adresování}
\label{Adresování}
Individuální adresa je v instalaci jedinečná, tj. neexistuje další stejná adresa a požívá se k přesné identifikaci přístroje na sběrnici. Adresa je 16-bitová a je rozdělená na tři části (viz Obr. \ref{fig:Struktura individuální adresy]}).
\begin{figure}[!h]
  \begin{center}
    \includegraphics[scale=0.6]{obrazky/Adresovani.png}
  \end{center}
  \caption[Struktura individuální adresy \cite{Celkovy prehled)}]{Struktura individuální adresy \cite{Celkovy prehled}}
  \label{fig:Struktura individuální adresy]}
\end{figure}

Nastavování individuální adresy na přístroji probíhá většinou stiskem programovacího tlačítka na přístroji. Při stisknutí tlačítka se rozsvítí programovácí LED. Individuální adresa se přístroji přiděluje natrvalo. Po přidělení již ETS posílá příslušná data (aplikace, konfigurace, parametry, skupinové adresy).

Při uvedení do provozu komunikace probíhá pomocí skupinových adres. Jedná se o adresy definované programátorem pro každou funkci v systému. Celkově je možno použít 65535 adres s tím, že adresa 0/0/0 je rezervována pro tzv. broadcast (Hlášení všem přístrojům na sběrnici). Programátor si také může zvolit, kterou z uvedených struktur použije (viz Obr. \ref{fig:Příklad struktury skupinových adres}). 
\begin{figure}[!h]
  \begin{center}
    \includegraphics[scale=0.6]{obrazky/Skupinove adresovani.png}
  \end{center}
  \caption[Příklad struktury skupinových adres \cite{Celkovy prehled}]{Příklad struktury skupinových adres \cite{Celkovy prehled}}
  \label{fig:Příklad struktury skupinových adres}  
\end{figure}
Nejčastěji se využívá třístupňová struktura kvůli přehlednosti. Hlavní skupina se používá na číslo podlaží,  střední skupina na funkci (např. 1 = osvětlení, 2 = topení, 3 = stínění etc.)a podskupina pro konkrétní spotřebič, nebo skupinu spotřebičů. \cite{Celkovy prehled}
\\
\\
\\
\subsection{Komunikace}
Komunikace přístrojů na sběrnici probíhá pomocí tzv. telegramů (viz Obr. \ref{fig:Struktura telegramu}), kde je délka dat závislá na typu datového bodu  (1bit - 14bytů).
Nejdůležitější části telegramu jsou tři bloky \cite{Celkovy prehled}:
\begin{itemize}
\item Zdrojová adresa - udává adresu přístroje který telegram vyslal
\item Cílová adresa - adresa přístroje, kterému je telegram určen
\item Užitečná data - příkaz co má daný přístroj vykonat\\
\end{itemize}
\begin{figure}[!h]
  \begin{center}
    \includegraphics[scale=0.7]{obrazky/Struktura telegramu.png}
  \end{center}
  \caption[Struktura telegramu \cite{Celkovy prehled}]{Struktura telegramu \cite{Celkovy prehled}}
  \label{fig:Struktura telegramu}
\end{figure}
Telegramy na sběrnici čtou všechny přístroje, ale vykoná jej pouze přístroj určený cílovou adresou.

Komunikace na sběrnici probíhá pouze v případě, že je na sběrnici logická "1". V opačném případě je sběrnice přeplněná a pokračuje ve vysílá pouze přístroj s logickou "0" (viz Obr. \ref{fig:Struktura bitu kroucené dvojlinky}). \cite{Celkovy prehled}

\begin{figure}[!h]
  \begin{center}
    \includegraphics[scale=0.7]{obrazky/Struktura bitu.png}
  \end{center}
  \caption[Struktura bitu kroucené dvojlinky \cite{Celkovy prehled}]{Struktura bitu kroucené dvojlinky \cite{Celkovy prehled}}
  \label{fig:Struktura bitu kroucené dvojlinky}
\end{figure}

Aby jsme se vyhli kolizím a jeden z přístrojů mohl vysílat je přenos řízen principem CSMA/CA (Carrier Sense Multiple Access with Collision Avoidance vícenásobný přenos s vyhnutím se kolizím), který funguje tak, že pokud přístroj odesílající logickou “1” detekuje logickou “0”, aby se uvolnila cesta pro přenos jinému přístroji. Přístroj s přerušeným přenosem sleduje provoz na sběrnici a vyčká do konce přenosu jiného zařízení a poté zkusí znova vysílat. \cite{Celkovy prehled}

\subsection{Datový bod}
Datové typy byly standardizovány za účelem zajištění kompatibility podobných přístrojů od různých výrobců. Jedná se například o stmívání, žaluzie a hodiny. Standardizace zahrnuje požadavky na formát dat a strukturu komunikačních objektů snímačů a akčních členů. I tak existuje více druhů datových bodů (DPT) se stejnou funkcionalitou. Kombinace různých typů DPT se nazývají funkčními bloky.\cite{Celkovy prehled}
\\Skutečná informace datového bodu:
\begin{itemize}
    \item Není uložena v paměti zařízení.
    \item Není nikdy součástí telegramu
    \item Je pouze v projektu ETS\\
\end{itemize}
Typy datových bodů jsou zvláště důležité pro diagnostiku to znamená, že umožňují ETS monitorovat data spojená se skupinovými objekty, např. místo "data = 85 A8" je zobrazeno "data = -6 °C". \cite{Datapoint}\\
\\Struktura datového bodu a notace \cite{Datapoint}:
\begin{itemize}
    \item Datový typ : formát + kódování
    \item Velikost: rozsah hodnot + jednotky
\end{itemize}
Notace datového bodu se píše ve tvaru X.YYY, neboli DATOVÝ TYP.VELIKOST.

\begin{table}[h]
 \caption[Přehled nejpoužívanějších datových typů]{Přehled nejpoužívanějších datových typů}
   \small
    \centering
	  \begin{tabular}{|c|c|c|}
	    \hline
	    Značení & Formát  & Funkce  \\
	    \hline\hline
	    1.yyy & boolean & přepínání (001), krok (007),... \\
	    \hline
	    3.yyy & boolean + 3-bit unsigned & stmívání \\
	    \hline
	    5.yyy & 8-bit unsigned + 3-bit unsigned & stmívání(0-100\%), pozice rolet(0-100\%)\\
	    \hline
	    7.yyy & boolean + 3-bit unsigned & čitač pulsů \\
	    \hline
	    9.yyy & 16-bit float & přenos hodnoty teploty, jasu, rychlost větru \\
	    \hline
	    14.yyy & 32-bit float & nastavení teploty\\
	    \hline
	    19.yyy & čas + data & výstupy obrazovek \\
	    \hline
	    20.yyy & 8-bit enumerace & Topení, chlazení a ventilace ('komfort',...) \\
	    \hline
	  \end{tabular}
\end{table}

Díky existenci datového bodu jsme schopní nastavit hodnotu osvětlení 3 různými způsoby \cite{Systemove Argumenty}:
\begin{itemize}
    \item Zapnutí/Vypnutí
    \item Krokové stmívání - Při poslání telegramu "start stmívání" osvětlení krokově roste o definovanou hodnotu. Po poslání "stop stmívání" hodnota neroste.
    \item Procentuální stmívání- Realizuje se pomocí cyklického posílání telegramu. Při každém přijetí telegramu se zvedne jas o nastavenou hodnotu.
\end{itemize}

\section{Zabezpečení}
Rozdíl mezi zařízeními KNX a zabezpečenými KNX Secure je ten, že zařízení KNX Secure jsou schopna šifrovat a dešifrovat telegramy. Tato technologie dodává instalaci extra zabezpečení, a to během uvádění instalace do provozu, tak i poté za běhu. Telegramy jsou zašifrované zabezpečenými zařízeními KNX se nazývají zabezpečené telegramy.
\\Lze rozlišit dva typy šifrovaných telegramů KNX \cite{KNX Secure}:
\begin{itemize}
\item{Zcela zašifrované}
\begin{itemize}
\item{Lze použít pouze na zařízeních KNX IP a je označováno jako KNX IP Secure.}
\item{Používá se, pro zabezpečení části intalace, která je vystavená externí IP síti (typicky se jedná o páteřní linku).}
\end{itemize}
\item{Čatečně zašifrované}
\begin{itemize}
\item{Lze požít na libovolné komunikační zařízní KNX. Zařízení používající tento typ zabezpečení se nazývají KNX Data Secure}
\item{Toto šifrování můžeme použít i pro KNX IP, ale pouze pro tu část instalace, která není vystavena externí IP síti.}
\end{itemize}
\end{itemize}
Oba typy zabezpečení obsahují MAC (Message Authentication Code).
\\Zabezpečená zařízení mají zabezpečený režim, který je v projektu ETS reprezentován vlastností nazvanou „Secure Commissioning“. Pouze když je tento režim aktivován, zařízení je schopno šifrovat a dešifrovat telegramy.

Zabezpečená zařízení mají tzv. "Tool Key". V moment, když je aktivován zabezpečený režim zařízení, je ETS schopen komunikovat s tímto zařízením pouze pokud zná Tool Key tohoto zařízení.

Zabezpečená zařízení obsahují také Factory Default Setup Key (FDSK). FDSK je jedinečný pro každé zařízení a nelze jej upravovat ani mazat. ETS tento klíč může načíst jenom pomocí certifikátu (25znakový kód, který obsahuje sériové číslo a FDSK). Tool Key je v zásadě z výroby nastaven na FDSK. Tool Key může být také zpětně nastaven na FDSK pomocí tzv. "master resetu", který uvadí výrobce. 

Po přidání zabezpečeného zařízení KNX  do ETS a po přidání jeho certifikátu, ETS automaticky nastaví svůj Tool Key v projektu. To znamená, že uživatel ETS nemůže definovat/upravit Tool Key ručně, Tool Key také není viditelný pro uživatele ETS. \cite{KNX Secure}

\section{Topologie}
Základním kamenem topologie je hlavní linie, na kterou lze připojit až 256 přístrojů (účastníků sběrnice - US). Tato linie lze rozdělit až na 15 dalších segmentů za použití liniových opakovačů/spojek (LS). Na takto vzniklé segmenty (linie) připojit dalších 256 US. To vše ovšem závisí také na spotřebě přístrojů použitých v instalaci. To znamená, že celková spotřeba všech přístrojů nesmí překročit jmenovitý proud na druhé straně sběrnicového zdroje, který každá linie musí mít vlastní. Také lze mít maximálně 4 000 US na celé topologii. Toto množství lze také navýšit za použití oblastní spojky (OS) díky na páteřní linii. Po připojení vznikne tzv. nadřazená páteřní linie, která může pojmout až 16 oblastních spojek a celek rozdělí na dílčí páteřní linie. Celkový počet US na takovéto linii může být až 61 000. Reálné množství je v tomto případě omezeno zdrojem s tlumivkou (NZ/TI). \cite{Topologie}\\\\
Pro sběrnici KNX lze použít pouze tyto struktury kabeláže:
\begin{itemize}
    \item Hvězdicová
     \item Liniová
     \item Stromová
     \item Kombinace výše uvedených\\ \\ \\
\end{itemize}
\begin{figure}[!h]
  \begin{center}
    \includegraphics[scale=0.6]{obrazky/Ukazka topologie.png}
  \end{center}
  \caption[Ukázka topologie KNX\cite{Topologie}]{Ukázka topologie KNX\cite{Topologie}}
  \label{fig:Ukázka topologie KNX}
\end{figure}

\newpage
\subsection{Individuální adresa}
Individuální adresa se nastavuje s ohledem na umístění v topologii (Viz. Podkapitola \ref{Adresování}).
\begin{table}[h]
 \caption[Individuální adresy v topologii \cite{Topologie}]{Individuální adresy v topologii \cite{Topologie}}
   \small
    \centering
	  \begin{tabular}{|c|c|c|}
	    \hline
	    Prvek & Adresa & Funkce  \\
	    \hline\hline
	    Oblast & 0 & adresuje účastníky v páteřní linii \\
	    \hline
	    Oblast & 1...15 & adresuje oblasti \\
	    \hline
	    Linie & 0 & adresuje hlavní linii příslušné oblasti \\
	    \hline
	    Linie & 1...15 & adresuje linie obsažené v oblasti\\ 
	    \hline
	    Účastník na sběrnici & 0 & adresuje liniovou spojku příslušné linie \\
	    \hline
	    Účastník na sběrnici & 1...255 & adresuje sběrnicové přístroje obsažené v linii \\
	    \hline
	  \end{tabular}
\end{table}
\newpage
\subsection{Spojka}
V případě,že jsou v instalaci použity spojky a mají přiřazeny správné individuální adresy, budou při projektování v programu ETS (Kapitola \ref{ETS}) automaticky vytvořeny filtrační tabulky jednotlivých spojek. Filtrační tabulka obsahuje skupinové adresy, které smí projít skrz příslušnou spojku (obsahuje všechny obsažené skupinové adresy, které adresují SU umístěné za spojkou). Tudíž každá linie pracuje nezávisle.

 Spojky jsou vytvořeny pro montáž na DIN lištu, kde se připojují primární i sekundární linie pomocí sběrnicové svorkovnice. Primární linie také funguje, jako napájení mikrokontroleru a při výpadku sítě ohlásí tuto skutečnost na sekundární linii. Jednou z výhod pojky je možnost programování z obou linií. Obsahují také žluté signalizující LED, které blikají pouze v případě, že spojka propustí telegram na příslušnou linii. Další vlastností spojky je galvanické oddělení mezi primární a sekundární linií. Poslední vlastností spojky je možnost přeměny na liniový opakovač. Opakovač se rozliší od spojky absencí nuly na konci individuální adresy (X.X.1 apod.). Využívá se pro rozšíření linie o další segment s 64 US. Tento úsek je limitován délkou kabelu, který může měřit maximálně 1000m. \cite{Topologie}
 
\subsection{Routingové číslo}
Každý telegram, který je vyslán přístrojem na obsahuje routingové číslo, které začíná na hodnotě 6. Toto číslo při každém průchodu spojkou, či opakovačem se dekrementuje dokud nedosáhne nulové hodnoty. Tuto vlastnost berou filtrační tabulky v potaz.
Pokud se jedná o servisní telegram, tak routingové číslo má hodnotu 7, která se při průchodu spojkou nedekrementuje. \footnote{Spojky vyrobeny po roce 2019 mají schopnost tuto hodnotu dekrementovat}. Tuto skutečnost berou v potaz i filtrační tabulky, které toto číslo ignorují, a tudíž všechny spojky tento telegram propustí. Tento telegram se vždy dostane k požadovanému účastníku bez ohledu na umístění.
Toto číslo také brání zasmyčkování (nekonečnému kolování) telegramu. \cite{Topologie}
\subsection{Interní a externí rozhraní}
Systém KNX je otevřený jiným systémům za použití vhodných rozhraní umístěných na libovolné linii (většinou se jedná o páteřní linii). Lze připojit například programovatelný logický automat (PLC), digitální síť integrovaných služeb (ISDN), systémová technika budov, internet a mnohé další.
Tato rozhraní přenáší obousměrně zprávy, které převede na komunikační protokol.

Nejedná se ovšem jenom o spojovaní KNX s externími médii, ale je možno spojit různá KNX média mezi sebou (např. spojení TP a RF). Existuje také možnost připojení částí instalace skrze optická vlákna. Tohle spojení přináší řadu výhod zejména galvanické oddělení celků a zvýšení celkové délky vedení. \cite{Topologie}

\chapter{ETS}
\label{ETS}
Jedná se o konfigurační softwarový nástroj nezávislý na výrobci pro navrhování a konfiguraci inteligentních instalací a pro řízení budov pomocí systému KNX. Tento software funguje pouze na počítačových platformách využívajících operační systém Windows. \cite{ETS Kecy}.

Pomocí softwaru lze \cite{Mitrenga}:
\begin{itemize}
    \item Vkládat katalogové produkty do projektu - Produkty schválené asociací jsou obsaženy v katalogu a lze je použít v projektu. Produkty lze také přidat manuálně prostřednictvím aplikačních programů s koncovkou ".knxprod".
    \item Vytvořit architekturu objektu - rozdělit objekt na celky(budovy, patra, místnosti,...)
    \item Parametrizace produktů
    \item Vytváření skupinových adres
    \item Nahrávání řešení projektu do přístrojů
    \item Vzdálené ovládání připojeného projektu
    \item Diagnostika
    \item Vytvoření dokumentace
\end{itemize}

\section{Tvorba instalace}
Při vytváření projektu byl zvolen typ páteřní linie na IP, skupinové adresy na třístupňové a topologie zvolena jakožto TP, která byla použita, při tvorbě panelu.

Po úspěšném založení projektu se program přepnul do pracovní části, která je složená z osmi oken \cite{Mitrenga}:
\begin{itemize}
    \item Budovy - Rozdělení objektu na celky
    \item Skupinové adresy - Vytvoření a přiřazení skupinových adres přístrojům
    \item Topologie - Zobrazení rozložení vytvořeného projektu v topologii
    \item Kořeny projektu - Zobrazení všech oken kde se pracovalo
    \item Přístroje - Seznam přístojů v projektu
    \item Zprávy - Okno zaměřené na tvorbu dokumentace projektu
    \item Katalog - Vyhledání a vložení produktů do projektu
    \item Diagnostika - Okno určené pro otestování instalace\\
\end{itemize}

Pro vytvoření pracovního prostoru bylo použito okno budova. Prostor byl pojmenován Demonstrativní panel a byl rozdělen na 5 celků. Tyto celky reprezentují pokoje zobrazené na panelu (vchod, kuchyň, koupelna, obývací pokoj a rozvaděč umístěný v zadní části panelu). Tohle rozdělení bylo vytvořeno čistě pro zvýšení přehlednosti objektu a následné ulehčení propojování přístrojů mezi sebou. Je nutno také dodat, že vytvoření jedné místnosti je podmínkou pro vkládání přístrojů do pracovní plochy.

Pro vložení přístrojů bylo nutno otevřít okno katalog, který ovšem neobsahoval použité přístroje. Díky této komplikaci bylo nutno navštívit webové stránky výrobců a následné stažení aplikačních programů. Tyto programy byly importovány do katalogu pomocí tlačítka "Import...". Vzhled projektu po přidání přístrojů lze vidět na Obr. \ref{fig:Projekt budovy v ETS}.

\begin{figure}[!h]
  \begin{center}
    \includegraphics[scale=0.7]{obrazky/Budova.png}
  \end{center}
  \caption[Projekt budovy v ETS]{Projekt budovy v ETS}
  \label{fig:Projekt budovy v ETS}
\end{figure}

Po přidání všech přístrojů se zobrazila pracovní plocha, která slouží k zobrazení přehledu všech přístrojů (Zabezpečení - KNX Secure, individuální adresa prvku, místnost v projektu, použitý aplikační program, stav přístroje - nahrána adresa, program, parametrizace, skupinová adresa a informace o produktu). V sloupcích vyjadřujících stav přístroje jsou většinově pomlčky, které znázorňují, že nebyly nahrány všechny části do přístrojů. Tahle skutečnost je zdůvodněná změnami parametrů a skupinových adres.

\begin{figure}[!ht]
  \begin{center}
    \includegraphics[scale=0.5]{obrazky/Přístroje v ETS.png}
  \end{center}
  \caption[Pracovní plocha v ETS]{Pracovní plocha v ETS}
  \label{fig:Pracovní plocha v ETS}
\end{figure}

\section{Parametrizace tlačítek a detektoru pohybu}
V této podkapitole bude vysvětleno parametrizování použitých tlačítek. Ty byla pomyslně rozdělená do místností a nastaveny, tak aby spolupracovaly s nejbližšími prvky (světly, žaluziemi, klimatizací a topením), které jsou zobrazeny na Obr. \ref{}. Pro vysvětlení byly vytvořeny tabulky popisu funkcí jednotlivých tlačítek. 

\begin{figure}[!h]
  \begin{center}
    \includegraphics[scale=0.7]{obrazky/Panel_vzhled.png}
  \end{center}
  \caption[Grafický návrh panelu \cite{Mitrenga}]{Grafický návrh panelu \cite{Mitrenga}}
  \label{fig:Vzheled panelu}
\end{figure}

\subsection{ABB - SBR/U6.0.1-84}
Jedná se o šestinásobné tlačítko se zabudovaným termostatem, které lze použít na regulaci teploty, ovládání žaluzií, ovládání osvětlení a nastavení dvou scén, které mohou obsahovat až osm objektů. Každé stisknutí tlačítka změní barvu signalizační LED na předem stanovenou hodnotu (rozpoznání zapnuto/vypnuto). \cite{ABB}

\begin{figure}[!ht]
  \begin{center}
    \includegraphics[scale=1.6]{obrazky/ABB.jpg}
  \end{center}
  \caption[Šestinásobné tlačítko s termostatem ABB - SBR/U6.0.1-84 \cite{ABB}]{Šestinásobné tlačítko s termostatem ABB - SBR/U6.0.1-84 \cite{ABB}}
  \label{fig:Šestinásobné tlačítko s termostatem ABB SBR/U6.0.1-84}
\end{figure}
Tlačítko bylo nastaveno na odesílání aktuální hodnoty teploty co deset minut. Tlačítka jsou rozložena po horizontálních párech s označením funkční blok 1 až 3. V záložce každého bloku byly nastaveny obě tlačítka na krátká a dlouhá stisknutí. V záložkách  \textit{Common parameter} byl vybrán typ objektu na 1-bit. Při krátkém stisknutí tlačítko odesílá hodnotu jedna, při dlouhém stisknutí posílá hodnotu 2. Následně v záložce \textit{Extended parameters} byly nastaveny hodnoty odesílaných objektů u dlouhého stisknutí na on ("1") a u krátkého na off ("0").
V záložkách \textit{LED Button} pro každý funkční blok byla každá dioda nastavena do modu status. Přijímaný objekt byl nastaven na 1-bit a hodnota jasu na \textit{bright} nastavena signalizační barva LED diod na bílou při vypnutí a červenou při zapnutí. 
\subsection{Berker - 75663593}
Osminásobné člačítko s termostatem by mělo být schopno regulovat pokojovou teplotu, ovládat žaluzie, ovládat osvětlení a scény. V této práci se ovšem nepovedlo nastavit ani při použití více zařízení a softwaru od Berkeru, který dokázal otevřít externí okno parametrizace v německém jazyce. Po ukončení parametrizace se parametry neuloží. \cite{Berker}
\begin{figure}[!ht]
  \begin{center}
    \includegraphics[scale=1.3]{obrazky/Berker.jpg}
  \end{center}
  \caption[Osminásobné tlačítko Berker - 75663593 \cite{Berker}]{Osminásobné tlačítko - Berker - 75663593 \cite{Berker}}
  \label{fig:Osminásobné člačítko s termostatem Berker 75663593}
\end{figure}
\newpage
\subsection{Ekinex - EK-ED2-TP-RW}
Jedná se o čtyřnásobné tlačítko se zabudovaným teplotním senzorem pro ovládání žaluzií, osvětlení a scén. \cite{Ekinex}

\begin{figure}[!ht]
  \begin{center}
    \includegraphics[scale=0.6]{obrazky/Ekinex.png}
  \end{center}
  \caption[Čtyřnásobné tlačítko Ekinex - EK-ED2-TP-RW \cite{Mitrenga}]{Čtyřnásobné tlačítko Ekinex - EK-ED2-TP-RW \cite{Mitrenga}}
  \label{fig:Čtyřnásobné tlačítko se zabudovaným teplotním senzorem Ekinex - EK-ED2-TP-RW}
\end{figure}

Tlačítko bylo nastaveno v záložce \textit{General} na dvě svislé klapky. Obě klapky byly nastaveny na dlouhé a krátké stisknutí. V případě první klapky se horní krátký stisk nastavil na funkci \textit{toggle} (přepínání). Dlouhý stisk představuje funkci \textit{off}. Pro dolní část klapky, je nastavení přesně opačné. Druhá klapka je nastavená stejným způsobem, akorát místo funkce \textit{toggle} byla použita funkce \textit{none}. Tato funkce zasílá "0", která znamená u žaluzií pohyb směrem nahorů.

\subsection{Basalte - Senido 202-03}
Další z použitých snímačů je čtyřnásobné dotykové tlačítko se zabudovaným snímačem teploty pro ovládání žaluzií, ovládání osvětlení a scén se schopností rozlišovat krátké a dlouhé stisknutí, a to nejen na jednom segmentu, ale má možnost snímat více segmentů najednou (multitouch). Dokáže ovládat, až šest scén s osmi objekty. Další ze schopností tlačítka je posílání tříbajtové hodnoty RGB. Poslední z funkcí tlačítka je zobrazování statusu díky RGB podsvícení. \cite{Basalte}

\begin{figure}[!ht]
  \begin{center}
    \includegraphics[scale=0.4]{obrazky/Basalte.jpg}
  \end{center}
  \caption[Čtyřnásobné dotykové tlačítko Basalte - Senido 202-03 \cite{Basalte}]{Čtyřnásobné dotykové tlačítko Basalte - Senido 202-03 \cite{Basalte}}
  \label{fig:Čtyřnásobné dotykové tlačítko Basalte - Senido 202-03}
\end{figure}

Tlačítko bylo nastaveno v záložce \textit{General} na čtyři různá tlačítka. Dále se v této záložce povolila funkce řadiče scén. První trojici tlačítek, byla nastavena scéna, kterou při stisknutí budou volat.  Každá z těchto scén byla nastavena v korespondující záložce označené číslem. Poslední tlačítko bylo nastaveno na demonstraci schopnosti zasílat hodnoty RGB. Jedná se o 2 nastavené hodnoty, které se rozlišují délkou stisku. Pro demonstraci funkce multitouch byly vybrána funkce \textit{room toggle + General on/off/scene}. Pro krátký stisk byla vybrána scéna, která se zapne při krátkém stisku. Při druhém stisku se panel vypne. Dlouhý stisk má přiřazenou vlastní scénu. V záložce \textit{Temperature senzor} bylo nastaveno, aby senzor zasílal teplotu každých 5 minut. Záložka \textit{Scene controller} určiná pro nastavení řadiče scén, byla nastavena na všech osmi výstupech na hodnotu 1-bit.

\subsection{Simon - 8400100-039}
Čtyřnásobné tlačítko se zabudovaným RGB podsvícením a teplotním senzorem pro ovládání žaluzií a osvětlení. \cite{Simon}

\begin{figure}[!ht]
  \begin{center}
    \includegraphics[scale=0.4]{obrazky/Simon.png}
  \end{center}
  \caption[Čtyřnásobné tlačítko Simon - 8400100-039 \cite{Simon}]{Čtyřnásobné tlačítko Simon - 8400100-039 \cite{Simon}}
  \label{fig:Čtyřnásobné tlačítko Simon - 8400100-039}
\end{figure}

V záložce \textit{General} bylo vybráno 4 tlačítkové provedení, které je použito na demonstrativním panelu. Jako další možnost, která byla povolena byl vnitřní senzor teploty. Poté v záložce \textit{FeedBack} byly nastaveny hodnoty jasu a hlasitosti na maximum. Dále zde byla aktivována možnost zapnout vibrace při doteku. Pro nastavení samotné funkcionality tlačítek se musela použít záložka \textit{Inputs}, kde se nastavilo oddělení tlačítek od sebe (všechna tlačítka jsou samostatně). Tlačítka v tomto případě jsou číslována od spodního levého rohu po sloupcích (1 a 2 levá strana, 3 a 4 pravá strana). Poté už se nastavovala samotná funkcionalita tlačítek. Byla vybrána možnost krátkého i dlouhého stisku. V případě krátkého stisku žaluzie vyjedou/sjedou samostatně. Dlouhý stisk znamená pohyb pouze v čase, kdy je tlačítko stisknuto.

\subsection{HDL - M/TBP6.1-A2}
Předposlední tlačítko je od společnosti HDL. Jedná se o šestinásobné dotykové tlačítko Čtyřnásobné tlačítko se zabudovaným RGB podsvícením pro ovládání žaluzií, osvětlení, stmívání a ovládání dvou scén s deseti objekty. Dále také obsahuje RGB kontrolér, který dokáže posílat 3-byte hodnotu obsahující informace o intenzitě každé složky. \cite{HDL}

\begin{figure}[!ht]
  \begin{center}
    \includegraphics[scale=0.25]{obrazky/HDL.jpg}
  \end{center}
  \caption[Šestinásobné tlačítko HDL - M/TBP6.1-A2 \cite{HDL}]{Šestinásobné tlačítko HDL - M/TBP6.1-A2 \cite{HDL}}
  \label{fig:Šestinásobné tlačítko HDL - M/TBP6.1-A2}
\end{figure}

\newpage První z parametrů, které je možno nastavit v záložce \textit{General} byla citlivost dotyku, a to na hodnotu 4. Dále se pak povolily scény. Následně se obě scény nastaví v záložkáckách \textit{Panel scene A} a \textit{Panel scene B}. První scéna byla nastavena na dle Obr. \ref{fig:Parametry scény A tlačítka HDL - M/TBP6.1-A2} a druhá dle Obr. \ref{fig:Parametry scény B tlačítka HDL - M/TBP6.1-A2}.

\begin{figure}[!ht]
  \begin{center}
    \includegraphics[scale=0.4]{obrazky/Scena A.png}
  \end{center}
  \caption[Parametry scény A tlačítka HDL - M/TBP6.1-A2]{Parametry scény A tlačítka HDL - M/TBP6.1-A2}
  \label{fig:Parametry scény A tlačítka HDL - M/TBP6.1-A2}
\end{figure}

\begin{figure}[!ht]
  \begin{center}
    \includegraphics[scale=0.4]{obrazky/Scena B.png}
  \end{center}
  \caption[Parametry scény B tlačítka HDL - M/TBP6.1-A2]{Parametry scény B tlačítka HDL - M/TBP6.1-A2}
  \label{fig:Parametry scény B tlačítka HDL - M/TBP6.1-A2}
\end{figure}

Při nastavování tlačítek bylo nutno nastavit krátké a dlouhé stisknutí tlačítek. Krátkému stisknutí byla přiřazena funkce \textit{toggle}, která dovoluje přepínat osvětlení mezi hodnotami zapnuto a vypnuto. Dlouhé stisknutí bylo nastaveno na dobu 1s a používá se na stmívání. Pro demonstraci stmívání byly nastaveny různé hodnoty kroku prvních 4 tlačítek. Každé z těchto tlačítek má nastavenou signaliční podsvícení na jinou hodnotu. První tlačítko bylo nastaveno červenou barvu, druhou na zelenou, třetí na modrou a čtvrté na bílou. Při signalizaci se zvýší jas barev o 70\%. Zbylá 2 tlačítka byla přepnuta do modu RGB kontrolér, který odesílají hodnotu RGB jak pro krátké, tak i pro dlouhé stisknutí. Tato hodnota se také signalizuje při stisknutí tlačítek. 

\subsection{Siemens - QMX3.P37}
Jedná se o ovládací panel určený na regulování pokojové teploty s integrovaným displejem. Tento displej dokáže zobrazovat vlhkost vzduchu, koncentraci CO2 v ovzduší a samotnou teplotu místnosti. Také obsahuje osm tlačítek, která obsahují žluté statusové LED. Tento panel umožňuje také ovládání žaluzií, osvětlení a scény. \cite{Siemens}

\begin{figure}[!ht]
  \begin{center}
    \includegraphics[scale=0.125]{obrazky/Siemens.jpg}
  \end{center}
  \caption[Ovladací panel Siemens - QMX3.P37 \cite{Siemens}]{Ovladací panel Siemens - QMX3.P37 \cite{Siemens}}
  \label{fig:Ovladací panel Siemens - QMX3.P37}
\end{figure}

V tomto případě bylo zařízení nastaveno na spínání pomocí jednoho tlačítka. Nejprve v záložce \textit{General} byla nastavena hodnota svitu signaližačnních LED na 100\% hodnotu. V další záložce byl nastaven teplotní senzor na odesílání hodnoty každých 10 minut. Poté se už nastavovaly jednotlivé tlačítkové páry. Funkce páru byla zvolena \textit{Individual}, která umožňonila nezávislé funguvání obou tlačítek. Dále se u obou tlačítek nastavila možnost \textit{1 - button switching / send value}, \textit{Short/long press} (dlouhé stisknutí po uplynutí 0,5s) a vybrala se možnost odesílání druhé hodnoty při dlouhém stisku. Levým tlačítkům byla přiřazena hodnota \textit{on} a pravým \textit{off}. Také byla nastavena signalizace stisku tlačítek. Kvůli tomu byla možnost \textit{LED display} nastavena na \textit{status object} a možnost \textit{LED activation} pro levá tlačítka na \textit{0 = LED off; 1 = LED on}. Pravá tlačítka byla nastavena \textit{0 = LED on; 1 = LED off}. Po pozdější úvaze o zefektivnění panelu se pro 1. a 4. řadu tlačítek změnil dlouhý stisk na \textit{Toggle}.  

\subsection{B.E.G - Indor 140-L-KNX-DX}
Poslední snímač, který se použil na panelu je detektor přítomnosti s teplotním senzorem a dvěmi tlačítky. \cite{BEG}

Parametrizace tohoto prvku byla celá v němčině, a to dosti zkomplikovalo postup. První záložka \textit{Grundeinstellungen} (Základní nastavení) se nastavila hodnota teploměru (Temperturmessung) na aktiviertz. Poté v záložce teploměru se v možností zasílání teploty (\textit{Temperaturwer senden}) zvolilo odesílání při změně (bei Änderung). Další parametry byly nastaveny v záložce \textit{Tastenfunktionen} (Klíčové funkce), kde se aktivovala tlačítka T1 a T2. Nastavení \textit{Präsenzmelder} (Detektoru pohybu) zůstal v plně automatickém řežimu (\textit{Vollautomatik}). V první podzáložce detektoru pohybu byla nastavená doba vypnutí na 30 sekund. Při nastavování obou tlačítek byl vybrán režim spínání (\textit{Betriebsart}).

\begin{figure}[!ht]
  \begin{center}
    \includegraphics[scale=0.45]{obrazky/BEG.png}
  \end{center}
  \caption[Detektor pohybu B.E.G - Indor 140-L-KNX-DX \cite{Mitrenga}]{Detektor pohybu B.E.G - Indor 140-L-KNX-DX  \cite{Mitrenga}}
  \label{fig:Detektor pohybu B.E.G - Indor 140-L-KNX-DX }
\end{figure}

\section{Parametrizace akčních členů}
Tahle podkapitola se zaměřuje na parametrizaci použitých akčních členů umístěných v rozvaděči na zadní straně panelu.
\subsection{ABB SA/S8.10.2.1}
Tento osmikanálový spínací člen nebyl nijak parametrizován za účelem dosáhnutí ponechání ve stavu spínacího aktoru, který zasílá status pouze při změně. Úkolem tohoto aktoru je spínání LED představující topení (šest) a klimatizaci (dvě).

\begin{figure}[!ht]
  \begin{center}
    \includegraphics[scale=0.25]{obrazky/ABB aktor1.jpg}
  \end{center}
  \caption[Osmikanálový spínací člen ABB - SA/S8.10.2.1 \cite{ABB aktor1}]{Osmikanálový spínací člen ABB - SA/S8.10.2.1  \cite{ABB aktor1}}
  \label{fig:Osmikanálový spínací člen ABB - SA/S8.10.2.1}
\end{figure}

\subsection{ABB - JRA/S4.230.2.1}
Jedná se o čtyřkanálový žaluziový člen, který je určen na ovládaní, žaluzií. Jelikož se v projektu používají pouze 2 žaluziové okruhy, tak se využívá pouze polovina akčního členu. Využívají se první dva kanály. Jediná změna od původní parametrizace je v záložkách \textit{Drive} pro jednotlivé kanály a to nastavení ukončení pohybu po 5 sekundách (tj. žaluzie může po stisku vyjíždět/sjíždět po dobu maximálně 5 sekund). 

\begin{figure}[!ht]
  \begin{center}
    \includegraphics[scale=0.25]{obrazky/ABB aktor2.jpg}
  \end{center}
  \caption[Čtyřkanálový žaluziový člen ABB - JRA/S4.230.2.1 \cite{ABB aktor2}]{Čtyřkanálový žaluziový člen ABB - JRA/S4.230.2.1  \cite{ABB aktor2}}
  \label{fig:Čtyřkanálový žaluziový člen ABB - JRA/S4.230.2.1}
\end{figure}

\subsection{HDL - M/R8.10.1}
Osmikanálový spínací člen HDL se v této práci využívá, na spínání osvětlení respektive 7 LED, které představují osvětlení umístěné v domě. V případě tohoto členu nebyla nutná žádná změna oproti původnímu nastavení parametrů. Všechny kanály jsou nastaveny, jako spínací aktor s typem kontaktu Normally Opened (NO). Zasílání statusu probíhá pouze při změně hodnoty.

\begin{figure}[!ht]
  \begin{center}
    \includegraphics[scale=0.35]{obrazky/HLD aktor1.jpg}
  \end{center}
  \caption[Osmikanálový spínací člen HDL - M/R8.10.1 \cite{HDL aktor1}]{Osmikanálový spínací člen HDL - M/R8.10.1  \cite{HDL aktor1}}
  \label{fig:Osmikanálový spínací člen HDL - M/R8.10.1}
\end{figure}

\subsection{HDL - M/DRGBW4.1}
Čtyřnásobný stmívací člen poskytnutý společností HDL, byl v této práci použit na ovládání RGBW LED pásku ukrytém v demonstrativním panelu. Výhodou tohoto členu je možnost ovládat kanál barevných složek zvlášť. Každý kanál (\textit{Channel}) je nastaven na odesílání stavové hodnoty (1bit) při změně. Dále se nastavily hodnoty času pro stmívání v záložkách \textit{dimming config} každého kanálu na 1 sekundu pro zapnutí i vypnutí. 

\begin{figure}[!ht]
  \begin{center}
    \includegraphics[scale=0.4]{obrazky/HLD aktor2.jpg}
  \end{center}
  \caption[Čtyřnásobný stmívací člen HDL - M/DRGBW4.1 \cite{HDL aktor2}]{Čtyřnásobný stmívací člen HDL - M/DRGBW4.1 \cite{HDL aktor2}}
  \label{fig:Čtyřnásobný stmívací člen HDL - M/DRGBW4.1}
\end{figure}

\section{Připojená komunikační rozhraní}
Pro umožnění parametrizace a externího řízení bylo nutno přidat do projektu dvě různá rozhraní pro komunikaci. Ani jedno z těchto rozhraní nebylo nijak parametrizováno a bylo ponecháno v původním stavu. První z nich je IP Secure router Siemens - 5WG1 146-1AB03, který je převážně určen k bezpečnému přenosu dat. Lze z něj také využít jako liniová spojka.. \cite{Siemens IP}

\begin{figure}[!ht]
  \begin{center}
    \includegraphics[scale=0.55]{obrazky/Siemens router.jpg}
  \end{center}
  \caption[IP Secure router Siemens - 5WG1 146-1AB03 \cite{Siemens IP}]{IP Secure router Siemens - 5WG1 146-1AB03 \cite{Siemens IP}}
  \label{fig:IP Secure router Siemens - 5WG1 146-1AB03}
\end{figure}

Druhé komunikační rozhraní použité na panelu je Weinzierl - KNX IP BAOS 774. Využívá se za účelem komunikace skrze telegramy, nebo datové body. Dále umožňuje přístup k objektům pomocí TCP/IP protokolu anebo za pomoci webového rozhraní. \cite{Weinzier}

\begin{figure}[!ht]
  \begin{center}
    \includegraphics[scale=0.2]{obrazky/IP BAOS.jpg}
  \end{center}
  \caption[Komunikační rozhraní Weinzierl - KNX IP BAOS 774 \cite{Weinzier ob}]{Komunikační rozhraní Weinzierl - KNX IP BAOS 774 \cite{Weinzier ob}}
  \label{fig:Komunikační rozhraní Weinzierl - KNX IP BAOS 774}
\end{figure}

\section{Vytvoření skupinových adres projektu}
V závislosti na informacích obsažených v podkapitole \ref{Adresování} se tato podkapitola zaměří pouze na tvorbu skupinových adres. První část této podkapitoly bude věnována vytvoření a popsání tabulek jednotlivých místností. Tyto tabulky bude použity pro popis funkce jednotlivých tlačítek a následně pro tvorbu skupinových adres. Dále tyto tabulky nebudou obsahovat tlačítko společnosti Berker, které nelze parametrizovat.

První místnost, které budou nastaveny jsou kuchyně a koupelna. Do těchto prostor byly pomyslně nainstalovány tlačíka společnosti Ekinex a Siemens. Aby se využilo maximálně těchto tlačítek, budou využita obě tlačítka i v jiných místnostech. Zejména se jedná o tlačítko Ekinex, které má na pravé klapce žaluzie. V případě tlačítka Siemens se jedná pouze o využití velkého množství tlačítek, které budou použity při dlouhém stisku na ovládání celé budovy. 

\begin{table}[h]
 \caption[Funkce kuchyňských tlačítek pro krátké stisknutí]{Funkce kuchyňských tlačítek pro krátké stisknutí}
   \small
    \centering
	  \begin{tabular}{|c|c|c|}
	    \hline
	    Tlačítko & Ekinex & Siemens  \\
	    \hline\hline
	    1. & S1 Zapnuto/Vypnuto & Ch3 Vypnuto \\
	    \hline
        2. & S2 Zapnuto/Vypnuto & Ch3 Zapnuto \\
	    \hline
        3. & Ž1, Ž2 krok nahorů & T3 Vypnuto \\
	    \hline
        4. & Ž1, Ž2 krok dolů & T3 Zapnuto \\
	    \hline
        5. & - & T3/1 Vypnuto \\
	    \hline 
        6. & - & T3/1 Zapnuto \\
	    \hline 
	    7. & - & T3/2 Vypnuto \\
	    \hline
	    8. & - & T3/2 Zapnuto \\
	    \hline
	  \end{tabular}
\end{table}

\begin{table}[h]
 \caption[Funkce kuchyňských tlačítek při dlouhém stisknutí]{Funkce kuchyňských tlačítek při dlouhém stisknutí}
   \small
    \centering
	  \begin{tabular}{|c|c|c|}
	    \hline
	    Tlačítko & Ekinex & Siemens  \\
	    \hline\hline
	    1. & S1, S2, S6 Zapnuto/Vypnuto &  Ch1 Zapnuto/Vypnuto \\
	    \hline
        2. & S6 Zapnuto/Vypnuto &  Ch2 Zapnuto/Vypnuto \\
	    \hline
        3. &  Ž1, Ž2 nahorů & T1, T2, T2 Vypnuto \\
	    \hline
        4. & Ž1, Ž2 dolů & T1, T2, T2 Zapnuto \\
	    \hline
        5. & - &  Ch1, Ch2, Ch3 Vypnuto\\
	    \hline 
        6. & - & Ch1, Ch2, Ch3 Zapnuto \\
	    \hline 
	    7. & - & S1,S2,S6 Vypnuto \\
	    \hline
	    8. & - & S3,S4,S5 Zapnuto \\
	    \hline
	  \end{tabular}
\end{table}

\newpage Dalším místností se dvěma pomyslně nainstalovanými tlačítky je obyvací pokoj. Jedná se o tlačítka společnosti ABB a Simon. Tlačítko Simon bude použito na ovládání žaluzií a tlačítko ABB na ovládání topení, chlazení a světel v místnosti.

\begin{table}[h]
 \caption[Funkce tlačítek obývacího pokoje při krátké stisknutí]{Funkce tlačítek obývacího pokoje při krátké stisknutí}
   \small
    \centering
	  \begin{tabular}{|c|c|c|}
	    \hline
	    Tlačítko & ABB & Simon  \\
	    \hline\hline
	    1. & T1 Vypnuto & Ž1 krok nahorů  \\
	    \hline
        2. & T2 Vypnuto & Ž1 krok dolů  \\
	    \hline
        3. & Ch1,Ch2 Vypnuto & Ž2 krok nahorů  \\
	    \hline
        4. & S3 Vypnuto & Ž2 krok dolů  \\
	    \hline
        5. & S4 Vypnuto & - \\
	    \hline 
        6. & S5 Vypnuto & - \\
	    \hline 
	  \end{tabular}
\end{table}

\begin{table}[h]
 \caption[Funkce tlačítek obývacího pokoje při dlouhém stisknutí]{Funkce tlačítek obývacího pokoje při dlouhém stisknutí}
   \small
    \centering
	  \begin{tabular}{|c|c|c|}
	    \hline
	    Tlačítko & ABB & Simon  \\
	    \hline\hline
	    1. & T1 Zapnuto & Ž1 nahorů  \\
	    \hline
        2. & T2 Zapnuto & Ž1 dolů  \\
	    \hline
        3. & Ch1,Ch2 Zapnuto & Ž2 nahorů  \\
	    \hline
        4. & S3 Zapnuto & Ž2 dolů  \\
	    \hline
        5. & S4 Zapnuto & - \\
	    \hline 
        6. & S5 Zapnuto & - \\
	    \hline 
	  \end{tabular}
\end{table}

Dotyková tlačítka společností Basalte a HDL byla určena na ovládání scén a barvy pozadí objektu. Tlačítko společnosti basalte v této práci reaguje pouze na  krátký dotek jednotlivých tlačítek. Tahle skutečnost je způsobena použitím scén. Při použití funkce volání scény nelze využít dlouhého dotek. Další z vlastností tlačítka je již zmiňovaný multitouch, který funguje na bázi doteku dvou a více ploch najednou. V této práci je použit krátký dotek na zavolání scény odchod a dlouhý dotek na volání scény příchod.

\begin{table}[h]
 \caption[Funkce dotykových tlačítek při krátké stisknutí]{Funkce dotykových tlačítek při krátké stisknutí}
   \small
    \centering
	  \begin{tabular}{|c|c|c|}
	    \hline
	    Tlačítko & Basalte & HDL \\
	    \hline\hline
	    1. & Scéna dovolená  & Červené podsvícení \\
	    \hline
        2. & Scéna léto  &  Zelené podsvícení \\
	    \hline
        3. & Scéna zima  & Modré podsvícení   \\
	    \hline
        4. & RGB Kontroler & Bílé podsvícení \\
	    \hline
        5. & - & Nastavená hodnota RGB 1 \\
	    \hline 
        6. & - &  Nastavená hodnota RGB 2 \\
	    \hline 
	  \end{tabular}
\end{table}

\begin{table}[h]
 \caption[Funkce dotykových tlačítek při dlouhém stisknutí]{Funkce dotykových tlačítek při dlouhém stisknutí}
   \small
    \centering
	  \begin{tabular}{|c|c|c|}
	    \hline
	    Tlačítko & Basalte & HDL  \\
	    \hline\hline
	    1. & - & Červené podsvícení stmívání  \\
	    \hline
        2. & - & Zelené podsvícení stmívání \\
	    \hline
        3. & - & Modré podsvícení stmívání  \\
	    \hline
        4. & - & Bílé podsvícení  stmívání \\
	    \hline
        5. & - & Nastavená hodnota RGB 3 \\
	    \hline 
        6. & - & Nastavená hodnota RGB 4 \\
	    \hline 
	  \end{tabular}
\end{table}

Poslední z použitých spínačů je detektor pohybu, kterému bylo logicky přiřazeno přední světlo domu.\\ \\Ze vzniklých tabulek byly vytvořeny skupinové adresy, které byly rozděleny do skupin dle přístroje (Světla, Žaluzie, Topení, Klimatizace, LED, Scény a Měření). Tyto skupiny se dále dělí na podle funkcionality a množství. Poslední vrstva již představuje jednotlivé objekty, nebo scény. Výpis skupinových adres je součástí příloh.
\chapter{Foxtrot}
\section{Komunikace KNX/IP}
\section{Komunikace MQTT}
\section{Web Server}
\chapter{Raspberry Pi 5}
Jedná se o jednodeskový počítač, který je určen pro široké spektrum aplikací od IoT, rozpoznáváni obrazu, strojového učení, robotiku až po multimediální aplikace. Mezi hlavní výhody patří nízká cena, malé provozní náklady, možná modularita přes GPIO piny, PCI expres a USB porty a velkým množstvím tzv. HAT modulů, které jsou určeny pro rozšíření funkcí. \cite{Raspberry Pi 5}.

Pro tuto aplikaci bylo zvoleno Raspberry Pi 5 s 16 GB RAM, pamětí 256 GB a operačním systémem Raspberry Pi OS, který je stavěn na Linuxové distribuci Debianu. Hlavním důvodem výběru byla nízká cena, ale i jednoduchá dostupnost pro případného zájemce o domácí automatizaci, který by si chtěl systém ovládání a vizualizace vytvořit sám.

Jediným závažnějším problémem totoho zařízení je "obrušování" paměťové karty na které je nahrán operační systém. Tento problém je způsoben častým zápisem, který postupně snižuje životnost, rychlost a velikost paměti. V krajních případech i může dojít k úplnému zničení a ztrátě dat. K zamezení tohoto problému se historicky používala technika \textit{wear leveling} - rozložení zápisu na celou paměťovou kartu, aby se snížil počet zápisů na jednotlivé buňky \cite{wear leveling}. Teď už je možné použít M.2 NVMe SSD disk, který je připojen přes PCI expres a je mnohem rychlejší než paměťová karta. Tento disk je možné použít i jako bootovací disk, což zamezí problémům s bootováním a ztrátou dat. Další možností jak zpomalit toto obrušování je připojení USB flash disku, který za obětování rychlosti dokáže zamezit opotřebování paměti díky přenesení souborů s vyšší frekvencí zápisu - v tomto případě databáze.
\begin{figure}[!ht]
    \begin{center}
        \includegraphics[scale=0.30]{obrazky/RaspberryPi5.jpg}
    \end{center}
    \caption[Raspberry Pi5 \cite{Raspberry Pi 5}]{Raspberry Pi5 \cite{Raspberry Pi 5}}
    \label{fig:RaspberryPi}
\end{figure}
\newpage
Alternativou k Raspberry Pi 5 můžou být LattePanda, ASUS NUC a podobné mikropočítače, které mají větší výkon, větší možnost výběru operačního systému (dostatečný výkon na Windows) a robustnost. Tyto výhody jsou kompenzovány vyšší pořizovací cenou, spotřebou a velikostí.
\section{Docker Compose}
Docker Compose je nástroj pro definici a správu více kontejnerových aplikací. Umožňuje uživatelům definovat aplikaci pomocí \textit{YAML} souboru, které obsahují informace o kontejnerových službách, sítích a uložištích. \cite{Docker Compose}

Instalace Docker Compose je jednoduchá a rozepsaná na oficiálních stránkách Dockeru. Obsahuje pouze tři kroky, které jsou již připravené pro kopírování do terminálu. \cite{DockerInstallationForDebian}

\subsection{Kontejnerizace}
Kontejnerizace je způsob virzualizace aplikací, které umožňuje spouštět v izolovaném prostředí s garancí funkčnosti v jakémkoliv prostředí, založený na linuxové technologii LXC (Linux Containers). Tento způsob je výhodný pro nasazení aplikací, které mají různé závislosti a konfigurace. Umožňuje také snadné nasazení a škálování aplikací. Dále tento způsob virtualizace dodávat kompletní prostředí, jako jiné virtuální stroje, ale s menšími režijními nároky na výkon a paměť, které přichází s fungováním separátního kernelu a simulací veškerého hardware. \cite{ContainerAndVirtualization}
V případě této práce byl navrhnut Docker Stack, který je složený z několika kontejnerů, které spolu komunikují a vytváří tak komplexní systém na ovládání a vizualizaci domácí automatizace. Vzhled tohoto stacku i s komunikacemi je zobrazen na Obr. \ref{fig:DockerStack}.
\begin{figure}[!ht]
    \begin{center}
        \includegraphics[scale=0.30]{obrazky/stack.png}
    \end{center}
    \caption[Docker Stack]{Docker Stack }
    \label{fig:DockerStack}
\end{figure}
\subsection{Tvorba YAML souboru}
YAML soubor je textový soubor, který obsahuje definici kontejnerů a jejich konfigurací. Je napsán v jazyce YAML (YAML Ain't Markup Language), což je formát pro serializaci dat \cite{YAML}. Níže je příklad vzhledu YAML souboru (Výpis \ref{lst:yaml}), který by měl pomoci ke snadnému pochopení struktury a syntaxe \cite{ComposeYamlExample}.

Dále pak může YAML soubor obsahovat i enviromentální proměnné, které se používají k nastavení kontejneru. Tyto proměnné mouhou být obsaženy v souboru .env, který se používá k uchování citlivých informací, jako jsou hesla a API klíče. Tento soubor je načítán při spuštění kontejneru. \cite{ENV}
\begin{lstlisting}[language=yaml, caption=Ukazka YAML souboru, label=lst:yaml]
version: "3" # Verze Docker Compose
services: # Sluzby
    name_of_service: # Jmeno sluzby
        image: name_of_image:latest # Jmeno image
        container_name: name_of_container # Jmeno kontejneru
        networks: # Site
            - name_of_network # Jmeno site na ktere pobezi kontejner
        depends_on: # Zavislosti
            - name_of_service # Jmeno kontejneru na kterem zavisi
        environment: # Promenne prostredi
        - PUID=1000 # ID uzivatele, ktery bude mit pristup k souborum
        - PGID=1000 # ID skupiny, ktera bude mit pristup k souborum
        - TZ=Europe/Prague # Casove pasmo
        - WEBUI_PORT=1234 # Port na kterem pobezi webove rozhrani 
        - DOCKER_MODS="linuxserver/mods:universal" # Docker modifikace
        volumes: # Slozky, ktere budou pripojeny do kontejneru
            - "/path/on/host:/path/in/container" # Cesta k souborum
        ports: # Porty pro pripojeni k hostitelske site
            - "host_port:container_port"
        deploy: # Nasazeni kontejneru
            resources:
                limits:
                    memory: 512m # Maximalni pamet
                    cpus: "1" # Maximalni CPU
                reservations:
                    memory: 256m # Minimalni pamet
                    cpus: "0.5" # Minimalni CPU
        restart: always # Restart kontejneru pri padu
        labels: # Metada
            - "com.docker.compose.project=project_name" # Nazev projektu
            - "com.docker.compose.service=service_name" # Nazev sluzby
            - "com.docker.compose.version=1.0" # Verze sluzby
\end{lstlisting}
\noindent Celková implementace je v přílohách \ref{apend:portainer} a \ref{apend:stackyaml}.
\subsection{Portainer}
Jedná se o open-source kontejner, který slouží jako webové rozhraní pro správu Dockeru. Umožňuje uživatelům spravovat kontejnery, Image, Stacky, Sítě, Uložiště, čtení logů, sledování výkonu, správa portů a další funkce. Jednou z předních výhod je jednoduchost použití kromě rozhraní je i propojení s Docker Hubem, což je veřejná knihovna kontejnerů. \cite{Portainer} 
\section{Mosquitto}
\section{Home Assistant}
\section{Influxdb}
\section{Grafana}

%%% Vložení souboru 'text/zaver' se závěrem
\chapter*{Závěr}
\phantomsection
\addcontentsline{toc}{chapter}{Závěr}

V rámci této bakalářské práce byly splněny všechny stanovené cíle týkající se návrhu a realizace vzdáleného řízení a vizualizace demonstračního panelu KNX pro ovládání funkcí osvětlení, žaluzií, topení a klimatizace. Práce poskytuje ucelený pohled na problematiku sběrnicového systému KNX, jeho možnosti a praktické využití v oblasti domácí automatizace. 

V první části práce byla provedena důkladná analýza technologie KNX, včetně její historie, základních principů fungování, struktury komunikace, zabezpečení a topologie. Tato část poskytla nezbytný teoretický základ pro následnou praktickou realizaci. 

Druhá část práce se zaměřila na parametrizaci konkrétních přístrojů a tvorbu dynamických světelných scén. Byly popsány jednotlivé kroky od návrhu projektu až po jeho implementaci, včetně identifikace problémů. 

Třetí část práce se věnovala samotnému řízení a vizualizaci prostřednictvím PLC společnosti TECO. Byly popsány nejen technické aspekty programování a komunikace mezi PLC a KNX, ale také implementace MQTT protokolu a tvorba webové vizualizace. V této části práce vznikl problém během odevzdání, kdy došlo k poškození komunikační brány, což mělo za následek následnou nefunkčnost komunikace mezi PLC a KNX instalací.

V poslední části byly zhodnoceny možnosti využití open-source platforem pro vizualizaci a správu domácí automatizace. Byla provedena implementace na jednodeskovém počítači Raspberry Pi s využitím kontejnerizace Docker a nasazením několika open-source nástrojů, jako jsou Home Assistant, InfluxDB či Grafana. Tyto nástroje byly vybrány a nasazeny s ohledem na jejich aktuální popularitu, dostupnost a možnosti rozšíření.

Celkově lze konstatovat, že práce splnila vytyčené cíle a přinesla komplexní řešení vzdáleného řízení a vizualizace KNX panelu. Byly ověřeny možnosti integrace různých technologií a platforem, přičemž důraz byl kladen na praktičnost, flexibilitu a budoucí rozšiřitelnost řešení. Výsledky práce mohou sloužit jako inspirace či základ pro další rozvoj v oblasti chytré domácnosti a automatizace budov.


%%% Vložení souboru 'text/literatura' se seznamem zdrojů
% Pro sazbu seznamu literatury použijte jednu z následujících možností

%%%%%%%%%%%%%%%%%%%%%%%%%%%%%%%%%%%%%%%%%%%%%%%%%%%%%%%%%%%%%%%%%%%%%%%%%
%1) Seznam citací definovaný přímo pomocí prostředí literatura / thebibliography
\begin{thebibliography}{99}

\bibitem{Datapoint}
    	Asociace KNX \emph{Datapoint Type}\/ Online. 
    	Dostupné z:
    \url{https://support.knx.org/hc/en-us/articles/115001133744-Datapoint-Type}
    [cit.\,23.\,12.\,2021]. 
    
\bibitem{KNX history}
		Asociace KNX 
		\emph{A History of KNX}\/ Online.
		Dostupné z:
    \url{https://crelectrics.com.au/wp-content/uploads/2015/05/a_history_of_KNX.pdf}  
    [cit.\,1.\,10.\,2021]. 
    
\bibitem{KNX basics}
		Asociace KNX
		\emph{KNX Basics}\/ Online.
		Dostupné z:
    \url{https://www.knx.org/wAssets/docs/downloads/Marketing/Flyers/KNX-Basics/KNX-Basics_cz.pdf} 
		[cit.\,1.\,10.\,2021].
    
\bibitem{KNX principles}
    	Asociace KNX \emph{Principy systému KNX}\/ Online. 
    	Dostupné z:
    \url{https://knxcz.cz/images/clanky/KNX-System-Principles_cz.pdf} 
    	[cit.\,1.\,10.\,2021].
    
\bibitem{KNX Secure}
    	Asociace KNX \emph{KNX Secure Devices}\/ Online. 
    	Dostupné z:
    \url{https://support.knx.org/hc/en-us/articles/360000216419-KNX-Secure-Devices} 
    	[cit.\,23.\,12.\,2021].
    
\bibitem{Celkovy prehled}
    Asociace KNX \emph{ISO/IEC 14543-3. KNX Celkový přehled.}
    
\bibitem{Systemove Argumenty}
    Asociace KNX \emph{ISO/IEC 14543-3. KNX Systémové argumenty.}
    
\bibitem{Topologie}
    Asociace KNX \emph{ISO/IEC 14543-3. KNX TP Topologie.}
    
\bibitem{Mitrenga}
    MITRENGA, Michal.:
    \emph{Realizace demonstrativního panelu inteligentní elektroinstalace KNX. Brno, 2021.}\/ Online. 
    [cit. 26.\,12.\,2021].
    Dostupné z:
    \url{https://www.vutbr.cz/studenti/zav-prace/detail/134788}
    \emph{Diplomová práce. Vysoké učení technické v Brně, Fakulta elektrotechniky a komunikačních technologií, Ústav automatizace a měřicí techniky. Vedoucí práce Petr Fiedler.}
 
% \bibitem{Kalfus}
%    KALFUS, Petr..:
%    \emph{Návrh demonstračního panelu KNX. Brno, 2020.}\/ Online. 
%  [cit. 29.\,12.\,2021].
%   Dostupné z:
%    \url{https://www.vutbr.cz/studenti/zav-prace/detail/127255}   
%    \emph{Bakalářská práce. Vysoké učení technické v Brně, Fakulta elektrotechniky a komunikačních technologií, Ústav elektroenergetiky. Vedoucí práce Branislav Bátora.}

\bibitem{Asociace KNX}
		knx.org\/ Online.
		Dostupné z:
    \url{https://www.knx.org}
		[cit.\,1.\,10.\,2021]. 
    
\bibitem{ETS Kecy}
		knx.org \emph{ETS Professional}\/ Online.
		Dostupné z:
    \url{https://www.knx.org/knx-en/for-professionals/software/ets-professional/}
		[cit.\,2.\,1.\,2022]. 

\bibitem{ABB}
		ABB - SBR/U6.0.1-84\/ Online. 
		Dostupné z:
    \url{https://new.abb.com/products/2CKA006330A0004/sbr-u6-0-1-84}
		[cit.\,28.\,12.\,2021].
    
\bibitem{ABB aktor1}
		ABB - SA/S8.10.2.1\/ Online.
		Dostupné z:
    \url{https://new.abb.com/products/2CDG110157R0011/sa-s8-10-2-1}
		[cit.\,28.\,12.\,2021]. 
    
\bibitem{ABB aktor2}
		ABB - JRA/S4.230.2.1\/ Online. 
		Dostupné z:
    \url{https://new.abb.com/products/2CDG110121R0011/jra-s4-230-2-1}
		[cit.\,28.\,12.\,2021].
    
\bibitem{Basalte}
		Basalte - Sentido aluminium - quad - Brushed black\/ Online.
		Dostupné z:
    \url{https://www.knxstore.cz/domu/1000403-basalte-sentido-aluminium-quad-brushed-black-5425025030224.html}
		[cit.\,28.\,12.\,2021]. 
    
 \bibitem{BEG}
		B.E.G - Indoor 140-L-KNX-DX\/ Online. 
		Dostupné z:
    \url{https://www.beg-luxomat.com/cz/produkty/luxomatnet/knx/knx-gen6-deluxe-pritomnostni-detektor/indoor-140-l-knx-dx/}
		[cit.\,28.\,12.\,2021].  
    
\bibitem{Berker}
		Berker - B.IQ push-button 3gang with thermostat Display, KNX\/ Online. 
		Dostupné z:
    \url{https://www.berker.com/en/e-catalogue/building-management-systems/knx-systems/berker-knx-system/b.iq-push-buttons-with-thermostat/75663593/355802.htm?lang=en}
		[cit.\,28.\,12.\,2021].
    
\bibitem{Ekinex}
		EKINEX - Pushbutton with thermostat\/ Online. 
		Dostupné z:
    \url{https://www.ekinex.com/en/15/pushbutton-with-thermostat.html}
		[cit.\,28.\,12.\,2021].
    
\bibitem{HDL}
        HDL - M/TBP6.1-A2-46 \textit{Ovládací prvek 6násobný iTouch, bílá}\/ Online. 
		Dostupné z:
    \url{https://b2b.hdl-automation.cz/cz/produkty/knx/ovladaci-prvky-hdl/ovladaci-prvky-itouch/hdl-m-tbp6-1-a2-46}
		[cit.\,28.\,12.\,2021].
    
\bibitem{HDL aktor1}
        HDL - M/R8.10.1 \textit{8CH 10A High Power Switch Actuator}\/ Online. 
		Dostupné z:
    \url{https://b2b.hdl-automation.cz/en/products/knx/switching-actuators/hdl-m-r8-10-1}
		[cit.\,28.\,12.\,2021].
    
\bibitem{HDL aktor2}
        HDL - M/DRGW4.1 \textit{Akční člen stmívací LED 4násobný, 7 A}\/ Online. 
		Dostupné z:
    \url{https://b2b.hdl-automation.cz/cz/produkty/knx/akcni-cleny-stmivaci/hdl-m-drgbw4-1}
		[cit.\,28.\,12.\,2021].
    
\bibitem{Siemens}
		SIEMENS - QMX3.P37 \textit{Prostorový KNX přístroj, displej pro regulaci HVAC, čidlo teploty, konfigurovatelná tlačítka pro osvětlení/žaluzie/scény}\/ Online. 
		Dostupné z:
    \url{https://hit.sbt.siemens.com/RWD/app.aspx?RC=CZ&lang=cs&MODULE=Catalog&ACTION=ShowProduct&KEY=S55624-H108}
		[cit.\,28.\,12.\,2021].
    
\bibitem{Siemens IP}
		SIEMENS - 5WG1 146-1AB03 \/ Online. 
		Dostupné z:
    \url{https://www.hqs.sbt.siemens.com/cps_product_data/data/search_find_en.htm?ssn=5WG11461AB03}
		[cit.\,28.\,12.\,2021].  
    
\bibitem{Simon}
		Simon - Standard button box 4 functions white Simon 82 Sense\/ Online. 
		Dostupné z:
    \url{https://www.simonelectric.com/intl/8000641-030-standard-button-box-4-functions-white-simon-82-sense.html}
		[cit.\,28.\,12.\,2021].

\bibitem{Weinzier}
		Weinzier - KNX IP BAOS 774\/ Online. 
		Dostupné z:
    \url{https://www.weinzierl.de/index.php/en/all-knx/knx-devices-en/knx-ip-baos-774-en}
		[cit.\,28.\,12.\,2021].
    
\bibitem{Weinzier ob}
		Weinzier - KNX IP BAOS 774 / Rozhraní BAOS do 1000 bodů\/ Online. 
		Dostupné z:
    \url{https://knx-trade.ru/weinzierl/597-5263.html}
		[cit.\,28.\,12.\,2021].

\bibitem{TECO}
		TECO - CP-2007\/ Online. 
		Dostupné z:
	\url{hhttps://wiki.tecomat.cz/clanek/cp-2007}
		[cit.\,23.\,4.\,2025].

\bibitem{KNXlib}
		TECO - Knihovna KnxLib\/ Online. 
		Dostupné z:
	\url{https://www.tecomat.cz/modules/DownloadManager/download.php?alias=txv00380_01}
		[cit.\,23.\,4.\,2025].

\bibitem{MQTTlib}
		TECO - Knihovna MQTTLib\/ Online. 
		Dostupné z:
	\url{https://support.tecomat.cz/storage/app/uploads/public/633/acd/862/633acd8625f3d859405244.pdf}
		[cit.\,23.\,4.\,2025].

\bibitem{WebMaker}
		TECO - WebMaker\/ Online. 
		Dostupné z:
	\url{https://www.tecomat.cz/modules/DownloadManager/download.php?alias=txv00328_01_mosaic_webmaker_cz}
		[cit.\,23.\,4.\,2025].
\end{thebibliography} 
%%%%%%%%%%%%%%%%%%%%%%%%%%%%%%%%%%%%%%%%%%%%%%%%%%%%%%%%%%%%%%%%%%%%%%%%%
%%2) Seznam citací pomocí BibTeXu
%% Při použití je nutné v TeXnicCenter ve výstupním profilu aktivovat spouštění BibTeXu po překladu.
%% Definice stylu seznamu
%\bibliographystyle{unsrturl}
%% Pro českou sazbu lze použít styl czechiso.bst ze stránek
%% http://www.fit.vutbr.cz/~martinek/latex/czechiso.tar.gz
%%\bibliographystyle{czechiso}
%% Vložení souboru se seznamem citací
%\bibliography{text/literatura}
%
%% Následující příkaz je pouze pro ukázku sazby literatury při použití BibTeXu.
%% Způsobí citaci všech zdrojů v souboru literatura.bib, i když nejsou citovány v textu.
%\nocite{*}

%%% Vložení souboru 'text/zkratky' se seznam použitých symbolů, veličin a zkratek
\cleardoublepage
\chapter*{\listofabbrevname}
\phantomsection
\addcontentsline{toc}{chapter}{\listofabbrevname}

\begin{acronym}[KolikMista]

	\acro{ETS}
		{Engineering Tool Software}

	\acro{TP}		% název/zkratka
		{Kroucená dvojlinka (\textit{Twisted pair})}
	
	\acro{PL}
		{Powerline}

	\acro{RF}
		{Radiofrekvenční komunikační médium}
		
    \acro{IP}
		{Internetový protokol}
		
	\acro{WLAN}
		{Bezdrátová lokální síť (\textit{Wireless Local Area Network})}
		
	\acro{EIBA}
		{European Installation Bus Association}
	
	\acro{BCI}
		{Batibus Club International}
	
	\acro{EHSA}
		{European Home Systems Association}
	
	\acro{ICT}
		{Information and Communication Technology}
	
	\acro{CSMA/CA}
		{Carrier Sense Multiple Access with Collision Avoidance}
	
	\acro{DPT}
		{Datový bod}
	
	\acro{MAC}
		{Message Authentication Code}
	
	\acro{FDSK}
		{Factory Default Setup Key}
	
	\acro{US}
		{Účastník sběrnice}
	
	\acro{LS}
		{Liniová spojka}
	
	\acro{OS}
		{Oblastní spojka}
	
	\acro{NZ/TI}
		{Zdroj s tlumivkou}
	
	\acro{PLC}
		{Programovatelný automat}
	
	\acro{ISDN}
		{Digitální síť integrovaných služeb}
	
	\acro{RGB}
		{Red-Green-Blue - aditivní způsob míchání barev}
	
	\acro{LED}
		{Elektroluminiscenční dioda}
	
	\acro{NO}
		{Normally Opened}
	
		
\end{acronym}


%%% Začátek příloh
\appendix

%%% Vysázení seznamu příloh
% (vynechejte, pokud máte dvě nebo méně příloh)
\listofappendices

%%% Vložení souboru 'text/prilohy' s přílohami
% Obvykle je přítomen alespoň popis co najdeme na přiloženém médiu
\chapter{Skupinové adresy}
\label{apend:skupinove_adresy}
\includepdf[pages = 1]{pdf/GroupAddressesReport}
\includepdf[pages = 2]{pdf/GroupAddressesReport}
\includepdf[pages = 3]{pdf/GroupAddressesReport}
\includepdf[pages = 4]{pdf/GroupAddressesReport}
\chapter{Globální proměnné a struktury PLC}
\label{apend:globalni_promenne}
\begin{lstlisting}[language=ST, breaklines=true, numbers=left, numberstyle=\small, numbersep=10pt, frame=single, basicstyle=\ttfamily\small, caption={Globální proměnné a struktury PLC}, label={lst:globalni_promenne}]
TYPE
  DT_CMD_BOOL : STRUCT
              CMD_VAL : BOOL;
              CMD : BOOL;
  END_STRUCT;
END_TYPE
VAR_GLOBAL
  SV1_FB          : BOOL := FALSE; //KNX COMM SV1 FB
  SV2_FB          : BOOL := FALSE; //KNX COMM SV2 FB
  SV3_FB          : BOOL := FALSE; //KNX COMM SV3 FB
  SV4_FB          : BOOL := FALSE; //KNX COMM SV4 FB
  SV5_FB          : BOOL := FALSE; //KNX COMM SV5 FB
  SV6_FB          : BOOL := FALSE; //KNX COMM SV6 FB
  SV7_FB          : BOOL := FALSE; //KNX COMM SV6 FB
  SV1_CMD         : DT_CMD_BOOL; //KNX CMD SV1
  SV2_CMD         : DT_CMD_BOOL; //KNX CMD SV2
  SV3_CMD         : DT_CMD_BOOL; //KNX CMD SV3
  SV4_CMD         : DT_CMD_BOOL; //KNX CMD SV4
  SV5_CMD         : DT_CMD_BOOL; //KNX CMD SV5
  SV6_CMD         : DT_CMD_BOOL; //KNX CMD SV6
  SV7_CMD         : DT_CMD_BOOL; //KNX CMD SV6
  HEAT1_FB        : BOOL := FALSE; //KNX COMM HEAT1 FB
  HEAT2_FB        : BOOL := FALSE; //KNX COMM HEAT2 FB
  HEAT3_FB        : BOOL := FALSE; //KNX COMM HEAT3 FB
  HEAT1_CMD       : DT_CMD_BOOL; //KNX CMD HEAT1
  HEAT2_CMD       : DT_CMD_BOOL; //KNX CMD HEAT2
  HEAT3_CMD       : DT_CMD_BOOL; //KNX CMD HEAT3
  COLD1_FB        : BOOL := FALSE; //KNX COMM COLD1 FB
  COLD2_FB        : BOOL := FALSE; //KNX COMM COLD2 FB
  COLD3_FB        : BOOL := FALSE; //KNX COMM COLD3 FB
  COLD1_CMD       : DT_CMD_BOOL; //KNX CMD COLD1
  COLD2_CMD       : DT_CMD_BOOL; //KNX CMD COLD2
  COLD3_CMD       : DT_CMD_BOOL; //KNX CMD COLD3
  SHUT1_FB_UP     : BOOL := FALSE; //KNX COMM SHUT1 FB
  SHUT1_FB_DOWN   : BOOL := FALSE; //KNX COMM SHUT1 FB
  SHUT1_CMD       : DT_CMD_BOOL; //KNX CMD SHUT1
  SHUT1_STEP_CMD  : DT_CMD_BOOL; //KNX CMD SHUT1 STEP
\end{lstlisting}
\begin{lstlisting}[language=ST, breaklines=true, numbers=left, firstnumber=38, numberstyle=\small, numbersep=10pt, frame=single, basicstyle=\ttfamily\small]
  SHUT2_FB_UP     : BOOL := FALSE; //KNX COMM SHUT2 FB
  SHUT2_FB_DOWN   : BOOL := FALSE; //KNX COMM SHUT2 FB
  SHUT2_CMD       : DT_CMD_BOOL; //KNX CMD SHUT1
  SHUT2_STEP_CMD  : DT_CMD_BOOL; //KNX CMD SHUT1 STEP
  KNX_TEMPER      : REAL := 0.0; //KNX TEMPERATURE OUT
  OUTSIDE_TEMPER  : REAL := 0.0; //OUTSIDE TEMPER FOR MQTT
  KITCHEN_TEMPER  : REAL := 0.0; //KITCHEN TEMPER FOR MQTT
  LIVINGR_TEMPER  : REAL := 0.0; //LIVINGROOM TEMPER FOR MQTT
  BATHROOM_TEMPER : REAL := 0.0; //BATHROOM TEMPER FOR MQTT
  SHUTDOWN_MQTT   : BOOL := FALSE; //SHUTDOWN FROM MQTT
END_VAR
\end{lstlisting}
\chapter{Definice funkčních složitějších funkčních bloků}
\label{apend:fb}
\subsection{fbRoomTempMod}
\label{apend:fbRoomTempMod}
\begin{lstlisting}[language=ST, breaklines=true, numbers=left, numberstyle=\small, numbersep=10pt, frame=single, basicstyle=\ttfamily\small, caption={fbRoomTempMod}, label={lst:fbRoomTempMod}]
FUNCTION_BLOCK fbRoomTempMod
  VAR_INPUT
    Heat_1       : BOOl; // Topení vstup 1[-]
    Heat_1_WATTS : REAL; // Topení výkon 1[W] =>[J/s]
    Heat_2       : BOOL; // Topení vstup 2[-]
    Heat_2_WATTS : REAL; // Topení výkon 2[W] =>[J/s]
    Cold_1       : BOOl; // Klimatizace vstup 1[-]
    Cold_1_WATTS : REAL; // Klimatizace výkon 1[W] =>[J/s]
    Cold_2       : BOOL; // Klimatizace vstup 2[-]
    Cold_2_WATTS : REAL; // Klimatizace výkon 2[W] =>[J/s]
    lenght       : REAL; // délka[m]
    width        : REAL; // šířka[m]
    height       : REAL; // výška[m]
    wall_temp1   : REAL; // Teplota za sousední zdí[degC]
    wall_temp2   : REAL; // Teplota za sousední zdí[degC]
    wall_temp3   : REAL; // Teplota za sousední zdí[degC]
    wall_temp4   : REAL; // Teplota za sousední zdí[degC]
    floor_temp   : REAL; // Teplota v místnosti pod[degC]
    ceiling_temp : REAL; // Teplota v místnosti nad[degC]
    wall_thic1   : REAL; // Šířka zdi1[m]
    wall_thic2   : REAL; // Šířka zdi2[m]
    wall_thic3   : REAL; // Šířka zdi3[m]
    wall_thic4   : REAL; // Šířka zdi4[m]
    floor_thic   : REAL; // Šířka podlahy[m]
    ceiling_thic : REAL; // Šířka stropu[m]
    TaskTime     : REAL; // Rychlost tasku[ms]
  END_VAR
  VAR_OUTPUT
    Temperature  : REAL := 20.0; // Teplota na výstupu[degC]
  END_VAR
  VAR
    INIT         : BOOL := FALSE; //INIT bloku
    TimeStep     : REAL := 0.0; // Hodnota kroku v ms
\end{lstlisting}
\pagebreak
\begin{lstlisting}[language=ST, breaklines=true, numbers=left, firstnumber=34, numberstyle=\small, numbersep=10pt, frame=single, basicstyle=\ttfamily\small]
    VAir         : REAL := 0.0; // Obsah vzduchu[m^3]
    MAir         : REAL := 0.0; // Váha vzduchu[Kg]
    QAir         : REAL := 0.0; // Energie změna o 1degC[J]
    RoomTemp     : REAL := 20.0;// Pokojová teplota[degC]
    DeltaTemp    : REAL := 0.0; // Teplota za cyklus[degC]
    KHeatRise    : REAL := 2.4; // Korekční člen tr[-]
    KColdRise    : REAL := 45.0;// Korekční člen kr[-]
    KHeatFall1   : REAL := 0.0; // Korekční člen tf 1[-]
    KColdFall1   : REAL := 0.0; // Korekční člen kf 1[-]
    KHeatFall2   : REAL := 0.0; // Korekční člen tf 2[-]
    KColdFall2   : REAL := 0.0; // Korekční člen kf 2[-]
    Epsilon      : REAL := 1.0; // Sp výkon za čas[w]
    AlphaHeat    : REAL := 0.0; // Logaritmus topení[-]
    AlphaCold    : REAL := 0.0; // Logaritmus klimatizace[-]
    FiTotal      : REAL := 0.0; // Celkový tepelný tok[J/s]
    FiHeat       : REAL := 0.0; // Celkový tepelný tok t[J/s]
    FiHeatTmp1   : REAL := 0.0; // Tepelný tok topení 1[J/s]
    FiHeatTmp2   : REAL := 0.0; // Tepelný tok topení 2[J/s]
    FiCold       : REAL := 0.0; // Celkový tepelný tok k[J/s]
    FiColdTmp1   : REAL := 0.0; // Tepelný tok klim 1[J/s]
    FiColdTmp2   : REAL := 0.0; // Tepelný tok klim 2[J/s]
    AreaWall1    : REAL := 0.0; // Plocha zdi 1[m^2]
    AreaWall2    : REAL := 0.0; // Plocha zdi 2[m^2]
    AreaWall3    : REAL := 0.0; // Plocha zdi 3[m^2]
    AreaWall4    : REAL := 0.0; // Plocha zdi 4[m^2]
    AreaFloor    : REAL := 0.0; // Plocha podlahy[m^2]
    AreaCeiling  : REAL := 0.0; // Plocha stropu[m^2]
  END_VAR
  VAR_TEMP
    DeltaTempWall_1     : REAL := 0; // Rozdíl 1[degC]
    DeltaTempWall_2     : REAL := 0; // Rozdíl 2[degC]
    DeltaTempWall_3     : REAL := 0; // Rozdíl 3[degC]
    DeltaTempWall_4     : REAL := 0; // Rozdíl 4[degC]
    DeltaTempFloor      : REAL := 0; // Rozdíl podlaha[degC]
    DeltaTempCeiling    : REAL := 0; // Rozdíl strop[degC]
    FiWall_1     : REAL := 0.0; // Tepelný tok 1[J/s]
    FiWall_2     : REAL := 0.0; // Tepelný tok 2[J/s]
    FiWall_3     : REAL := 0.0; // Tepelný tok 3[J/s]
    FiWall_4     : REAL := 0.0; // Tepelný tok 4[J/s]
    FiFloor      : REAL := 0.0; // Tepelný tok podlaha[J/s]
\end{lstlisting}
\pagebreak
\begin{lstlisting}[language=ST, breaklines=true, numbers=left, firstnumber=74, numberstyle=\small, numbersep=10pt, frame=single, basicstyle=\ttfamily\small]
    FiCeiling    : REAL := 0.0; // Tepelný tok strop[J/s]
  END_VAR
  VAR CONSTANT
    RoAir : REAL := 1.204; // Hustota vzduchu[Kg/m^3]
    CpAir : REAL := 1005.0; // Tepelná kap vzduchu[J/(kg*K)]
    LambdaBrick  : REAL := 0.4; // Tepelná vod cihly[W/(m*K)]
    MaxTemp      : REAL := 24.0; // Maximální teplota[degC]
    MinTemp      : REAL := 16.0; // Minimální teplota[degC]
    TimeRise     : REAL := 15.0; // Čas náběhu výkonu[s]
    TimeFallHeat : REAL := 15.0; // Čas klesání výkonu t[s]
    TimeFallCold : REAL := 10.0; // Čas klesání výkonu k[s]
    TargTimeHeat : REAL := 5.06; // Čas na dosažení 80% t[s]
    TargTimeCold : REAL := 3.29; // Čas na dosažení 80% k[s]
  END_VAR
IF NOT(INIT) THEN
   TimeStep := TaskTime / 1000.0; // ms => s
END_IF;

(* Výpočet objemu, hmotnosti, energie *)
IF NOT(INIT) THEN
  VAir := lenght * width * height;
  MAir := RoAir * VAir;
  QAir := MAir * CpAir;
END_IF;

(* Výpočet ploch *)
IF NOT(INIT) THEN
  AreaWall1 := height * width;
  AreaWall2 := height * width;
  AreaWall3 := height * lenght;
  AreaWall4 := height * lenght;
  AreaFloor := lenght * width;
  AreaCeiling := lenght * width;
END_IF;

(* Výpočet deltaT *)
DeltaTempWall_1 := wall_temp1 - RoomTemp;
DeltaTempWall_2 := wall_temp2 - RoomTemp;
DeltaTempWall_3 := wall_temp3 - RoomTemp;
DeltaTempWall_4 := wall_temp4 - RoomTemp;
\end{lstlisting}
\pagebreak
\begin{lstlisting}[language=ST, breaklines=true, numbers=left, firstnumber=114, numberstyle=\small, numbersep=10pt, frame=single, basicstyle=\ttfamily\small]
DeltaTempFloor := floor_temp - RoomTemp;
DeltaTempCeiling := ceiling_temp - RoomTemp;

(* Ošetření teploty *)
(* Výpočet tepelných toků přes stěny *)
FiWall_1 := (LambdaBrick * AreaWall1 * DeltaTempWall_1) / wall_thic1;
FiWall_2 := (LambdaBrick * AreaWall2 * DeltaTempWall_2) / wall_thic2;
FiWall_3 := (LambdaBrick * AreaWall3 * DeltaTempWall_3) / wall_thic3;
FiWall_4 := (LambdaBrick * AreaWall4 * DeltaTempWall_4) / wall_thic4;
FiFloor := (LambdaBrick * AreaFloor * DeltaTempFloor) / floor_thic;
FiCeiling := (LambdaBrick * AreaCeiling * DeltaTempCeiling)/ceiling_thic;

(* Výpočet korekčních členu *)
IF NOT(INIT) THEN
  KHeatRise := (TimeRise / TargTimeHeat) - 1.0;
  KColdRise := (TimeRise / TargTimeCold) - 1.0;
  KHeatFall1 := (LN(Heat_1_WATTS/Epsilon))/TimeFallHeat;
  KHeatFall2 := (LN(Heat_2_WATTS/Epsilon))/TimeFallHeat;
  KColdFall1 := (LN(Cold_1_WATTS/Epsilon))/TimeFallCold;
  KColdFall2 := (LN(Cold_2_WATTS/Epsilon))/TimeFallCold;
END_IF;

(* Tepelný výkon topení a klimatizace *)
(* Výpočet alfy *)
IF NOT(INIT) THEN
  AlphaHeat := LN(1 + KHeatRise * (TimeStep)) / LN(1 + KHeatRise * TimeRise);
  AlphaCold := LN(1 + KColdRise * (TimeStep)) / LN(1 + KColdRise * TimeRise);
END_IF;

(* Topení 1 *)
IF Heat_1 THEN
  FiHeatTmp1 := FiHeatTmp1 + AlphaHeat * (Heat_1_WATTS - FiHeatTmp1);
ELSE
  FiHeatTmp1 := FiHeatTmp1 * EXP(-KHeatFall1 * (TimeStep));
END_IF;
(* Topení 2 *)
IF Heat_2 THEN
  FiHeatTmp2 := FiHeatTmp2 + AlphaHeat * (Heat_2_WATTS - FiHeatTmp2);
\end{lstlisting}
\pagebreak
\begin{lstlisting}[language=ST, breaklines=true, numbers=left, firstnumber=152, numberstyle=\small, numbersep=10pt, frame=single, basicstyle=\ttfamily\small]
ELSE
  FiHeatTmp2 := FiHeatTmp2 * EXP(-KHeatFall2 * (TimeStep));
END_IF;
(* Klimatizace 1 *)
IF Cold_1 THEN
  FiColdTmp1 := FiColdTmp1 + AlphaCold * (Cold_1_WATTS - FiColdTmp1);
ELSE
  FiColdTmp1 := FiColdTmp1 * EXP(-KColdFall1 * (TimeStep));
END_IF;
(* Klimatizace 2 *)
IF Cold_2 THEN
  FiColdTmp2 := FiColdTmp2 + AlphaCold * (Cold_2_WATTS - FiColdTmp2);
ELSE
  FiColdTmp2 := FiColdTmp2 * EXP(-KColdFall2 * (TimeStep));
END_IF;

(* Suma výkonů *)
FiHeat := FiHeatTmp1 + FiHeatTmp2;
FiCold := FiColdTmp1 + FiColdTmp2;

(* Celkový tepelný tok *)
FiTotal := FiHeat - FiCold + FiWall_1 + FiWall_2 + FiWall_3 + FiWall_4 + FiFloor + FiCeiling;

(* Výpočet přírůstku teploty za task *)
DeltaTemp := (FiTotal / QAir) * TimeStep;
RoomTemp := RoomTemp + DeltaTemp;

(* Výstup *)
Temperature := RoomTemp;

(* NOT INIT *)
INIT := TRUE;
END_FUNCTION_BLOCK
\end{lstlisting}
\newpage
\subsection{Popis funkčního bloku fbBath}
\label{apend:fbBath}
\begin{lstlisting}[language=ST, breaklines=true, numbers=left, numberstyle=\small, numbersep=10pt, frame=single, basicstyle=\ttfamily\small, caption={fbBath}, label={lst:fbBath}]
  FUNCTION_BLOCK fbBath
  VAR_INPUT
    SV6_ON       : BOOl; //Vizu světlo 6 on
    SV6_OFF      : BOOl; //Vizu světlo 6 off
    SV6_KNX_FB   : BOOl; //KNX světlo 6 feedback
  END_VAR
  VAR_OUTPUT
    SV6          : BOOL;   //Vizualizace světla 6
    SV6_CMD      : DT_CMD_BOOL;   //Příkaz světla 6
    TemperBath   : REAL;   //Vizualizace + komunikace
  END_VAR
  VAR
    Time_s   : REAL := 0.0;
    fbSV6 : fbKNXVisuBool;
  END_VAR
  VAR CONSTANT
    PI       : REAL := 3.14159265;
    Freq     : REAL := 0.005; // perioda cca 3 minuty 20 sekund
    TaskTime : REAL := 400.0; // 250 ms
  END_VAR
\end{lstlisting}
\begin{figure}[!ht]
  \begin{center}
  \includegraphics[scale=0.55]{obrazky/fbBath.png}
  \end{center}
  \caption[Logika funkčního bloku fbBath]{Logika funkčního bloku fbBath}
  \label{fig:fbBath}
\end{figure}
\pagebreak
\newpage
\subsection{fbKitch}
\label{apend:fbKitch}
\begin{lstlisting}[language=ST, breaklines=true, numbers=left, numberstyle=\small, numbersep=10pt, frame=single, basicstyle=\ttfamily\small, caption={fbKitch}, label={lst:fbKitch}]
  FUNCTION_BLOCK fbKitch
  VAR_INPUT
    SV1_ON         : BOOl; //Vizu světlo 1 on
    SV1_OFF        : BOOl; //Vizu světlo 1 off
    SV1_KNX_FB     : BOOl; //KNX světlo 1 feedback,
    SV2_ON         : BOOl; //Vizu světlo 2 on
    SV2_OFF        : BOOl; //Vizu světlo 2 off
    SV2_KNX_FB     : BOOl; //KNX světlo 2 feedback
    Heater3_ON     : BOOl; //Vizu topení 3 on
    Heater3_OFF    : BOOl; //Vizu topení 3 off
    Heater3_KNX_FB : BOOl; //KNX klimatizace 3 feedback
    Climat3_ON     : BOOl; //Vizu klimatizace 3 on
    Climat3_OFF    : BOOl; //Vizu klimatizace 3 off
    Climat3_KNX_FB : BOOl; //KNX topení 3 feedback
    wall_temp1     : REAL; // Teplota koupelna[degC]
    wall_temp2     : REAL; // Teplota ven[degC]
    wall_temp3     : REAL; // Teplota ven[degC]
    wall_temp4     : REAL; // Teplota ven[degC]
    ceiling_temp   : REAL; // Teplota obyvak[degC]
  END_VAR
  VAR_OUTPUT
    SV1            : BOOL;   //Vizualizace světla 1
    SV2            : BOOL;   //Vizualizace světla 2
    Heater3        : BOOL;   //Vizualizace topení 3
    Climat3        : BOOL;   //Vizualizace klimatizace 3
    SV1_CMD        : DT_CMD_BOOL;   //Příkaz světla 1
    SV2_CMD        : DT_CMD_BOOL;   //Příkaz světla 2
    Heater3_CMD    : DT_CMD_BOOL;   //Příkaz topení 3
    Climat3_CMD    : DT_CMD_BOOL;   //Příkaz klimatizace 3
    TemperKitchen  : REAL;   //Vizualizace + komunikace
  END_VAR
  VAR
    fbSV1 : fbKNXVisuBool;
    fbSV2 : fbKNXVisuBool;
    fbHeater3 : fbKNXVisuBool;
    fbKitchMod : fbRoomTempMod;
    fbClimat3 : fbKNXVisuBool;
  END_VAR
\end{lstlisting}
\pagebreak
\begin{figure}[!ht]
  \begin{center}
  \includegraphics[scale=0.7]{obrazky/fbKitch.png}
  \end{center}
  \caption[Logika funkčního bloku fbKitch]{Logika funkčního bloku fbKitch}
  \label{fig:fbKitch}
\end{figure}

\pagebreak
\subsection{fbLivRoom}
\label{apend:fbLivRoom}
\begin{lstlisting}[language=ST, breaklines=true, numbers=left, numberstyle=\small, numbersep=10pt, frame=single, basicstyle=\ttfamily\small, caption={fbLivRoom}, label={lst:fbLivRoom}]
FUNCTION_BLOCK fbLivRoom
  VAR_INPUT
    SV3_ON          : BOOl; //Vizu světlo 3 on
    SV3_OFF         : BOOl; //Vizu světlo 3 off
    SV3_KNX_FB      : BOOl; //KNX světlo 3 feedback,
    SV4_ON          : BOOl; //Vizu světlo 4 on
    SV4_OFF         : BOOl; //Vizu světlo 4 off
    SV4_KNX_FB      : BOOl; //KNX světlo 4 feedback
    SV5_ON          : BOOl; //Vizu světlo 5 on
    SV5_OFF         : BOOl; //Vizu světlo 5 off
    SV5_KNX_FB      : BOOl; //KNX světlo 5 feedback
    Shut1_UP        : BOOl; //Vizu rolety 1 on
    Shut1_DW        : BOOl; //Vizu rolety 1 down
    Shut1_STEP_UP   : BOOl; //Vizu rolety 1 on krok
    Shut1_STEP_DW   : BOOl; //Vizu rolety 1 off krok
    Shut1_KNX_FB_UP : BOOl; //KNX rolety 1 feedback up
    Shut1_KNX_FB_DW : BOOl; //KNX rolety 1 feedback down
    Shut2_UP        : BOOl; //Vizu rolety 2 on
    Shut2_DW        : BOOl; //Vizu rolety 2 down
    Shut2_STEP_UP   : BOOl; //Vizu rolety 2 on krok
    Shut2_STEP_DW   : BOOl; //Vizu rolety 2 off krok
    Shut2_KNX_FB_UP : BOOl; //KNX rolety 2 feedback up
    Shut2_KNX_FB_DW : BOOl; //KNX rolety 2 feedback down
    Heater1_ON      : BOOl; //Vizu topení 1 on
    Heater1_OFF     : BOOl; //Vizu topení 1 off
    Heater1_KNX_FB  : BOOl; //KNX klimatizace 1 feedback
    Heater2_ON      : BOOl; //Vizu topení 2 on
    Heater2_OFF     : BOOl; //Vizu topení 2 off
    Heater2_KNX_FB  : BOOl; //KNX klimatizace 2 feedback
    Climat1_ON      : BOOl; //Vizu klimatizace 1 on
    Climat1_OFF     : BOOl; //Vizu klimatizace 1 off
    Climat1_KNX_FB  : BOOl; //KNX topení 1 feedback
    Climat2_ON      : BOOl; //Vizu klimatizace 2 on
    Climat2_OFF     : BOOl; //Vizu klimatizace 2 off
    Climat2_KNX_FB  : BOOl; //KNX topení 2 feedback
    wall_temp1      : REAL; // Teplota koupelna[degC]
    wall_temp2      : REAL; // Teplota ven[degC]
    wall_temp3      : REAL; // Teplota ven[degC]
\end{lstlisting}
\begin{lstlisting}[language=ST, breaklines=true, numbers=left, firstnumber=39, numberstyle=\small, numbersep=10pt, , frame=single, basicstyle=\ttfamily\small]
    wall_temp4      : REAL; // Teplota ven[degC]
    floor_temp      : REAL; // Teplota v místnosti pod[degC]
    ceiling_temp    : REAL; // Teplota obyvak[degC]
  END_VAR
  VAR_OUTPUT
    SV3          : BOOL;   //Vizualizace světla 3
    SV4          : BOOL;   //Vizualizace světla 4
    SV5          : BOOL;   //Vizualizace světla 5
    SV3_CMD      : DT_CMD_BOOL;   //Příkaz světla 3
    SV4_CMD      : DT_CMD_BOOL;   //Příkaz světla 4
    SV5_CMD      : DT_CMD_BOOL;   //Příkaz světla 5
    Heater1        : BOOL;   //Vizualizace topení 1
    Heater2        : BOOL;   //Vizualizace topení 2
    Heater1_CMD    : DT_CMD_BOOL;   //Příkaz topení 1
    Heater2_CMD    : DT_CMD_BOOL;   //Příkaz topení 2
    Climat1        : BOOL;   //Vizualizace klimatizace 1
    Climat2        : BOOL;   //Vizualizace klimatizace 2
    Climat1_CMD    : DT_CMD_BOOL;   //Příkaz klimatizace 1
    Climat2_CMD    : DT_CMD_BOOL;   //Příkaz klimatizace 2
    Shut1_UP_OUT   : BOOL;   //Vizualizace žaluzie 1 UP
    Shut1_DOWN_OUT : BOOL;   //Vizualizace žaluzie 1 DOWN
    Shut1_CMD      : DT_CMD_BOOL;   //Příkaz žaluzie 1
    Shut1_STEP_CMD : DT_CMD_BOOL;   //Příkaz žaluzie 1 KROK
    Shut2_UP_OUT   : BOOL;   //Vizualizace žaluzie 2 UP
    Shut2_DOWN_OUT : BOOL;   //Vizualizace žaluzie 2 DOWN
    Shut2_CMD      : DT_CMD_BOOL;   //Příkaz žaluzie 2
    Shut2_STEP_CMD : DT_CMD_BOOL;   //Příkaz žaluzie 2 KROK
    TemperLivingR  : REAL;   //Vizualizace + komunikace
  END_VAR
  VAR
    fbSV3 : fbKNXVisuBool;
    fbSV4 : fbKNXVisuBool;
    fbSV5 : fbKNXVisuBool;
    fbHeater1 : fbKNXVisuBool;
    fbClimat1 : fbKNXVisuBool;
    fbHeater2 : fbKNXVisuBool;
    fbClimat2 : fbKNXVisuBool;
    fbLivingRMod : fbRoomTempMod;
\end{lstlisting}
\pagebreak
\begin{lstlisting}[language=ST, breaklines=true, numbers=left, firstnumber=77, numberstyle=\small, numbersep=10pt, frame=single, basicstyle=\ttfamily\small]
    fbShut1 : fbKNXShutters;
    fbShut2 : fbKNXShutters;
  END_VAR
\end{lstlisting}
\begin{figure}[!ht]
  \begin{center}
  \includegraphics[scale=0.58]{obrazky/fbLivRoom.png}
  \end{center}
  \caption[Logika funkčního bloku fbLivRoom]{Logika funkčního bloku fbLivRoom}
  \label{fig:fbLivRoom}
\end{figure}
\pagebreak
\subsection{fbOutz}
\label{apend:fbOutz}
\begin{lstlisting}[language=ST, breaklines=true, numbers=left, numberstyle=\small, numbersep=10pt, frame=single, basicstyle=\ttfamily\small, caption={fbOutz}, label={lst:fbOutz}]
  FUNCTION_BLOCK fbOutz
  VAR_INPUT
    SV7_ON       : BOOl R_EDGE; //Vizu světlo 7 on
    SV7_OFF      : BOOl R_EDGE; //Vizu světlo 7 off
    SV7_KNX_FB   : BOOl; //KNX světlo 7 feedback
    KNX_OUT_TEMP : REAL; // KNX Teplota venku
  END_VAR
  VAR_OUTPUT
    SV7          : BOOL;   //Vizualizace světla 7
    SV7_CMD      : DT_CMD_BOOL;   //Příkaz světla 7
    TemperOut    : REAL;   //Vizualizace + komunikace
  END_VAR
  VAR
    fbSV7 : fbKNXVisuBool;
  END_VAR
\end{lstlisting}
\begin{figure}[!ht]
  \begin{center}
  \includegraphics[scale=1.0]{obrazky/fbOutz.png}
  \end{center}
  \caption[Logika funkčního bloku fbOutz]{Logika funkčního bloku fbOutz}
  \label{fig:fbOutz}
\end{figure}
\pagebreak
\chapter{Program komunikace mezi PLC a KNX}
\label{apend:KNXComm}
\begin{lstlisting}[language=ST, breaklines=true, numbers=left, numberstyle=\small, numbersep=10pt, frame=single, basicstyle=\ttfamily\small, caption={Program komunikace mezi PLC a KNX}, label={lst:prgKNXComm}]
  PROGRAM prgKNXComm
  VAR
    init : BOOL;
    knx  : fbKnxIpBaosBin;
    knxObjectList    : ARRAY[1..34] OF UDINT; // pole adres
    SHUT1_FB_PULSE : TON;
    SHUT2_FB_PULSE : TON;
    datapoint1       : T_KNX_OBJECT_DPT1;  // SV1_FB
    datapoint2       : T_KNX_OBJECT_DPT1;  // SV2_FB
    datapoint3       : T_KNX_OBJECT_DPT1;  // SV3_FB
    datapoint4       : T_KNX_OBJECT_DPT1;  // SV4_FB
    datapoint5       : T_KNX_OBJECT_DPT1;  // SV5_FB
    datapoint6       : T_KNX_OBJECT_DPT1;  // SV6_FB
    datapoint7       : T_KNX_OBJECT_DPT1;  // SV7_FB
    datapoint8       : T_KNX_OBJECT_DPT1;  // HEAT1_FB
    datapoint9       : T_KNX_OBJECT_DPT1;  // HEAT2_FB
    datapoint10      : T_KNX_OBJECT_DPT1;  // HEAT3_FB
    datapoint11      : T_KNX_OBJECT_DPT1;  // COLD1_FB
    datapoint12      : T_KNX_OBJECT_DPT1;  // COLD2_FB
    datapoint13      : T_KNX_OBJECT_DPT1;  // COLD3_FB
    datapoint14      : T_KNX_OBJECT_DPT1;  // SHUT1_FB
    datapoint15      : T_KNX_OBJECT_DPT1;  // SHUT1_CMD
    datapoint16      : T_KNX_OBJECT_DPT1;  // SHUT2_FB
    datapoint17      : T_KNX_OBJECT_DPT1;  // SHUT2_CMD
    datapoint18      : T_KNX_OBJECT_DPT1;  // SV1_CMD
    datapoint19      : T_KNX_OBJECT_DPT1;  // SV2_CMD
    datapoint20      : T_KNX_OBJECT_DPT1;  // SV3_CMD
    datapoint21      : T_KNX_OBJECT_DPT1;  // SV4_CMD
    datapoint22      : T_KNX_OBJECT_DPT1;  // SV5_CMD
    datapoint23      : T_KNX_OBJECT_DPT1;  // SV6_CMD
    datapoint24      : T_KNX_OBJECT_DPT1;  // SV7_CMD
    datapoint25      : T_KNX_OBJECT_DPT1;  // HEAT1_CMD
    datapoint26      : T_KNX_OBJECT_DPT1;  // HEAT2_CMD
    datapoint27      : T_KNX_OBJECT_DPT1;  // HEAT3_CMD
    datapoint28      : T_KNX_OBJECT_DPT1;  // COLD1_CMD
    datapoint29      : T_KNX_OBJECT_DPT1;  // COLD2_CMD
    datapoint30      : T_KNX_OBJECT_DPT1;  // COLD3_CMD
\end{lstlisting}
\pagebreak
\begin{lstlisting}[language=ST, breaklines=true, numbers=left, firstnumber=38, numberstyle=\small, numbersep=10pt, frame=single, basicstyle=\ttfamily\small]
    datapoint31      : T_KNX_OBJECT_DPT1;  // krok Ž 1
    datapoint32      : T_KNX_OBJECT_DPT1;  // krok Ž 2
    datapoint33      : T_KNX_OBJECT_DPT18; // scéna
    datapoint34      : T_KNX_OBJECT_DPT9;  // teplota
  END_VAR
IF NOT init THEN // pole adres
  knxObjectList[1]  := PTR_TO_UDINT( ADR(datapoint1));   
  knxObjectList[2]  := PTR_TO_UDINT( ADR(datapoint2));   
  knxObjectList[3]  := PTR_TO_UDINT( ADR(datapoint3));   
  knxObjectList[4]  := PTR_TO_UDINT( ADR(datapoint4));   
  knxObjectList[5]  := PTR_TO_UDINT( ADR(datapoint5));   
  knxObjectList[6]  := PTR_TO_UDINT( ADR(datapoint6));   
  knxObjectList[7]  := PTR_TO_UDINT( ADR(datapoint7));   
  knxObjectList[8]  := PTR_TO_UDINT( ADR(datapoint8));   
  knxObjectList[9]  := PTR_TO_UDINT( ADR(datapoint9));   
  knxObjectList[10] := PTR_TO_UDINT( ADR(datapoint10));  
  knxObjectList[11] := PTR_TO_UDINT( ADR(datapoint11));  
  knxObjectList[12] := PTR_TO_UDINT( ADR(datapoint12));  
  knxObjectList[13] := PTR_TO_UDINT( ADR(datapoint13));  
  knxObjectList[14] := PTR_TO_UDINT( ADR(datapoint14));  
  knxObjectList[15] := PTR_TO_UDINT( ADR(datapoint15));  
  knxObjectList[16] := PTR_TO_UDINT( ADR(datapoint16));  
  knxObjectList[17] := PTR_TO_UDINT( ADR(datapoint17));  
  knxObjectList[18] := PTR_TO_UDINT( ADR(datapoint18));  
  knxObjectList[19] := PTR_TO_UDINT( ADR(datapoint19));  
  knxObjectList[20] := PTR_TO_UDINT( ADR(datapoint20));  
  knxObjectList[21] := PTR_TO_UDINT( ADR(datapoint21));  
  knxObjectList[22] := PTR_TO_UDINT( ADR(datapoint22));  
  knxObjectList[23] := PTR_TO_UDINT( ADR(datapoint23));  
  knxObjectList[24] := PTR_TO_UDINT( ADR(datapoint24));  
  knxObjectList[25] := PTR_TO_UDINT( ADR(datapoint25));  
  knxObjectList[26] := PTR_TO_UDINT( ADR(datapoint26));  
  knxObjectList[27] := PTR_TO_UDINT( ADR(datapoint27));  
  knxObjectList[28] := PTR_TO_UDINT( ADR(datapoint28));  
  knxObjectList[29] := PTR_TO_UDINT( ADR(datapoint29));  
  knxObjectList[30] := PTR_TO_UDINT( ADR(datapoint30));    
  knxObjectList[31] := PTR_TO_UDINT( ADR(datapoint31));
  knxObjectList[32] := PTR_TO_UDINT( ADR(datapoint32));  
  knxObjectList[33] := PTR_TO_UDINT( ADR(datapoint33));    
  knxObjectList[34] := PTR_TO_UDINT( ADR(datapoint34));
\end{lstlisting}
\pagebreak
\begin{lstlisting}[language=ST, breaklines=true, numbers=left, firstnumber=78, numberstyle=\small, numbersep=10pt, frame=single, basicstyle=\ttfamily\small]   
  init := TRUE;
END_IF
knx( firstKnxObject := 1,
     lastKnxObject := 34,
     ethCode := ETH2_uni2,
     knxIP := STRING_TO_IPADR('192.168.xxx.xx'),
     knxList := void( knxObjectList));

// Feedbacky
SV1_FB       := datapoint1 .value;
SV2_FB       := datapoint2 .value;
SV3_FB       := datapoint3 .value;
SV4_FB       := datapoint4 .value;
SV5_FB       := datapoint5 .value;
SV6_FB       := datapoint6 .value;
SV7_FB       := datapoint7 .value;
HEAT1_FB     := datapoint8 .value;
HEAT2_FB     := datapoint9 .value;
HEAT3_FB     := datapoint10.value;
COLD1_FB     := datapoint11.value;
COLD2_FB     := datapoint12.value;
COLD3_FB     := datapoint13.value;
SHUT1_FB_PULSE(IN := datapoint14.altValue, PT := T#1s);
SHUT2_FB_PULSE(IN := datapoint16.altValue, PT := T#1s);
SHUT1_FB_UP   := datapoint14.value AND SHUT1_FB_PULSE.Q;
SHUT1_FB_DOWN := NOT(datapoint14.value) AND SHUT1_FB_PULSE.Q;
SHUT2_FB_UP   := datapoint16.value AND SHUT2_FB_PULSE.Q;
SHUT2_FB_DOWN := NOT(datapoint16.value) AND SHUT2_FB_PULSE.Q;

// Příkazy
IF SV1_CMD.CMD       THEN datapoint18.value := SV1_CMD.CMD_VAL; END_IF;
IF SV2_CMD.CMD       THEN datapoint19.value := SV2_CMD.CMD_VAL; END_IF;
IF SV3_CMD.CMD       THEN datapoint20.value := SV3_CMD.CMD_VAL; END_IF;
IF SV4_CMD.CMD       THEN datapoint21.value := SV4_CMD.CMD_VAL; END_IF;
IF SV5_CMD.CMD       THEN datapoint22.value := SV5_CMD.CMD_VAL; END_IF;
IF SV6_CMD.CMD       THEN datapoint23.value := SV6_CMD.CMD_VAL; END_IF;
IF SV7_CMD.CMD       THEN datapoint24.value := SV7_CMD.CMD_VAL; END_IF;
IF HEAT1_CMD.CMD     THEN datapoint25.value := HEAT1_CMD.CMD_VAL; END_IF;
IF HEAT2_CMD.CMD     THEN datapoint26.value := HEAT2_CMD.CMD_VAL; END_IF;
IF HEAT3_CMD.CMD     THEN datapoint27.value := HEAT3_CMD.CMD_VAL; END_IF;
\end{lstlisting}
\pagebreak
\begin{lstlisting}[language=ST, breaklines=true, numbers=left, firstnumber=118, numberstyle=\small, numbersep=10pt, frame=single, basicstyle=\ttfamily\small]
IF COLD1_CMD.CMD     THEN datapoint28.value := COLD1_CMD.CMD_VAL; END_IF;
IF COLD2_CMD.CMD     THEN datapoint29.value := COLD2_CMD.CMD_VAL; END_IF;
IF COLD3_CMD.CMD     THEN datapoint30.value := COLD3_CMD.CMD_VAL; END_IF;
IF SHUT1_CMD.CMD     THEN datapoint15.value  := SHUT1_CMD.CMD_VAL; END_IF;
IF SHUT1_STEP_CMD.CMD THEN datapoint31.value := SHUT1_STEP_CMD.CMD_VAL; END_IF;
IF SHUT2_CMD.CMD     THEN datapoint17.value  := SHUT1_CMD.CMD_VAL; END_IF;
IF SHUT2_STEP_CMD.CMD THEN datapoint32.value := SHUT1_STEP_CMD.CMD_VAL; END_IF;
// Shutdown scéna
IF SHUTDOWN_MQTT THEN
  datapoint33.control := TRUE;
  datapoint33.scene   := 5;
END_IF
// Posílání teploty
KNX_TEMPER := datapoint34.value;
END_PROGRAM
\end{lstlisting}
\pagebreak
\chapter{Program komunikace mezi PLC a Home Assistant - MQTT}
\label{apend:MQTTComm}
\begin{lstlisting}[language=ST, breaklines=true, numbers=left, numberstyle=\small, numbersep=10pt, frame=single, basicstyle=\ttfamily\small, caption={Program komunikace mezi PLC a Home Assistant \protect\textendash{} MQTT}, label={lst:prgMQTTComm}]
PROGRAM prgMQTTComm
  VAR
  Pub       : fbMQTTPublisher;
  Subs      : fbMQTTSubscriber;

  brokerIPaddr : STRING := '192.168.255.1';
  brokerPort   : UINT   := 1883;
  localPort    : UINT   := 60000;
  localPort2   : UINT   := 50000;
  keepAlive            : BOOL  := TRUE;
  keepAliveInterval    : TIME  := T#60s;
  pingInterval         : TIME  := T#10s;
  connTimeOut          : TIME  := T#5s;

  state : USINT := 0;
  connMessage : STRING := 'Connected';
  pubConnTopicMsg : STRING := 'plc/Connect/Publisher';
  subConnTopicMsg : STRING := 'plc/Connect/Subscriber';
  dataTopicMsg : STRING := 'plc/ALL_TEMPS';
  pubComParam   : T_MQTT_COM_PUB_PARAM := (
    pRetain := TRUE,
    qos     := 1,
    dup     := FALSE,
    clean   := FALSE
  );
  willParamPub  : T_MQTT_COM_WILL_PARAM := (
    wRetain := FALSE,
    topic   := 'plc/Connect/Publisher',
    mess    := 'Disconnected',
    flag    := TRUE,
    qos     := 1
  );
  subComParam : T_MQTT_COM_SUB_PARAM := (
    qos     := 1,
    clean   := FALSE
  );
\end{lstlisting}
\pagebreak
\begin{lstlisting}[language=ST, breaklines=true, numbers=left, firstnumber=37, numberstyle=\small, numbersep=10pt, frame=single, basicstyle=\ttfamily\small]
  willParamSub  : T_MQTT_COM_WILL_PARAM := (
    wRetain := FALSE,
    topic   := 'plc/Connect/Subscriber',
    mess    := 'Disconnected',
    flag    := TRUE,
    qos     := 1
  );

  jsonPayload       : STRING[255];
  sOut   : STRING[16];
  sKit   : STRING[16];
  sLiv   : STRING[16];
  sBath  : STRING[16];
  currentTopic : STRING[80];
  currentData : STRING[255];
  currentSendCom : BOOL;

  pubTimer : TON;
  stateTimer : TON;
  stateCooldown : TON;
  END_VAR
sOut := REAL_TO_STRINGF(OUTSIDE_TEMPER, '%.2f');
sKit := REAL_TO_STRINGF(KITCHEN_TEMPER, '%.2f');
sLiv := REAL_TO_STRINGF(LIVINGR_TEMPER, '%.2f');
sBath := REAL_TO_STRINGF(BATHROOM_TEMPER, '%.2f');
jsonPayload := CONCAT(
  '{',
    '"OUTSIDE_TEMPER": ',  sOut,  ',',
    '"KITCHEN_TEMPER": ',  sKit,  ',',
    '"LIVINGR_TEMPER": ',  sLiv,  ',',
    '"BATHROOM_TEMPER": ', sBath,
  '}'
);

CASE state OF
  0: // Inicializace
    currentTopic := '';
    currentData := '';
    currentSendCom := FALSE;
\end{lstlisting}
\pagebreak
\begin{lstlisting}[language=ST, breaklines=true, numbers=left, firstnumber=76, numberstyle=\small, numbersep=10pt, frame=single, basicstyle=\ttfamily\small]
    IF Pub.connected_to_broker THEN
      stateTimer(IN:=TRUE, PT:=T#5s);
      IF stateTimer.Q THEN
        currentTopic := pubConnTopicMsg;
        currentData := connMessage;
        currentSendCom := TRUE;
        stateTimer(IN:=FALSE);
        state := 1;
      END_IF;
    ELSE
      stateTimer(IN:=FALSE);
    END_IF;
    
  1: // Publisher Connected
    IF NOT Pub.busy AND NOT currentSendCom THEN
      stateCooldown(IN:=TRUE, PT:=T#3s);
      IF stateCooldown.Q THEN
        currentTopic := subConnTopicMsg;
        currentData := connMessage;
        currentSendCom := TRUE;
        stateCooldown(IN:=FALSE);
        state := 2;
      END_IF;
    END_IF;

  2: // Subs Connected
    IF NOT Pub.busy AND NOT currentSendCom THEN
      stateCooldown(IN:=TRUE, PT:=T#3s);
      IF stateCooldown.Q THEN
        stateCooldown(IN:=FALSE);
        state := 3;
        pubTimer(IN:=FALSE);
      END_IF;
    END_IF;

      3: // Normální provoz
    currentTopic := dataTopicMsg;
    currentData := jsonPayload;
    pubTimer(IN:=TRUE, PT:=T#5s);
\end{lstlisting}
\pagebreak
\begin{lstlisting}[language=ST, breaklines=true, numbers=left, firstnumber=115, numberstyle=\small, numbersep=10pt, frame=single, basicstyle=\ttfamily\small]
    IF pubTimer.Q AND NOT Pub.busy AND NOT currentSendCom THEN
      currentSendCom := TRUE;
      pubTimer(IN:=FALSE);
    END_IF;
    
    IF NOT Pub.connected_to_broker AND NOT Subs.connected_to_broker THEN
      state := 0;
    END_IF;
END_CASE;

Pub(
  chanCode          := ETH2_uni1,
  brokerIP          := STRING_TO_IPADR(brokerIPaddr),
  brokerPort        := brokerPort,
  localPort         := localPort,
  connect           := TRUE,
  keepAlive         := keepAlive,
  keepAliveInterval := keepAliveInterval,
  pingInterval      := pingInterval,
  connTimeOut       := connTimeOut,
  clientId_auto     := FALSE,
  clientId          := 'PLC_TECO_CP2007_PUB',
  comParam          := pubComParam,
  willParam         := willParamPub,
  topicTxt          := currentTopic,
  dataTxt           := currentData,
  sendCom           := currentSendCom
);

IF currentSendCom AND Pub.busy THEN
  currentSendCom := FALSE;
END_IF;

Subs(
  chanCode          := ETH2_uni0,
  brokerIP          := STRING_TO_IPADR(brokerIPaddr),
  brokerPort        := brokerPort,
  localPort         := localPort2,
  connect           := TRUE,
  keepAlive         := keepAlive,
\end{lstlisting}
\pagebreak
\begin{lstlisting}[language=ST, breaklines=true, numbers=left, firstnumber=155, numberstyle=\small, numbersep=10pt, frame=single, basicstyle=\ttfamily\small]
  keepAliveInterval := keepAliveInterval,
  pingInterval      := pingInterval,
  connTimeOut       := connTimeOut,
  clientId_auto     := FALSE,
  clientId          := 'PLC_TECO_CP2007_SUB',
  loginName         := '',
  loginPass         := '',
  comParam          := subComParam,
  willParam         := willParamSub,
  subRq             := Subs.connected_to_broker,
  unSubRq           := FALSE,
  topicTxt          := 'ha/SHUTDOWN'
);

IF Subs.dataRec THEN
  SHUTDOWN_MQTT := (Subs.dataTxt = 'true');
ELSE
  SHUTDOWN_MQTT := FALSE;
END_IF;
END_PROGRAM
\end{lstlisting}
\chapter{Docker YAML soubory}
\label{apend:dockeryaml}
\subsection{Portainer YAML}
\label{apend:portaineryaml}
\begin{lstlisting}[language=YAML, breaklines=true, numbers=left, numberstyle=\small, numbersep=10pt, frame=single, basicstyle=\ttfamily\small, caption={Portainer YAML}, label={lst:portaineryaml}]
services:
  portainer:
    image: portainer/portainer-ce:latest
    container_name: portainer
    ports:
      - 9000:9000
    volumes:
      - portainer_data:/data
      - /var/run/docker.sock:/var/run/docker.sock
    restart: unless-stopped
\end{lstlisting}
\pagebreak
\subsection{Stack YAML}
\label{apend:stackyaml}
\begin{lstlisting}[language=YAML, breaklines=true, numbers=left, numberstyle=\small, numbersep=10pt, frame=single, basicstyle=\ttfamily\small, caption={Stack YAML}, label={lst:stackyaml}]
services:
  home-assistant:
    container_name: home-assistant
    image: homeassistant/home-assistant:2025.4
    volumes:
      - /mnt/sda1/usb/config/home-assistant:/config
      - /etc/localtime:/etc/localtime:ro
      - /etc/timezone:/etc/timezone:ro
      - /run/dbus:/run/dbus:ro
    restart: unless-stopped
    privileged: true
    network_mode: host
  influxdb:
    container_name: influxdb
    image: influxdb:2.7.11
    ports:
      - 8086:8086
    restart: unless-stopped
    volumes:
      - /mnt/sda1/usb/data/influxdb:/var/lib/influxdb2
      - /mnt/sda1/usb/config/influxdb:/etc/influxdb2
      - /etc/localtime:/etc/localtime:ro
      - /etc/timezone:/etc/timezone:ro
    environment:
      - INFLUXDB_DB=db0
      - DOCKER_INFLUXDB_INIT_MODE=setup
      - DOCKER_INFLUXDB_INIT_USERNAME=${INFLUXDB_INIT_USERNAME}
      - DOCKER_INFLUXDB_INIT_PASSWORD=${INFLUXDB_INIT_PASSWORD}
      - DOCKER_INFLUXDB_INIT_ORG=fekt
      - DOCKER_INFLUXDB_INIT_BUCKET=kyblicek
  grafana:
    image: grafana/grafana-oss:11.6.0
    container_name: grafana
    user: "${UID_GRAFANA}:${GID_GRAFANA}"
    ports:
      - 3000:3000
    restart: unless-stopped
\end{lstlisting}
\pagebreak
\begin{lstlisting}[language=YAML, breaklines=true, numbers=left, firstnumber=38, numberstyle=\small, numbersep=10pt, frame=single, basicstyle=\ttfamily\small]
    volumes:
      - /mnt/sda1/usb/data/grafana:/var/lib/grafana
      - /etc/grafana/provisioning:/etc/grafana/provisioning
      - /etc/localtime:/etc/localtime:ro
      - /etc/timezone:/etc/timezone:ro
    depends_on:
      - influxdb
    environment:
      - GF_DATABASE_TIMEOUT=1000
      - GF_DATABASE_MAX_OPEN_CONN=2
      - GF_DATABASE_MAX_IDLE_CONN=2
  mosquitto:
    container_name: mosquitto
    image: eclipse-mosquitto:2.0.21
    restart: unless-stopped
    user: "${UID_MOSQUITTO}:${GID_MOSQUITTO}"
    ports:
      - "1883:1883"
      - "9001:9001"
    volumes:
      - /mnt/sda1/usb/config/mosquitto:/mosquitto/config
      - /mnt/sda1/usb/data/mosquitto:/mosquitto/data
      - /mnt/sda1/usb/logs/mosquitto:/mosquitto/log
\end{lstlisting}
\begin{figure}[!ht]
  \begin{center}
  \includegraphics[scale=0.4]{obrazky/portainer.png}
  \end{center}
  \caption[Portainer Stack]{Portainer Stack}
  \label{fig:portainer}
\end{figure}
\chapter{Home Assistant Konfigurační soubor}
\label{apend:configyaml}
\begin{lstlisting}[language=YAML, breaklines=true, numbers=left, numberstyle=\small, numbersep=10pt, frame=single, basicstyle=\ttfamily\small, caption={Home Assistant configuration.yaml}, label={lst:configyaml}]
# Loads default set of integrations. Do not remove.            
default_config:

# Load frontend themes from the themes folder                  
frontend:                                                      
  themes: !include_dir_merge_named themes

automation: !include automations.yaml
script: !include scripts.yaml                    
scene: !include scenes.yaml                                    

influxdb:
  api_version: 2           
  ssl: false
  host: localhost                                              
  port: 8086                                                   
  token: xxx
  organization: xxx                                
  bucket: kyblicek                                             
  tags:                                                        
    source: HA                                                 
  tags_attributes:                                              
    - friendly_name                                             
  default_measurement: units                                   

mqtt:
  sensor:
    - name: "Venek teplota"
      state_topic: "plc/ALL_TEMPS"
      value_template: "{{ value_json.OUTSIDE_TEMPER | float }}"
      device_class: "temperature"
      unit_of_measurement: "..C"
      unique_id: "tmp_out"
    - name: "Kuchyn teplota"
      state_topic: "plc/ALL_TEMPS"                             
      value_template: "{{ value_json.KITCHEN_TEMPER | float }}"
      device_class: "temperature"
\end{lstlisting}
\pagebreak
\begin{lstlisting}[language=YAML, breaklines=true, numbers=left, firstnumber=38, numberstyle=\small, numbersep=10pt, frame=single, basicstyle=\ttfamily\small]
      unit_of_measurement: "..C"                                
      unique_id: "tmp_kitch"                                   
    - name: "Obyvak teplota"                                   
      state_topic: "plc/ALL_TEMPS"                             
      value_template: "{{ value_json.LIVINGR_TEMPER | float }}" 
      device_class: "temperature"                               
      unit_of_measurement: "..C"                               
      unique_id: "tmp_liv"                                     
    - name: "Koupelna teplota"                                 
      state_topic: "plc/ALL_TEMPS"                             
      value_template: "{{ value_json.BATHROOM_TEMPER | float }}"
      device_class: "temperature"                              
      unit_of_measurement: "..C"                               
      unique_id: "tmp_bath"
  binary_sensor:           
    - name: "PLC MQTT Publisher"              
      state_topic: "plc/Connect/Publisher"         
      payload_on:  "Connected"                                 
      payload_off: "Disconnected"   
      device_class: connectivity
      unique_id: plc_mqtt_publisher
    - name: "PLC MQTT Subscriber"
      state_topic: "plc/Connect/Subscriber"                    
      payload_on:  "Connected"                                 
      payload_off: "Disconnected"
      device_class: connectivity
      unique_id: plc_mqtt_subscriber
  button:                                                      
    - name: "Central stop"                                     
      unique_id: "MQTT_Shutdown"                                
      command_topic: "ha/SHUTDOWN"                              
      payload_press: "true" 
\end{lstlisting}
\chapter{Nastavení přehledů v Grafaně}
\label{apend:grafana}
\subsection{Grafana - Spotřeba}
\begin{lstlisting}[language=flux, breaklines=true, numbers=left, numberstyle=\small, numbersep=10pt, frame=single, basicstyle=\ttfamily\small, caption={Grafana - Spotřeba klimatizace}, label={lst:grafanaKlimatizaceEnergie}]
from(bucket: "kyblicek")
  |> range(start: -1h)
  |> filter(fn: (r) => r["domain"] == "switch")
  |> filter(fn: (r) => r["_field"] == "value")
  |> filter(fn: (r) => r["friendly_name"] == "Klimatizace u televize" or r["friendly_name"] == "Leva zaluzie" or r["friendly_name"] == "Prava zaluzie")
  |> map(fn: (r) => ({ r with _value: float(v: r._value) * 1000.0 }))
  |> aggregateWindow(every: 1m, fn: last, createEmpty: true)
  |> fill(column: "_value", usePrevious: true)
  |> window(every: 1h, createEmpty: false)
  |> integral(unit: 1h)
  |> last()
  |> group(columns: [])
  |> sum(column: "_value")
  |> map(fn: (r) => ({ r with _value: r._value / 1000.0 }))
  |> yield(name: "Spotreba klimatizace za posledni hodinu (kWh)")
\end{lstlisting} 
\begin{lstlisting}[language=flux, breaklines=true, numbers=left, numberstyle=\small, numbersep=10pt, frame=single, basicstyle=\ttfamily\small, caption={Grafana - Spotřeba topení}, label={lst:grafanaTopeniEnergie}]
from(bucket: "kyblicek")
  |> range(start: -1h)
  |> filter(fn: (r) => r["domain"] == "switch")
  |> filter(fn: (r) => r["_field"] == "value")
  |> filter(fn: (r) => r["friendly_name"] == "Topeni kuchyne" or r["friendly_name"] == "Topeni pracovna" or r["friendly_name"] == "Topeni u televize")
  |> map(fn: (r) => ({ r with _value: float(v: r._value) * 4000.0 }))
  |> aggregateWindow(every: 1m, fn: last, createEmpty: true)
  |> fill(column: "_value", usePrevious: true)
  |> window(every: 1h, createEmpty: false)
  |> integral(unit: 1h)
  |> last()
  |> group(columns: [])
  |> sum(column: "_value")
  |> map(fn: (r) => ({ r with _value: r._value / 1000.0 }))
  |> yield(name: "Spotreba topeni za posledni hodinu (kWh)")
\end{lstlisting}
\pagebreak
\begin{lstlisting}[language=flux, breaklines=true, numbers=left, numberstyle=\small, numbersep=10pt, frame=single, basicstyle=\ttfamily\small, caption={Grafana - Spotřeba světla}, label={lst:grafanaSvetloEnergie}]
from(bucket: "kyblicek")
  |> range(start: -1h)
  |> filter(fn: (r) => r["domain"] == "light")
  |> filter(fn: (r) => r["_field"] == "value")
  |> map(fn: (r) => ({ r with _value: float(v: r._value) * 15.0 }))
  |> aggregateWindow(every: 1m, fn: last, createEmpty: true)
  |> fill(column: "_value", usePrevious: true)
  |> window(every: 1h, createEmpty: false)
  |> integral(unit: 1h)
  |> last()
  |> group(columns: [])
  |> sum(column: "_value")
  |> map(fn: (r) => ({ r with _value: r._value / 1000.0 }))
  |> yield(name: "Spotreba svetel za posledni hodinu (kWh)")
\end{lstlisting}
\begin{lstlisting}[language=flux, breaklines=true, numbers=left, numberstyle=\small, numbersep=10pt, frame=single, basicstyle=\ttfamily\small, caption={Grafana - Rozložení spotřeby}, label={lst:grafanaSpotreba}]
  from(bucket: "kyblicek")
  |> range(start: -1h)
  |> filter(fn: (r) => r["domain"] == "switch")
  |> filter(fn: (r) => r["_field"] == "value")
  |> filter(fn: (r) => r["friendly_name"] == "Klimatizace kuchyne" or r["friendly_name"] == "Klimatizace pracovna" or r["friendly_name"] == "Klimatizace u televize")
  |> map(fn: (r) => ({ r with _value: float(v: r._value) * 1000.0 }))
  |> aggregateWindow(every: 1m, fn: last, createEmpty: true)
  |> fill(column: "_value", usePrevious: true)
  |> window(every: 1h, createEmpty: false)
  |> integral(unit: 1h)
  |> last()
  |> group(columns: [])
  |> sum(column: "_value")
  |> map(fn: (r) => ({ r with _value: r._value / 1000.0 }))
  |> yield(name: "Spotreba klimatizace za posledni hodinu (kWh)")
from(bucket: "kyblicek")
  |> range(start: -1h)
  |> filter(fn: (r) => r["domain"] == "switch")
  |> filter(fn: (r) => r["_field"] == "value")
  |> filter(fn: (r) => r["friendly_name"] == "Leva zaluzie" or r["friendly_name"] == "Prava zaluzie")
  |> map(fn: (r) => ({ r with _value: float(v: r._value) * 150.0 }))
\end{lstlisting}
\pagebreak
\begin{lstlisting}[language=flux, breaklines=true, numbers=left, firstnumber=22, numberstyle=\small, numbersep=10pt, frame=single, basicstyle=\ttfamily\small]
  |> aggregateWindow(every: 1m, fn: last, createEmpty: true)
  |> fill(column: "_value", usePrevious: true)
  |> window(every: 1h, createEmpty: false)
  |> integral(unit: 1h)
  |> last()
  |> group(columns: [])
  |> sum(column: "_value")
  |> map(fn: (r) => ({ r with _value: r._value / 1000.0}))
  |> yield(name: "Spotreba zaluzii")
from(bucket: "kyblicek")
  |> range(start: -1h)
  |> filter(fn: (r) => r["domain"] == "switch")
  |> filter(fn: (r) => r["_field"] == "value")
  |> filter(fn: (r) => r["friendly_name"] == "Topeni kuchyne" or r["friendly_name"] == "Topeni pracovna" or r["friendly_name"] == "Topeni u televize")
  |> map(fn: (r) => ({ r with _value: float(v: r._value) * 4000.0 }))
  |> aggregateWindow(every: 1m, fn: last, createEmpty: true)
  |> fill(column: "_value", usePrevious: true)
  |> window(every: 1h, createEmpty: false)
  |> integral(unit: 1h)
  |> last()
  |> group(columns: [])
  |> sum(column: "_value")
  |> map(fn: (r) => ({ r with _value: r._value / 1000.0 }))
  |> yield(name: "Spotreba topeni za posledni hodinu (kWh)")
from(bucket: "kyblicek")
  |> range(start: -1h)
  |> filter(fn: (r) => r["domain"] == "light")
  |> filter(fn: (r) => r["_field"] == "value")
  |> map(fn: (r) => ({ r with _value: float(v: r._value) * 15.0 }))
  |> aggregateWindow(every: 1m, fn: last, createEmpty: true)
  |> fill(column: "_value", usePrevious: true)
  |> window(every: 1h, createEmpty: false)
  |> integral(unit: 1h)
  |> last()
  |> group(columns: [])
  |> sum(column: "_value")
  |> map(fn: (r) => ({ r with _value: r._value / 1000.0 }))
  |> yield(name: "Spotr;eba svetel za posledni hodinu (kWh)")
\end{lstlisting}
\pagebreak
\subsection{Grafana - Spínání}
\begin{lstlisting}[language=flux, breaklines=true, numbers=left, numberstyle=\small, numbersep=10pt, frame=single, basicstyle=\ttfamily\small, caption={Grafana - Spínání klimatizace}, label={lst:grafanaKlimatizace}]
from(bucket: "kyblicek")
  |> range(start: -24h)
  |> filter(fn: (r) => r["domain"] == "switch")
  |> filter(fn: (r) => r["_field"] == "value")
  |> filter(fn: (r) => r["friendly_name"] == "Klimatizace kuchyne" or r["friendly_name"] == "Klimatizace pracovna" or r["friendly_name"] == "Klimatizace u televize")
  |> aggregateWindow(every: 1m, fn: last, createEmpty: true)
  |> fill(column: "_value", usePrevious: true)
  |> yield(name: "Stav klimatizaci kazdou minutu")
\end{lstlisting}
\begin{lstlisting}[language=flux, breaklines=true, numbers=left, numberstyle=\small, numbersep=10pt, frame=single, basicstyle=\ttfamily\small, caption={Grafana - Spínání topení}, label={lst:grafanaTopeni}]
from(bucket: "kyblicek")
  |> range(start: -24h)
  |> filter(fn: (r) => r["domain"] == "switch")
  |> filter(fn: (r) => r["_field"] == "value")
  |> filter(fn: (r) => r["friendly_name"] == "Topeni kuchyne" or r["friendly_name"] == "Topeni pracovna" or r["friendly_name"] == "Topeni u televize")
  |> aggregateWindow(every: 1m, fn: last, createEmpty: true)
  |> fill(column: "_value", usePrevious: true)
  |> yield(name: "Stav topeni kazdou minutu")
\end{lstlisting}
\begin{lstlisting}[language=flux, breaklines=true, numbers=left, numberstyle=\small, numbersep=10pt, frame=single, basicstyle=\ttfamily\small, caption={Grafana - Spínání světel}, label={lst:grafanaSvetla}]
from(bucket: "kyblicek")
  |> range(start: -24h)
  |> filter(fn: (r) => r["domain"] == "light")
  |> filter(fn: (r) => r["_field"] == "value")
  |> aggregateWindow(every: 1m, fn: last, createEmpty: true)
  |> fill(column: "_value", usePrevious: true)
  |> yield(name: "Stav svetel kazdou minutu")
\end{lstlisting}
\pagebreak
\begin{lstlisting}[language=flux, breaklines=true, numbers=left, numberstyle=\small, numbersep=10pt, frame=single, basicstyle=\ttfamily\small, caption={Grafana - Spínání žaluzií}, label={lst:grafanaZaluzie}]
from(bucket: "kyblicek")
  |> range(start: -24h)
  |> filter(fn: (r) => r["domain"] == "switch")
  |> filter(fn: (r) => r["_field"] == "value")
  |> filter(fn: (r) => r["friendly_name"] == "Leva zaluzie" or r["friendly_name"] == "Prava zaluzie")
  |> aggregateWindow(every: 1m, fn: last, createEmpty: true)
  |> fill(column: "_value", usePrevious: true)
  |> yield(name: "Stav zaluzii kazdou minutu")
\end{lstlisting}
\pagebreak
\subsection{Grafana - Celkové měsíční náklady}
\begin{lstlisting}[language=flux, breaklines=true, numbers=left, numberstyle=\small, numbersep=10pt, frame=single, basicstyle=\ttfamily\small, caption={Grafana - Celkové měsíční náklady}, label={lst:grafanaCelkovyMesic}]
price = 3.2
klimatizace = from(bucket: "kyblicek")
  |> range(start: -30d)
  |> filter(fn: (r) => r["domain"] == "switch")
  |> filter(fn: (r) => r["_field"] == "value")
  |> filter(fn: (r) => r["friendly_name"] == "Klimatizace kuchyne" or r["friendly_name"] == "Klimatizace pracovna" or r["friendly_name"] == "Klimatizace u televize")
  |> map(fn: (r) => ({ r with _value: if r._value > 0.0 then 1.0 else 0.0 }))
  |> aggregateWindow(every: 1h, fn: mean, createEmpty: false)
  |> map(fn: (r) => ({ r with _value: r._value * 1.0 }))
  |> sum()
  |> map(fn: (r) => ({ r with _value: r._value * price, kategorie: "Klimatizace" }))
topeni = from(bucket: "kyblicek")
  |> range(start: -30d)
  |> filter(fn: (r) => r["domain"] == "switch")
  |> filter(fn: (r) => r["_field"] == "value")
  |> filter(fn: (r) => r["friendly_name"] == "Topeni kuchyne" or r["friendly_name"] == "Topeni pracovna" or r["friendly_name"] == "Topeni u televize")
  |> map(fn: (r) => ({ r with _value: if r._value > 0.0 then 1.0 else 0.0 }))
  |> aggregateWindow(every: 1h, fn: mean, createEmpty: false)
  |> map(fn: (r) => ({ r with _value: r._value * 4.0 }))
  |> sum()
  |> map(fn: (r) => ({ r with _value: r._value * price, kategorie: "Topeni" }))
zaluzie = from(bucket: "kyblicek")
  |> range(start: -30d)
  |> filter(fn: (r) => r["domain"] == "switch")
  |> filter(fn: (r) => r["_field"] == "value")
  |> filter(fn: (r) => r["friendly_name"] == "Leva zaluzie" or r["friendly_name"] == "Prava zaluzie")
  |> map(fn: (r) => ({ r with _value: if r._value > 0.0 then 1.0 else 0.0 }))
  |> aggregateWindow(every: 1h, fn: mean, createEmpty: false)
\end{lstlisting}
\pagebreak
\begin{lstlisting}[language=flux, breaklines=true, numbers=left, firstnumber=32, numberstyle=\small, numbersep=10pt, frame=single, basicstyle=\ttfamily\small]
  |> map(fn: (r) => ({ r with _value: r._value * 0.15 }))
  |> sum()
  |> map(fn: (r) => ({ r with _value: r._value * price, kategorie: "Zaluzie" }))
  svetla = from(bucket: "kyblicek")
  |> range(start: -30d)
  |> filter(fn: (r) => r["domain"] == "light")
  |> filter(fn: (r) => r["_field"] == "value")
  |> map(fn: (r) => ({ r with _value: if r._value > 0.0 then 1.0 else 0.0 }))
  |> aggregateWindow(every: 1h, fn: mean, createEmpty: false)
  |> map(fn: (r) => ({ r with _value: r._value * 0.015 }))
  |> sum()
  |> map(fn: (r) => ({ r with _value: r._value * price, kategorie: "Svetla" }))
  union(tables: [klimatizace, topeni, zaluzie, svetla]) // celkova sumu
  |> group()
  |> sum()
  |> map(fn: (r) => ({ r with typ: "Celkove naklady" }))
  |> yield(name: "Celkove mesicni naklady (Kc)")
\end{lstlisting}
\pagebreak
\subsection{Grafana - Průběh teploty}
\begin{lstlisting}[language=flux, breaklines=true, numbers=left, numberstyle=\small, numbersep=10pt, frame=single, basicstyle=\ttfamily\small, caption={Grafana - Průběh teploty - Query A}, label={lst:grafanaTeplota}]
from(bucket: "kyblicek")
  |> range(start: -1h)
  |> filter(fn: (r) => r["domain"] == "sensor")
  |> filter(fn: (r) => r["_field"] == "value")
  |> filter(fn: (r) => r["_measurement"] == "C")
  |> filter(fn: (r) => r["entity_id"] == "venek_teplota")
  |> aggregateWindow(every: 1m, fn: mean, createEmpty: false)
  |> yield(name: "Teplota venku")
\end{lstlisting}
\begin{lstlisting}[language=flux, breaklines=true, numbers=left, numberstyle=\small, numbersep=10pt, frame=single, basicstyle=\ttfamily\small, caption={Grafana - Průběh teploty - Query B}, label={lst:grafanaTeplota}]
from(bucket: "kyblicek")
  |> range(start: -1h)
  |> filter(fn: (r) => r["domain"] == "sensor")
  |> filter(fn: (r) => r["_field"] == "value")
  |> filter(fn: (r) => r["_measurement"] == "C")
  |> filter(fn: (r) => r["entity_id"] == "kuchyn_teplota")
  |> aggregateWindow(every: 1m, fn: mean, createEmpty: false)
  |> yield(name: "Teplota v kuchyni")
\end{lstlisting}
\begin{lstlisting}[language=flux, breaklines=true, numbers=left, numberstyle=\small, numbersep=10pt, frame=single, basicstyle=\ttfamily\small, caption={Grafana - Průběh teploty - Query C}, label={lst:grafanaTeplota}]
from(bucket: "kyblicek")
  |> range(start: -1h)
  |> filter(fn: (r) => r["domain"] == "sensor")
  |> filter(fn: (r) => r["_field"] == "value")
  |> filter(fn: (r) => r["_measurement"] == "C")
  |> filter(fn: (r) => r["entity_id"] == "koupelna_teplota")
  |> aggregateWindow(every: 1m, fn: mean, createEmpty: false)
  |> yield(name: "Teplota v koupelne")
\end{lstlisting}
\pagebreak
\begin{lstlisting}[language=flux, breaklines=true, numbers=left, numberstyle=\small, numbersep=10pt, frame=single, basicstyle=\ttfamily\small, caption={Grafana - Průběh teploty - Query D}, label={lst:grafanaTeplota}]
from(bucket: "kyblicek")
  |> range(start: -1h)
  |> filter(fn: (r) => r["domain"] == "sensor")
  |> filter(fn: (r) => r["_field"] == "value")
  |> filter(fn: (r) => r["_measurement"] == "C")
  |> filter(fn: (r) => r["entity_id"] == "obyvak_teplota")
  |> aggregateWindow(every: 1m, fn: mean, createEmpty: false)
  |> yield(name: "Teplota v obyvacim pokoji")
\end{lstlisting}
\begin{figure}[!ht]
  \begin{center}
  \includegraphics[scale=0.37]{obrazky/Grafana.png}
  \end{center}
  \caption[Vzhled přehledu v Grafaně]{Vzhled přehledu v Grafaně}
  \label{fig:GrafanaDashboard}
\end{figure}
\end{document}
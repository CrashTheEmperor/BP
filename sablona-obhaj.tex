% Soubory musí být v kódování, které je nastaveno v příkazu \usepackage[...]{inputenc}

\documentclass[%        Základní nastavení
  %draft,    				  % Testovací překlad
  12pt,       				% Velikost základního písma je 12 bodů
	t,                  % obsah slajdů bude vždy začínat od shora (nebude vertikálně centrovaný)
	aspectratio=1610,   % poměr stran bude 16:10 (všechny projektory v učebnách na Technické 12 Brno),
	                    % další volby jsou 43, 149, 169, 54, 32.
	unicode,						% Záložky a informace budou v kódování unicode
]{beamer}				    	% Dokument třídy 'zpráva', vhodná pro sazbu závěrečných prací s kapitolami
%\usepackage{etex}

\usepackage[utf8]		  % Kódování zdrojových souborů je v UTF-8
	{inputenc}					% Balíček pro nastavení kódování zdrojových souborů
	
\usepackage{graphicx} % Balíček 'graphicx' pro vkládání obrázků
											% Nutné pro vložení logotypů školy a fakulty

\usepackage[          % Balíček 'acronym' pro sazby zkratek a symbolů
	nohyperlinks				% Nebudou tvořeny hypertextové odkazy do seznamu zkratek
]{acronym}						
											% Nutné pro použití prostředí 'acronym' balíčku 'thesis'

%% Balíček hyperref je volán třídou beamer automaticky, proto není třeba následujícího kódu:
%\usepackage[
%	breaklinks=true,		% Hypertextové odkazy mohou obsahovat zalomení řádku
%	hypertexnames=false % Názvy hypertextových odkazů budou tvořeny
%											% nezávisle na názvech TeXu
%]{hyperref}						% Balíček 'hyperref' pro sazbu hypertextových odkazů
%											% Nutné pro použití příkazu 'nastavenipdf' balíčku 'thesis'

\usepackage{cmap} 		% Balíček cmap zajišťuje, že PDF vytvořené `pdflatexem' je
											% plně "prohledávatelné" a "kopírovatelné"

%\usepackage{upgreek}	% Balíček pro sazbu stojatých řeckých písmem
											%% např. stojaté pí: \uppi
											%% např. stojaté mí: \upmu (použitelné třeba v mikrometrech)
											%% pozor, grafická nekompatibilita s fonty typu Computer Modern!

%\usepackage{amsmath} %balíček pro sabu náročnější matematiky

\usepackage{booktabs} % Balíček, který umožňuje v tabulce používat
                      % příkazy \toprule, \midrule, \bottomrule


%%%%%%%%%%%%%%%%%%%%%%%%%%%%%%%%%%%%%%%%%%%%%%%%%%%%%%%%%%%%%%%%%
%%%%%%      Definice informací o dokumentu             %%%%%%%%%%
%%%%%%%%%%%%%%%%%%%%%%%%%%%%%%%%%%%%%%%%%%%%%%%%%%%%%%%%%%%%%%%%%

\input{nastaveni}      % v tomto souboru doplňte údaje o sobě, o názvu práce...
                       % (tento soubor je sdílený s textem práce)

%%%%%%%%%%%%%%%%%%%%%%%%%%%%%%%%%%%%%%%%%%%%%%%%%%%%%%%%%%%%%%%%%%%%%%%%

%%%%%%%%%%%%%%%%%%%%%%%%%%%%%%%%%%%%%%%%%%%%%%%%%%%%%%%%%%%%%%%%%%%%%%%%
%%%%%%     Nastavení polí ve Vlastnostech dokumentu PDF      %%%%%%%%%%%
%%%%%%%%%%%%%%%%%%%%%%%%%%%%%%%%%%%%%%%%%%%%%%%%%%%%%%%%%%%%%%%%%%%%%%%%
%% Při vloženém balíčku 'hyperref' lze použít příkaz '\pdfsettings'
\pdfsettings
%  Nastavení polí je možné provést také ručně příkazem:
%\hypersetup{
%  pdftitle={Název studentské práce},    	% Pole 'Document Title'
%  pdfauthor={Autor studenstké práce},   	% Pole 'Author'
%  pdfsubject={Typ práce}, 						  	% Pole 'Subject'
%  pdfkeywords={Klíčová slova}           	% Pole 'Keywords'
%}
\hypersetup{pdfpagemode=FullScreen}       % otevření rovnou v režimu celé obrazovky
%%%%%%%%%%%%%%%%%%%%%%%%%%%%%%%%%%%%%%%%%%%%%%%%%%%%%%%%%%%%%%%%%%%%%%%

\usetheme{VUT} 				% barvy a rozložení prezentace odpovídající VUT FEKT
% alternativně lze použít jiná berevná témata, ale bez záruky. Například: 
%\usetheme{Darmstadt} \usecolortheme{default2}
\logoheader					% vytvoření zkráceného loga VUT FEKT v hlavičce slajdu, nechte odkomentované



%\begin{document}

% v případě zakomentování následujícího se zobrazí v pravém dolním rohu slajdů klikatelné navigační symboly 
\disablenavigationsymbols

% titulní snímek, vysazen bez horních, dolních a postranních lišt (volba plain),
% není tak vysazen ani nadpis snímku
\maketitle

\begin{frame} 
	% nadpis snímku
	\frametitle{Obsah}
	\begin{itemize}
			\item Cíle práce
			\item Tvorba instalace
			\item Parametrizace
			\item Tvorba skupinových adres
			\item Výsledky
			\item Závěr
	\end{itemize}
\end{frame}

\begin{frame} 
	% nadpis snímku
	\frametitle{Cíle práce}
	\begin{itemize}
			\item Seznámení s technologií KNX.
			    \begin{itemize}
					\item Historie.
					\item Možnosti použití technologie.
					\item Sběrnicová instalace.
					\item Zabezpečení.
					\item Topologie.
				\end{itemize}
			\item Seznámení se společností Flowbox.
			\item Vytvoření programu pomocí softwaru ETS.
				\begin{itemize}
					\item Tvorba instalace.
					\item Parametrizace.
					\item Skupinové adresy.
				\end{itemize}
	\end{itemize}
\end{frame}

%%%%%%%%%%%%%
\begin{frame} 
	\frametitle{Tvorba instalace}
	\begin{itemize}
	    \item Kroky tvorby:
	        \begin{itemize}
	            \item Zvolení typu páteřní linie, skupinové adresy a topologie.
	            \item Segmentace budovy.
	            \item Výběr prvku z katalogu.
	        \end{itemize}
	\end{itemize}
			\begin{figure}%	
				\centering
				\vspace{0.2cm}	              % horizontální mezera
				\includegraphics[width=1\columnwidth]{obrazky/Katalog.png}
				\caption{Katalog}
				%\caption{Popisek obrázku}%
				%\label{obr:ukazka}
			\end{figure}
\end{frame}


\begin{frame} 
	\frametitle{Tvorba instalace}
	
	\begin{columns}[T] 								% prostředí sloupce s umístěním nahoře
		\begin{column}{0.35\textwidth}		% první sloupec
			Pracovní plocha znázorňuje:\\[2ex]
			\begin{itemize}
				\item Umístění.
				\item Individuální adresy.
				\item Aplikační program.
				\item Výrobce.
				\item Produkt.
			\end{itemize}
		\end{column}
		%
		\begin{column}{0.75\textwidth}		% druhý sloupec
			\begin{figure}%	
				\centering
				\vspace{1cm}	              % horizontální mezera
				\includegraphics[width=1\columnwidth]{obrazky/Přístroje v ETS.png}
				\caption{Přístroje v ETS}
				%\caption{Popisek obrázku}%
				%\label{obr:ukazka}
			\end{figure}
		\end{column}
	\end{columns}											% ukončení prostředí sloupce
\end{frame}

\begin{frame} 
	\frametitle{Parametrizace}
	\begin{itemize}
	    \item Výběr provedení.
	    \item Nastavování vnitřních senzorů.
	    \item Nastavování funkcí.
	    \item Nastavování scén.
	\end{itemize}
			\begin{figure}%	
				\centering
				\vspace{0.2cm}	              % horizontální mezera
				\includegraphics[width=.5\columnwidth]{obrazky/Ukazka parametrizace.png}
				\caption{Ukázka parametrizace}
				%\caption{Popisek obrázku}%
				%\label{obr:ukazka}
			\end{figure}
\end{frame}

\begin{frame} 
	\frametitle{Tvorba skupinových adres}
	\begin{itemize}
	    \item Kroky tvorby skupinových adres:	
	        \begin{itemize}
	            \item Vytvoření hlavní skupinové adresy.
	            \item Vytvoření střední skupinové adresy.
	            \item Vytvoření podskupinové adresy.
	            \item Vložení objektů z přístrojů.
	        \end{itemize}
	\end{itemize}

			\begin{figure}%	
				\centering
				\vspace{0.2cm}	              % horizontální mezera
				\includegraphics[width=1\columnwidth]{obrazky/Ukazka skupinovych adres.png}
				\caption{Ukázka skupinových adres}
				%\caption{Popisek obrázku}%
				%\label{obr:ukazka}
			\end{figure}
\end{frame}

%%%%%%%%%%%%%
\begin{frame} 
	\frametitle{Výsledky}
	\begin{columns}[T] 								% prostředí sloupce s umístěním nahoře
		\begin{column}{0.6\textwidth}		% první sloupec
	        \begin{itemize}
	            \item Hotová kapitola zaměřená na sběrnicový systém KNX.
	            \item Navázaný kontakt se společností Flowbox.
	            \item Hotová kapitola zaměřená na práci v ETS.
	            \item Funkční program.
	            \item Nenastavitelné tlačítko Berker.
	        \end{itemize}
	    \end{column}
	    
	\begin{column}{0.4\textwidth}		% druhý sloupec
			\begin{figure}%	
				\includegraphics[width=.8\columnwidth]{obrazky/IMG_20211217_121402.jpg}
				\caption{Nastavený panel}
				%\caption{Popisek obrázku}%
				%\label{obr:ukazka}
			\end{figure}
		\end{column}
	\end{columns}
\end{frame}


%%%%%%%%%%%%%
\begin{frame} 
	\frametitle{Závěr}
\begin{itemize}
    \item Pro bližší seznámení s problematikou nutno absolvovat školení KNX.
    \item Více, než polovina práce je hotova.
    \item Do dalšího vydání práce:
        \begin{itemize}
            \item Nutnost změny dynamických scén.
            \item Vytvoření komunikačního driveru ve spolupráci se společností Flowbox.
            \item Vytvoření vizualizace skrze server Flowbox.
        \end{itemize}
\end{itemize}
\end{frame}


% podekovani
\begin{frame}[c] 
% bez nadpisu snímku
	\frametitle{\mbox{ }}
	\begin{center}
		{\Huge Děkuji za pozornost!}
	\end{center}
\end{frame}

% otázky oponenta
%\frame{
%\frametitle{Otázky oponenta}
%	\emph{Jaká je souvislost Vašeho vzorce (1.2) s~Maxwellovými rovnicemi v~integrálním tvaru?}\\[2ex]
	%
%	Již staří Římané\,\dots
%}

\end{document}


% Soubory musí být v kódování, které je nastaveno v příkazu \usepackage[...]{inputenc}

\documentclass[%        Základní nastavení
  %draft,    				  % Testovací překlad
  12pt,       				% Velikost základního písma je 12 bodů
	t,                  % obsah slajdů bude vždy začínat od shora (nebude vertikálně centrovaný)
	aspectratio=1610,   % poměr stran bude 16:10 (všechny projektory v učebnách na Technické 12 Brno),
	                    % další volby jsou 43, 149, 169, 54, 32.
	unicode,						% Záložky a informace budou v kódování unicode
]{beamer}				    	% Dokument třídy 'zpráva', vhodná pro sazbu závěrečných prací s kapitolami
%\usepackage{etex}

\usepackage[utf8]		  % Kódování zdrojových souborů je v UTF-8
	{inputenc}					% Balíček pro nastavení kódování zdrojových souborů
	
\usepackage{graphicx} % Balíček 'graphicx' pro vkládání obrázků
											% Nutné pro vložení logotypů školy a fakulty

\usepackage[          % Balíček 'acronym' pro sazby zkratek a symbolů
	nohyperlinks				% Nebudou tvořeny hypertextové odkazy do seznamu zkratek
]{acronym}						
											% Nutné pro použití prostředí 'acronym' balíčku 'thesis'

%% Balíček hyperref je volán třídou beamer automaticky, proto není třeba následujícího kódu:
%\usepackage[
%	breaklinks=true,		% Hypertextové odkazy mohou obsahovat zalomení řádku
%	hypertexnames=false % Názvy hypertextových odkazů budou tvořeny
%											% nezávisle na názvech TeXu
%]{hyperref}						% Balíček 'hyperref' pro sazbu hypertextových odkazů
%											% Nutné pro použití příkazu 'nastavenipdf' balíčku 'thesis'

\usepackage{cmap} 		% Balíček cmap zajišťuje, že PDF vytvořené `pdflatexem' je
											% plně "prohledávatelné" a "kopírovatelné"

%\usepackage{upgreek}	% Balíček pro sazbu stojatých řeckých písmem
											%% např. stojaté pí: \uppi
											%% např. stojaté mí: \upmu (použitelné třeba v mikrometrech)
											%% pozor, grafická nekompatibilita s fonty typu Computer Modern!

%\usepackage{amsmath} %balíček pro sabu náročnější matematiky

\usepackage{booktabs} % Balíček, který umožňuje v tabulce používat
                      % příkazy \toprule, \midrule, \bottomrule


%%%%%%%%%%%%%%%%%%%%%%%%%%%%%%%%%%%%%%%%%%%%%%%%%%%%%%%%%%%%%%%%%
%%%%%%      Definice informací o dokumentu             %%%%%%%%%%
%%%%%%%%%%%%%%%%%%%%%%%%%%%%%%%%%%%%%%%%%%%%%%%%%%%%%%%%%%%%%%%%%

\input{nastaveni}      % v tomto souboru doplňte údaje o sobě, o názvu práce...
                       % (tento soubor je sdílený s textem práce)

%%%%%%%%%%%%%%%%%%%%%%%%%%%%%%%%%%%%%%%%%%%%%%%%%%%%%%%%%%%%%%%%%%%%%%%%

%%%%%%%%%%%%%%%%%%%%%%%%%%%%%%%%%%%%%%%%%%%%%%%%%%%%%%%%%%%%%%%%%%%%%%%%
%%%%%%     Nastavení polí ve Vlastnostech dokumentu PDF      %%%%%%%%%%%
%%%%%%%%%%%%%%%%%%%%%%%%%%%%%%%%%%%%%%%%%%%%%%%%%%%%%%%%%%%%%%%%%%%%%%%%
%% Při vloženém balíčku 'hyperref' lze použít příkaz '\pdfsettings'
\pdfsettings
%  Nastavení polí je možné provést také ručně příkazem:
%\hypersetup{
%  pdftitle={Název studentské práce},    	% Pole 'Document Title'
%  pdfauthor={Autor studenstké práce},   	% Pole 'Author'
%  pdfsubject={Typ práce}, 						  	% Pole 'Subject'
%  pdfkeywords={Klíčová slova}           	% Pole 'Keywords'
%}
\hypersetup{pdfpagemode=FullScreen}       % otevření rovnou v režimu celé obrazovky
%%%%%%%%%%%%%%%%%%%%%%%%%%%%%%%%%%%%%%%%%%%%%%%%%%%%%%%%%%%%%%%%%%%%%%%

\usetheme{VUT} 				% barvy a rozložení prezentace odpovídající VUT FEKT
% alternativně lze použít jiná berevná témata, ale bez záruky. Například: 
%\usetheme{Darmstadt} \usecolortheme{default2}
\logoheader					% vytvoření zkráceného loga VUT FEKT v hlavičce slajdu, nechte odkomentované



\begin{document}

% v případě zakomentování následujícího se zobrazí v pravém dolním rohu slajdů klikatelné navigační symboly 
\disablenavigationsymbols

% titulní snímek, vysazen bez horních, dolních a postranních lišt (volba plain),
% není tak vysazen ani nadpis snímku
\maketitle

\begin{frame} 
	% nadpis snímku
	\frametitle{Obsah}
	\begin{itemize}
			\item Cíle práce.
			\item Architektura.
			\item Tvorba instalace.
			\item Realizace ovládání a komunikace skrze PLC.
			\item Simulace teploty.
			\item WebMaker.
			\item Docker Compose.
			\item Vizualizace Home Assistant.
			\item Vizualizace Grafana.
			\item Výsledky.
			\item Závěr.
	\end{itemize}
\end{frame}

\begin{frame} 
	% nadpis snímku
	\frametitle{Cíle práce}
	\begin{itemize}
			\item Seznámení s technologií KNX.
			\begin{itemize}
				\item Historie, možnosti použití, sběrnicová instalace, zabezpečení a topologie.
			\end{itemize}
			\item Vytvoření programu pomocí softwaru ETS.
			\begin{itemize}
				\item Tvorba instalace, parametrizace a tvorba skupinových adres.
			\end{itemize}
			\item Vizualizace skrze PLC - CP-2007.
			\begin{itemize}
				\item Logiky, komunikace mezi PLC a panelem, komunikace mezi PLC a Raspberry Pi, vizualizace.
			\end{itemize}
			\item Vizualizace za použití open-source řešení - Raspberry Pi 5.
			\begin{itemize}
				\item Docker Compose, Portainer, Mosquitto, Home Assistant, InfluxDB a Grafana.
			\end{itemize}
	\end{itemize}
\end{frame}

\begin{frame} 
	% nadpis snímku
	\frametitle{Architektura}
			\begin{figure} [!ht]
				\centering
				%\vspace{0.2cm}	              % horizontální mezera
				\includegraphics[width=0.75\columnwidth]{obrazky/architecture.png}
				\caption{Architektura}
			\end{figure}
\end{frame}

\begin{frame} 
	\frametitle{Tvorba instalace}
	\begin{itemize}
		\item Kroky tvorby:
			\begin{itemize}
				\item Výběr topologie – volba přenosového média(TP, IP, RF, PL), páteřní linky a segmentace sítě.
				\item Výběr prvku z katalogu.
				\item Parametrizace.
				\item Vytvoření skupinových adres.
			\end{itemize}
	\end{itemize}
	\begin{figure} [!ht]	
		\centering
		\vspace{0.2cm}	              % horizontální mezera
		\includegraphics[width=0.8\columnwidth]{obrazky/Přístroje v ETS.png}
		\caption{Přístroje v ETS}
		%\caption{Popisek obrázku}%
		%\label{obr:ukazka}
	\end{figure}										% ukončení prostředí sloupce
\end{frame}

\begin{frame} 
	\frametitle{Realizace ovládání a komunikace skrze PLC}
	\begin{columns}
		\begin{column}{0.7\textwidth}
			
			\begin{itemize}
				\item Vytvoření logiky v PLC.
				\begin{itemize}
					\item Funkční bloky určený k ovládání.
					\item Funkční blok pro simulaci teploty.
					\item Funkční bloky realizující pokoje.
				\end{itemize}
				\item Komunikace mezi PLC a panelem - KNX/IP.
				\item Komunikace mezi PLC a Raspberry Pi - MQTT.
			\end{itemize}
		\end{column}
		\begin{column}{0.3\textwidth}
			\begin{figure} [!ht]	
				\centering
				\includegraphics[width=1\columnwidth]{obrazky/plc.png}
				\caption{PLC - CP-2007}
			\end{figure}		
		\end{column}
	\end{columns}
\end{frame}

\begin{frame} 
	\frametitle{Simulace teploty}
	\begin{itemize}
		\item Funkce růstvu v rekurzivním tvaru:
	\end{itemize}
	\begin{equation}
		y = y_{Posl} + \frac{ln(1 + k \cdot \Delta t)}{ln(1 + k \cdot T_{Max})} \cdot (y_{Max} - y_{Posl}) 
	\end{equation}
	\begin{itemize}
		\item Funkce poklesu v rekurzivním tvaru:
	\end{itemize}
	\begin{equation}
		y = y_{Posl} \cdot e^{-k \cdot \Delta t}
	\end{equation}		
\end{frame}

\begin{frame}
	\frametitle{Simulace teploty}
	\begin{figure} [!ht]	
		\centering              % horizontální mezera
		\includegraphics[width=0.65\columnwidth]{obrazky/simulace_teploty_kuchyne.pdf}
		\caption{Příklad simulace teploty}
	\end{figure}										% ukončení prostředí sloupce
\end{frame}

\begin{frame} 
	\frametitle{WebMaker}
	\begin{itemize}
		\item Tvorba webového rozhraní.
		\item Vytvoření přístupových údajů a přidělování práv.
		\item Propojení objektů s proměnnými.
	\end{itemize}
	\begin{columns}
		\begin{column}{0.5\textwidth}
			\begin{figure} [!ht]	
				\centering
				\includegraphics[width=1\columnwidth]{obrazky/Přehled.png}
				\caption{WebMaker - Přehled}
			\end{figure}	
		\end{column}
		\begin{column}{0.5\textwidth}
			\begin{figure} [!ht]	
				\centering
				\includegraphics[width=1\columnwidth]{obrazky/Obyvaci_pokoj.png}
				\caption{WebMaker - Obývací pokoj}
			\end{figure}
		\end{column}
	\end{columns}
\end{frame}

\begin{frame} 
	\frametitle{Docker Compose}
	\begin{columns}
		\begin{column}{0.6\textwidth}
			\begin{itemize}
				\item Izolace aplikací a jejich závislostí. % Izolace znamená, že každá aplikace běží v samostatném kontejneru a jejich závislosti jsou oddělené od ostatních aplikací.
				\item Snadné nasazení a škálování aplikací.
				\item Opakovatelné a konzistentní prostředí napříč vývojovými a produkčními servery.
				\item Jednoduchá správa více služeb v rámci jednoho projektu.
				\item Efektivní využití systémových prostředků.
			\end{itemize}
		\end{column}
		\begin{column}{0.4\textwidth}
			\begin{figure} [!ht]	
				\centering
				\includegraphics[width=1\columnwidth]{obrazky/stack.png}
				\caption{Docker Compose - Stack}
			\end{figure}
		\end{column}
	\end{columns}
\end{frame}

\begin{frame} 
	\frametitle{Vizualizace Home Assistant}
	\begin{itemize}
		\item Podpora široké škály zařízení a protokolů.
		\item Umožňuje vytvářet automatizace, scénáře a vizualizace.
		\item Komunita a ekosystém s množstvím integrací.
		\item Možnost přizpůsobení uživatelského rozhraní.
		\item Lze měřit a spravovat spotřebu energie(spotřebiče, síť, obnovitelné zdroje).
	\end{itemize}
	\begin{columns}
		\begin{column}{0.5\textwidth}
			\begin{figure} [!ht]	
				\centering
				\includegraphics[width=0.875\columnwidth]{obrazky/Dashboard2.png}
				\caption{Funkční vizualizace}
			\end{figure}
		\end{column}
		\begin{column}{0.5\textwidth}
			\begin{figure} [!ht]	
				\centering
				\includegraphics[width=0.9\columnwidth]{obrazky/Dashboard3.png}
				\caption{Vizualizace dle umístění}
			\end{figure}
		\end{column}
	\end{columns}
\end{frame}

\begin{frame} 
	\frametitle{Vizualizace Grafana}
	\begin{columns}
		\begin{column}{0.6\textwidth}
	\begin{itemize}
		\item Vizualizace a analýza dat.
		\item Podpora různých datových zdrojů (InfluxDB, Prometheus, MySQL).
		\item Umožňuje vytvářet interaktivní grafy a zobrazovače.
		\item Možnost sledování a upozorňení na události v reálném čase.
	\end{itemize}
		\end{column}
		\begin{column}{0.4\textwidth}
			\begin{figure} [!ht]	
				\centering              % horizontální mezera
				\includegraphics[width=0.9\columnwidth]{obrazky/Grafana.png}
				\caption{Grafana}
			\end{figure}	
		\end{column}
	\end{columns}
\end{frame}									% ukončení prostředí sloupce

%%%%%%%%%%%%%
\begin{frame} 
	\frametitle{Výsledky}
	\begin{columns}[T] 								% prostředí sloupce s umístěním nahoře
		\begin{column}{0.6\textwidth}		% první sloupec
			\begin{itemize}
				\item Funknčí instalace KNX.
				\item Otestované komunikace mezi jednotlivými částmi.
				\item Ovládání instalace z více míst. 
				\item Ukládání dat do databáze.
				\item Různé možnosti vizualizace.
				\item Nenastavitelné tlačítko Berker.
				\item Překročený počet zápisů na komunikační bráně IP Baos.
			\end{itemize}
		\end{column}
	\begin{column}{0.4\textwidth}		% druhý sloupec
			\begin{figure}%	
				\includegraphics[width=.8\columnwidth]{obrazky/IMG_20211217_121402.jpg}
				\caption{Nastavený panel}
				%\caption{Popisek obrázku}%
				%\label{obr:ukazka}
			\end{figure}
		\end{column}
	\end{columns}
\end{frame}


%%%%%%%%%%%%%
\begin{frame} 
	\frametitle{Závěr}
	\begin{itemize}
		\item Práce obsahuje:
			\begin{itemize}
				\item Základní informace o sběrnici KNX.
				\item Tvorbu instalace v softwaru ETS.
				\item Realizaci ovládání, komunikace a vizualizaci skrze PLC.
				\item Jiný přístup k vizualizaci pomocí open-source řešení.
			\end{itemize} 
		\item Přidaná hodnota práce:
			\begin{itemize}
				\item Popis komunikačního protokolu MQTT.
				\item Zvětšení povědomí o možnostech použití open-source platforem.
				\item Rozšíření informací obsažených v knihovnách PLC.
			\end{itemize}
	\end{itemize}
\end{frame}


% podekovani
\begin{frame}[c] 
% bez nadpisu snímku
	\frametitle{\mbox{ }}
	\begin{center}
		{\Huge Děkuji za pozornost!}
	\end{center}
\end{frame}

% otázky oponenta
\frame{
\frametitle{Otázky oponenta}
	\emph{Z jakého důvodu nebyl využit větší potenciál vizualizace v prostředí WebMaker (výhody, nevýhody, popis problémů v komunikaci)?}\\[2ex]
	\begin{itemize}
		\item Výhody:
		\begin{itemize}
			\item Průmyslové řešení.
			\item Složitější a přesnější operace skrze PLC (regulátory).
			\item Jednoducé ovládání.
	\end{itemize}
		\item Nevýhody:
		\begin{itemize}
			\item Jednoduché objekty.
			\item Malé rozlišení.
			\item Základní grafy.
			\item Stáří řešení.
		\end{itemize}
		\item Problémy v komunikaci:
		\begin{itemize}
			\item Porucha vnitřní paměti KNX IP BAOS 774 - EEPROM.
			\item Nedokáže z DNS dostat IP adresu.
			\item Nedokáže ukládat nastavení objektů.
		\end{itemize}
	\end{itemize}
}

\frame{
\frametitle{Otázky oponenta}
	\emph{Jaké jsou nevýhody, hlavně bezpečnostní, pří využití open source platforem pro vizualizaci a ovládání systému?}\\[2ex]
	\begin{itemize}
		\item Nevýhody:
		\begin{itemize}
			\item Závislost na komunitě.
			\item Nedostatek oficiální podpory.
			\item Možné problémy s kompatibilitou.
			\item Vyšší nároky na znalosti uživatele.
		\end{itemize}
		\item Bezpečnostní rizika:
		\begin{itemize}
			\item Výpadky napájení.
			\item Zabezpečení je závislé přímo na zdatnosti zhotovitele.
			\item Možnost chyb v dalších aktualizacích.
			\item Známé zranitelnosti v open-source software, které ještě nebyly opraveny.
			\item Nešifrované komunikace některých protokolů.
			\item Zadní vrátka v open-source software.
		\end{itemize}
	\end{itemize}
}

\end{document}


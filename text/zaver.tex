\chapter*{Závěr}
\phantomsection
\addcontentsline{toc}{chapter}{Závěr}

Úkolem této semestrální práce je seznámit se sběrnicovým systémem KNX, který jsem prostudoval a popsal v první kapitole. Kapitola začala krátkým úvodem obsahujícím informace o asociaci, podmínkami pro přijetí do asociace a informovala o existenci norem. V první podkapitole je představena historie asociace od vzniku až dodnes. Další podkapitola nastínila možnosti využití sběrnicového systému. Třetí podkapitola je obsáhlejší a popisuje sběrnicové přístroje. Nejdříve jejich funkcionalitu, poté vysvětluje problematiku adresování, které je nedílnou součástí správné komunikace po sběrnici. Další část vysvětluje, jak tato komunikace vlastně probíhá a jakou mají strukturu data, která na ní kolují. Poté následuje vysvětlení funkcionality datového bodu, který se používá všemi sběrnicovými přístroji KNX. Čtvrtá podkapitola se zabývá zabezpečením tohoto systému. Vysvětluje rozdíl mezi klasickými zařízeními a zabezpečenými zařízeními. Dále popisuje šifrování telegramů a končí vysvětlením režimu Secure Commissioning a funkce FDSK. Poslední podkapitola první části se zabývá topologií. Přesněji přibližuje základní rozdělení kabeláže na celky, individuální adresování v topologii, funkci spojek, funkci routingového čísla a končí vysvětlením funkce komunikačních rozhraní. Všechny tyto informace byly čerpány z materiálů školení poskytnutých samotnou asociací za účelem přípravy na školení.

Druhá kapitolo je zaměřená na společnost FLOWBOX, a to zejména na seznámení se společností a její platformou. Tahle kapitola bude při příštím vydání práce rozšířena.

Třetí kapitola se popisuje praktickou část semestrální práce. Začátek kapitoly je koncipován, jako seznámení s prostředím, které se zvolna změnilo na popis tvorby projektu od založí, přes instalaci, parametrizaci až po vytvoření skupinových adres. U jednoho přístroje bohužel nešlo i přes různé pokusy změnit parametry, nebo jakkoli přidat do skupinové adresy. Proto byl vyřazen z řešení této práce.
\chapter*{Závěr}
\phantomsection
\addcontentsline{toc}{chapter}{Závěr}

V rámci této bakalářské práce byly splněny všechny stanovené cíle týkající se návrhu a realizace vzdáleného řízení a vizualizace demonstračního panelu KNX pro ovládání funkcí osvětlení, žaluzií, topení a klimatizace. Práce poskytuje ucelený pohled na problematiku sběrnicového systému KNX, jeho možnosti a praktické využití v oblasti domácí automatizace. 

V první části práce byla provedena důkladná analýza technologie KNX, včetně její historie, základních principů fungování, struktury komunikace, zabezpečení a topologie. Tato část poskytla nezbytný teoretický základ pro následnou praktickou realizaci. 

Druhá část práce se zaměřila na parametrizaci konkrétních přístrojů a tvorbu dynamických světelných scén. Byly popsány jednotlivé kroky od návrhu projektu až po jeho implementaci, včetně identifikace problémů. 

Třetí část práce se věnovala samotnému řízení a vizualizaci prostřednictvím PLC společnosti TECO. Byly popsány nejen technické aspekty programování a komunikace mezi PLC a KNX, ale také implementace MQTT protokolu a tvorba webové vizualizace. V této části práce vznikl problém během odevzdání, kdy došlo k poškození komunikační brány, což mělo za následek následnou nefunkčnost komunikace mezi PLC a KNX instalací.

V poslední části byly zhodnoceny možnosti využití open-source platforem pro vizualizaci a správu domácí automatizace. Byla provedena implementace na jednodeskovém počítači Raspberry Pi s využitím kontejnerizace Docker a nasazením několika open-source nástrojů, jako jsou Home Assistant, InfluxDB či Grafana. Tyto nástroje byly vybrány a nasazeny s ohledem na jejich aktuální popularitu, dostupnost a možnosti rozšíření.

Celkově lze konstatovat, že práce splnila vytyčené cíle a přinesla komplexní řešení vzdáleného řízení a vizualizace KNX panelu. Byly ověřeny možnosti integrace různých technologií a platforem, přičemž důraz byl kladen na praktičnost, flexibilitu a budoucí rozšiřitelnost řešení. Výsledky práce mohou sloužit jako inspirace či základ pro další rozvoj v oblasti chytré domácnosti a automatizace budov.

\chapter*{Úvod}
\phantomsection
\addcontentsline{toc}{chapter}{Úvod}

S postupem času se rozšiřují možnosti využití elektrotechniky ve všech technických oblastech. Elektronika je dnes běžnou součástí většiny produktů, což se promítá i do moderních instalací. Oproti minulosti lze nyní ovládat nejen osvětlení, ale také topení, klimatizaci, žaluzie, zabezpečení budov, spotřebu energie a mnoho dalších funkcí. S rostoucí komplexností instalací dříve narůstalo i množství kabeláže, která sloužila převážně ke spínání jednotlivých spotřebičů. V současnosti lze tyto požadavky řešit pomocí napájecích zdrojů řízených sběrnicemi, což umožňuje automatizaci rozsáhlejších objektů, jako jsou hotely, nemocnice, vily nebo i továrny. Důležitým aspektem je také možnost vzdáleného řízení, které uživateli umožňuje ovládat celý systém odkudkoliv. Sběrnicové instalace se tak stávají stále populárnějším řešením, přičemž mezi nejrozšířenější standardy patří KNX, využívaný po celém světě.

Tato bakalářská práce si klade za cíl seznámit čtenáře se sběrnicovým systémem KNX, vybraným serverem pro řízení demonstračního panelu a tvorbou programu pro demonstraci funkcí tohoto systému. Panel bude využíván při různých akcích k prezentaci systému KNX, a to nejen prostřednictvím fyzických ovládacích prvků, ale také pomocí webového rozhraní programovatelného logického automatu nebo aplikace Home Assistant. Zároveň bude sloužit k prezentaci společností, které poskytly zařízení použité v panelu.

Teoretická část práce se zaměřuje na základy sběrnicového systému KNX, jeho historii, možnosti využití a principy fungování sběrnicových přístrojů včetně jejich adresace a komunikace. Dále je popsáno zabezpečení systému a topologie KNX. Následuje kapitola věnovaná použitému PLC, jeho knihovnám, komunikačnímu protokolu MQTT a vizualizačnímu serveru. Závěrem je popsána realizace vizualizace pomocí Raspberry Pi 5 s využitím kontejnerů (Portainer, Home Assistant, InfluxDB, Mosquitto, Grafana).

Praktická část práce se soustředí na tvorbu instalace v softwaru ETS, programování PLC a práci s Raspberry Pi. Nejprve je popsán software ETS a jeho možnosti, následně tvorba instalace od přidání zařízení do projektu, přes nastavení komunikace až po přiřazení skupinových adres. V části věnované PLC jsou popsány základní vytvořené funkční bloky, nastavení komunikace a realizace vizualizace prostřednictvím integrovaného webového serveru. Poslední kapitola se věnuje instalaci a nastavení Raspberry Pi, Dockeru a jednotlivých kontejnerů.
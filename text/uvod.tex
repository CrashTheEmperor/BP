\chapter*{Úvod}
\phantomsection
\addcontentsline{toc}{chapter}{Úvod}

S postupem času rostou možnosti využití elektrotechniky ve všech technických oblastech. Dnes již není probléme najít elektroniku ve většině produktů a z toho plyne, že se elektronika dostala i do moderních instalací. Oproti dobám minulým dokážeme ovládat nejen světla, ale i topení, klimatizaci, žaluzie, zabezpečení budovy, spotřebu a mnoho dalších činností s tím souvisejícím. Také rostou požadavky na komplexnost celé instalace, což znamenalo s jednoduchou instalací velké množství kabeláže, kterou šlo většinou jenom spínat různé spotřebiče. Dnes už je možno tyhle požadavky realizovat pomocí napájecích zdrojů řízených sběrnicemi. Tohle řízení nám dává možnost automatizovat větší komplexy například hotely, nemocnice, vily, nebo v některých případěc i samotné továrny. Další důležitý aspekt dnešní doby je vzdálené řízení, které umožní uživateli ovládat celý komplex, aniž by se musel dostavit na požadované místo. Z těchto informací je patrné, že sběrnicové instalace začínají být velice populární řešení, a to zejména sběrnicový standard KNX, který se používá celosvětově.

Tato semestrální práce má za cíl seznámit čtenáře, se sběrnicovým systémem KNX, zvoleným serverem pro řízení demonstračního panelu a vytvoření programu pro demonstraci funkcí sběrnicového systému KNX. Tento panel bude cestovat po různých akcích za účelem zvyšování povědomí o systému a demonstraci používání tohoto systému nejen pomocí fyzických ovládacích prvků, ale i za pomoci webového rozhraní serveru FLOWBOX. Dále bude využíván na prezentaci společností, které poskytly zařízení použité v panelu.

Teoretická část práce se zabývá základy sběrnicového systému KNX, včetně základních informací o asociaci a o společnosti FLOWBOX. První podkapitola systému KNX se zabývá historií asociace od vzniku, až doposud. Další podkapitola nese informace o možnostech použití technologie. Byly zde vybrány body, které jsou pro většinu uživatelů důležité a následně vysvětleny. Následující kapitola je poměrně obsáhlejší a vysvětluje základy sběrnicových přístrojů. To znamená, z čeho se zkládají, jak se adresují, jak komunikují a jaká data mezi sebou přenášejí. Předposlední podkapitola pojednává o zabezpečení, které je nutností pro klid většiny uživatelů. Poslední podkapitola vysvětluje topologii sběrnicového systému KNX, její adresování, funkci spojek, využití routingového čísla a popisuje využití komunikačních rozhraní v systému KNX. Další kapitola představuje společnost FLOWBOX, která poskytla server za účely vzdáleného řízení. Tato kapitola bude rozšířena o popis systému v příštím vydání.

Praktická část této práce se zaměřuje na tvorbu instalace v softwaru ETS. V první části se popisuje software a jeho možnosti. Druhá část popisuje tvorbu instalace od přidání do projektu, až po přiřazení skupinové adresy.